%%===========================================================%%
%%                                                           %%
%%                       FORWARD PROTONS                     %%
%%                                                           %%
%%===========================================================%%


\chapter{Forward protons}\label{chap:forwardProtons}

The main detector system used in our analyses is the Roman Pot setup in Phase II* configuration (Roman Pots in Phase I took data with STAR during special runs in last days of $p+p$ collisions in 2009, see e.g. Ref.~\cite{A_N,A_N_note,Sikora:2014hca}). It allowed to trigger on forward protons and reconstruct their momentum with high efficiency and precision.

\section{Detector layout}

\begin{figure}[h]\vspace*{-10pt}%
\centering\includegraphics[width=\linewidth]{graphics/rpSim/RP_phaseII.pdf}%
\caption[Schematic represention (top view) of the Roman Pot Phase II* at STAR.]{Schematic represention (top view) of the Roman Pot Phase II* at STAR (not to scale).}\label{fig:RPphaseII}%
\end{figure}

As presented in Figure~\ref{fig:RPphaseII} the Roman Pot Phase II* setup consists of detectors located in two stations on each side of the interaction point (IP) in a distance of 15.8~m and 17.6~m from the IP. Each station has two Roman Pots positioned vertically, one above and the other below the beamline (Fig.~\ref{fig:RpTrackDefinition}). Detectors are situated downstream the DX dipole magnets responsible for head-on targeting of the incoming beams and bending outgoing beams back into the accelerator pipeline. The constant and uniform magnetic field of the DX magnet works as a~spectrometer and thus knowledge on the track angle and position in the detector allows complete reconstruction of the proton momentum, including the fractional momentum loss $\xi$ (as described in~\cite{MomentumReco}). The naming convention of elements of RP setup is described in Ref.~\cite{Labeling}.

Single Roman Pot (the vessel, Fig.~\ref{fig:RPphoto}) houses a package of 4 silicon strip detector planes (Fig.~\ref{fig:SSDphoto}) - one pair of SSDs with vertical and one with horizontal orientation of the strips, and hence measurement of the position of a proton hit is possible in both transverse spatial coordinates, $x$ and $y$. The pitch (distance between neighbouring strips) in a single detector is 100~$\mu$m, therefore intrinsic spatial resolution is at the level of 30~$\mu$m. In addition to the silicon detectors, the package contains plastic scintillator that covers whole active area of the silicon, attached at the back. Two lightguides are glued at the top edge of scintillator which direct the light generated when ionizing particle passes through it to the photomultiplier tubes (PMTs) connected at the very end of each. This counter is used to trigger on forward protons and also provides the timing information.

\section{Roman Pot data reconstruction}

\subsection{Roman Pot track points and tracks}

Roman Pot data is stored in MuDST in StMuRpsCollection. This class contains objects reconstructed with St\_pp2pp\_Maker. The basic (low-level) data objects are the clusters characterized by their length (number of adjacent strips with signal above the threshold), energy (sum of ADC counts in each constituent strip) %
\begin{figure}[hb]%
\centering\includegraphics[width=\linewidth]{graphics/rpSim/trackDefinition.pdf}%
\caption[Side view of the Roman Pot Phase II* setup with an illustration of Roman Pot track point and track.]{Side view of the Roman Pot Phase II* setup (not to scale) with an illustration of Roman Pot track point and track.}\label{fig:RpTrackDefinition}%
\end{figure}%
and position. Vectors of clusters are provided independently for each silicon plane. Another low-level data are informations about time (TAC) and signal strength (ADC) for each PMT.

Our physics analyses utilized mainly the high-level objects which are the track points (StMuRpsTrackPoint) and tracks (StMuRpsTrack) stored in vector members of StMuRpsCollection. In short, these objects represent real particles (e.g. their momentum vector) in the same way as the TPC tracks represent particles traversing TPC. Concept of track point and track is depicted in Fig.~\ref{fig:RpTrackDefinition} and  described in some more details in Ref.~\cite{RpInStEvent}.

The algorithm for RP track reconstruction is implemented in St\_pp2pp\_Maker. It is a multi-track algorithm which first forms track points from clusters (there may be many track points in single RP), and then form tracks from all possible combinations of track points in Roman Pots in the same branch~\cite{RpRecoAlgo,RpInStEvent}. The track points and tracks are additionally tuned with Roman Pot ``afterburner'' package (StMuRpsUtil, ~\cite{RpAfterburner}) which recalculates positions of hits and momenta of tracks according to the final alignment corrections and known vertex position.

\subsection{Alignment}

\begin{figure}[b!]%
\centering\vspace{-7pt}%
\parbox{0.495\textwidth}{
  \centering
  \includegraphics[width=\linewidth,page=1]{graphics/rpSim/VxVy.pdf}\\
  \includegraphics[width=\linewidth,page=2]{graphics/rpSim/VxVy.pdf}
}~
\parbox{0.495\textwidth}{
  \centering
  \includegraphics[width=\linewidth,page=3]{graphics/rpSim/VxVy.pdf}\\
  \includegraphics[width=\linewidth,page=4]{graphics/rpSim/VxVy.pdf}
}\vspace{-5pt}%
\caption[Correlation between the hit position of constituent track point in the first RP station and the local angle of track in elastic scattering events.]%
{Correlation between the hit position of constituent track point in the first RP station ($y$-axis) and the local angle of track ($x$-axis) in elastic scattering events. The $y$-intercept has interpretation of the average position of the interaction vertex in given coordinate.}\label{fig:VxVy}%
\end{figure}


Precise knowledge of positioning of detectors in space is crucial for correct reconstruction of proton momentum. Therefore a process of detector alignment was done, which involved a few steps. At first, dedicated detector survey~\cite{surveyNote} was performed before the start of run 15, which provided initial calibration of the LVDTs installed in Roman Pot movement system. This was sufficient to know the positioning of detectors at a $\gtrsim$1~mm level. Next, the alignment analysis using elastic scattering events was done, as described in the analysis note of elastic proton-proton scattering~\cite{ElasticNote}. This analysis provided detectors alignment at a level of single pitch ($\sim 100\mu\text{m}$). In the last step determination of the average vertex position was done, as described below.

Vertex position is not necessary to correctly reconstruct forward proton observables in elastic scattering events  e.g. squarred four-momentum transfer $|t|$ - this is because one can use momentum balance constraint of elastically scattered protons (collinearity constraint) and calculate scattering angle from the straight line fit to all track points of east and west proton tracks, without knowledge where the interaction vertex was. The same approach cannot be used in other processes like Single or Central Diffraction, since there is only single forward proton (SD) or two forward protons are indepent in terms of scattering angles and momentum loss (CD, CEP).

This fact led to development of the method of extraction of the average vertex position using the elastic scattering data, as presented in Ref.~\cite{AverageVertex}. For this purpose RP\_ET triggers from randomly chosen runs (16085056 and 16085057) were used with single (exactly one) global RP track reconstructed on each side of the IP. Tracks were required to be collinear at 2$\sigma$ level. Selected clean sample of elastic scattering events was used to prepare the plots of the position of the track point in near RP station vs. the local angle of the RP track with respect to the global $z$-axis (Fig.~\ref{fig:VxVy}). The least squares fit of a line (with perpendicular offsets) to all data points in the scatterplot was performed. As a result four lines were obtained, one per arm per transverse coordinate. The slope of the line has interpretation of the distance from the nominal IP ($z=0$) to the $1^{\text{st}}$ RP station at 15.8~m. One can see that the slopes are well consistent with this value. The intercept of the line equals to the average position of the vertex in given coordinate. One finds that $\langle x\rangle_{\text{IP}}$ obtained from the fits to data points in two indepent elastic arms are perfectly consistent, while in $\langle y\rangle_{\text{IP}}$ parameters differ by 1.5 standard deviations. We conclude that extracted values of average positions of the vertex are trustworthy and we can average the numbers obtained from two indepent arms. As a results we use in our analyses numbers $\langle x\rangle_{\text{IP}} = 0.42$~mm, $\langle y\rangle_{\text{IP}} = 0.45$~mm, both in the reconstruction/recalculation of proton tracks with StMuRpsUtil package, and generation of MC events.

In Ref.~\cite{AverageVertex} the method of $\langle z\rangle_{\text{IP}}$ extraction is also presented, however the result (3 cm) is much smaller than event-by-event changes of the vertex position along $z$-axis (vertex spread is $\sigma(z_{\text{vtx}})=50$~cm). Such small offset has effectively no influence on the error on proton momentum reconstruction, therefore we neglect it.

Due to time variation of the beam conditions, automatic beam orbit corrections etc., the average position of the vertex may change from run to run. The measure of this variation is a by-product of Roman Pot alignment procedure which was done for every run. In Ref.~\cite{AverageVertexBogdan} the middle points of the track (MPTs) are plotted as a function of run number. One can see that MPTs in $x$ and $y$ are roughly constant along the entire data taking period in 2015 and consistent with numbers derived in Fig.~\ref{fig:VxVy}. Another cross-check for the corectness of extracted $\langle x\rangle_{\text{IP}}$ and $\langle y\rangle_{\text{IP}}$ was done with the use of elastic scattering MC simulation in Geant4. Several MC samples were generated with differently positioned vertex. The same procedure was performed as one described in this section and the output values of $\langle x\rangle_{\text{IP}}$ and $\langle y\rangle_{\text{IP}}$ were always consistent with the true values. Also the comparisons of the hit maps of elastic scattering protons were done between the data and MC, and the best matching was found for vertex generated at $\langle x\rangle_{\text{IP}} = 0.42$~mm and $\langle y\rangle_{\text{IP}} = 0.45$~mm (e.g.~\cite{AlignmentValidation}).

\section{Roman Pot simulation}

\subsection{Detector model}
%---------------------------
\begin{figure}[ht]
\centering
\parbox{0.2\textwidth}{
  \centering
  \includegraphics[height=145pt]{graphics/rpSim/romanpot.jpg}
  \caption[Roman Pot vessel (photo).]{Roman Pot vessel (photo).\newline}
  \label{fig:RPphoto}
}%
\quad%
\parbox{0.40\textwidth}{
  \centering
  \includegraphics[height=145pt]{graphics/rpSim/SSD.jpg}
  \caption[Silicon Strip Detector packages stored in the protective atmosphere (photo).]{Silicon Strip Detector packages stored in the protective atmosphere (photo).\newline}
  \label{fig:SSDphoto}
}%
\quad%
\parbox{0.33\textwidth}{
  \centering
  \includegraphics[height=145pt]{graphics/rpSim/g4Rp.png}
  \caption{Geant4 implementations of the Roman Pot vessel and SSD package with trigger counter.}
   \label{fig:g4Rp}
}
\end{figure}
%---------------------------

\subsection{Aperture tuning}

\begin{figure}%
\centering\includegraphics[width=\linewidth]{graphics/rpSim/geant4plot.png}%
\caption{View of the Geant4 implementation od the Roman Pot Phase II* detector setup.}\label{fig:geant4plot}%
\end{figure}

\begin{figure}[hb]%
\caption[Apertures.]{Apertures.}\label{fig:aperturesWithFit_Sample}%
\centering
\parbox{0.495\textwidth}{
  \centering
  \includegraphics[width=\linewidth,page=1]{graphics/rpSim/Apertures_swapedAxes_withFit_beforeDxShift.pdf}
}~
\parbox{0.495\textwidth}{
  \centering
  \includegraphics[width=\linewidth,page=1]{graphics/rpSim/Apertures_swapedAxes_withFit.pdf}
}%
\end{figure}

\subsection{Embedding technique}