%%===========================================================%%
%%                                                           %%
%%                       EFFICIENCIES                        %%
%%                                                           %%
%%===========================================================%%


\chapter{Efficiencies}\label{chap:efficiencies}

\section{TPC track acceptance and reconstruction efficiency}\label{sec:tpcAccAndEff}
We defined joint acceptance and efficiency of reconstruction of a track in the TPC, $\epsilon_{\textrm{\tiny TPC}}$, as the probability that particle from the primary interaction generates signal in the detector which is reconstructed as a global track that satisfies all quality criteria (cuts~\ref{sec:TpcQualityCuts}).
To derive this quantity the single particle STARsim MC embedded into zero-bias triggers was used.
\subsection{STAR nominal method}
Technically, the common method used by the STAR to obtain  $\epsilon_{\textrm{\tiny TPC}}$ is the following procedure:
\begin{enumerate}
	\item True-level primary particles of given ID and charge were selected ($set~A$).
	\item Each particle from $set~A$ was checked if at least one global TPC track with more than half of hit points generated by this particle was reconstructed (definition of true level particle-track matching, idTruth from StMuTrack collection). All particles from $set~A$ which have associated global track satisfying quality criteria (cut~\ref{sec:TpcQualityCuts}) formed $set~B$.
	\item The joint TPC acceptance and efficiency was calculated as the ratio of the histograms of true level quantities (such as $p_{T}$, $\eta$, $z_{\textrm{vtx}}$) for particles from $set~B$ and particles from $set~A$:
	\begin{equation}\label{eq:tpcAccAndEffDefinition}
	\epsilon_{\textrm{\tiny TPC}}\left(p_{T}, \eta, z_{vtx};~\textrm{sign},\textrm{PID}\right) = \frac{(p_{T},\eta, z_{vtx})~\textrm{histogram for particles of given sign and ID from}~set~B}{(p_{T},\eta, z_{vtx})~\textrm{histogram for particles of given sign and ID from}~set~A}.
	\end{equation}
	
\end{enumerate}
\subsubsection{Global track to true level particle matching}
It was found during the analysis that in some events there are more than one reconstructed global tracks matched with the same true level particle, which is shown in Fig.~\ref{fig:trackSplittingNominal}. 
\begin{figure}[ht]
	\centering
	\parbox{0.329\textwidth}{
		\centering
		\includegraphics[width=\linewidth,page=1]{graphics/eff/trackSplitting_CD.pdf}\\
		\includegraphics[width=\linewidth,page=4]{graphics/eff/trackSplitting_CD.pdf}\\
	}~
	\parbox{0.329\textwidth}{
		\centering
		\includegraphics[width=\linewidth,page=2]{graphics/eff/trackSplitting_CD.pdf}\\
		\includegraphics[width=\linewidth,page=5]{graphics/eff/trackSplitting_CD.pdf}\\
	}%
	\parbox{0.329\textwidth}{
		\centering
		\includegraphics[width=\linewidth,page=3]{graphics/eff/trackSplitting_CD.pdf}\\
		\includegraphics[width=\linewidth,page=6]{graphics/eff/trackSplitting_CD.pdf}\\
	}%
	\caption[Number of reconstructed global tracks, satisfying all quality criteria, matched with one true level primary particle.]{Number of reconstructed global tracks, satisfying all quality criteria (cuts~\ref{sec:TpcQualityCuts}), matched with one true level primary particle.}\label{fig:trackSplittingNominal}
\end{figure}

The true level particle end vertex $V_r^{end}$ is not specified if the particle neither interacted with the dead material nor decayed. The analysis showed that the reconstructed track is sometimes matched to the parent true level particle, e.g. with $V_{r}^{end}<48$~cm (in front of TPC), which means that not all children particles are stored in the StMuMcTrack collection. Also, there are problems in the closure tests, where  the  reconstructed-level distributions weighted by the nominal efficiency corrections do not describe the true level distributions.
The~distance 
\begin{equation}\label{eq:tpcMatchingDeltaSquare}
\delta^{2}\left(\eta,\phi\right)=\left(\eta^{true}-\eta^{reco}\right)^2+\left(\phi^{true}-\phi^{reco}\right)^2
\end{equation}
between the true level particle and global track assigned to it, shown in Fig.~\ref{fig:trackSplittingNominalDelta_1} for particles with only one  matched global track and Fig.~\ref{fig:trackSplittingNominalDelta_2} for particles with at least two  matched global tracks, indicates that some part of the tracks taken for the efficiency calculation are measured very bad ($\delta^{2}\left(\eta,\phi\right)$ is very large), even if there is only one global track matched to true-particle.

Additionally, in some cases reconstructed tracks have different PID than true level matched particles, which can be seen on the $dE/dx$ distribution in Fig.~\ref{fig:trackSplittingNominaldEdx}.   Since that, the new definition of true level particle-track matching was introduced.




\begin{figure}[hb]
	\centering
	\parbox{0.329\textwidth}{
		\centering
		\includegraphics[width=\linewidth,page=25]{graphics/eff/trackSplitting_CD.pdf}\\
		\includegraphics[width=\linewidth,page=28]{graphics/eff/trackSplitting_CD.pdf}\\
	}~
	\parbox{0.329\textwidth}{
		\centering
		\includegraphics[width=\linewidth,page=26]{graphics/eff/trackSplitting_CD.pdf}\\
		\includegraphics[width=\linewidth,page=29]{graphics/eff/trackSplitting_CD.pdf}\\
	}%
	\parbox{0.329\textwidth}{
		\centering
		\includegraphics[width=\linewidth,page=27]{graphics/eff/trackSplitting_CD.pdf}\\
		\includegraphics[width=\linewidth,page=30]{graphics/eff/trackSplitting_CD.pdf}\\
	}%
	\caption[$\delta^{2}\left(\eta,\phi\right)$ between true level particles and tracks assigned to them.]{$\delta^{2}\left(\eta,\phi\right)$ between true level particles and tracks assigned to them. Only true level particles with only one reconstructed track matched to them were selected.}\label{fig:trackSplittingNominalDelta_1}
\end{figure}

\begin{figure}[hb]
	\centering
	\parbox{0.329\textwidth}{
		\centering
		\includegraphics[width=\linewidth,page=19]{graphics/eff/trackSplitting_CD.pdf}\\
		\includegraphics[width=\linewidth,page=22]{graphics/eff/trackSplitting_CD.pdf}\\
	}~
	\parbox{0.329\textwidth}{
		\centering
		\includegraphics[width=\linewidth,page=20]{graphics/eff/trackSplitting_CD.pdf}\\
		\includegraphics[width=\linewidth,page=23]{graphics/eff/trackSplitting_CD.pdf}\\
	}%
	\parbox{0.329\textwidth}{
		\centering
		\includegraphics[width=\linewidth,page=21]{graphics/eff/trackSplitting_CD.pdf}\\
		\includegraphics[width=\linewidth,page=24]{graphics/eff/trackSplitting_CD.pdf}\\
	}%
	\caption[$\delta^{2}\left(\eta,\phi\right)$ between true level particles and tracks assigned to them.]{$\delta^{2}\left(\eta,\phi\right)$ between true level particles and tracks assigned to them. Only true level particles with at least two reconstructed tracks matched to them were selected.}\label{fig:trackSplittingNominalDelta_2}
\end{figure}
\begin{figure}[ht]
	\centering
	\parbox{0.329\textwidth}{
		\centering
		\includegraphics[width=\linewidth,page=31]{graphics/eff/trackSplitting_CD.pdf}\\
		\includegraphics[width=\linewidth,page=34]{graphics/eff/trackSplitting_CD.pdf}\\
	}~
	\parbox{0.329\textwidth}{
		\centering
		\includegraphics[width=\linewidth,page=32]{graphics/eff/trackSplitting_CD.pdf}\\
		\includegraphics[width=\linewidth,page=35]{graphics/eff/trackSplitting_CD.pdf}\\
	}%
	\parbox{0.329\textwidth}{
		\centering
		\includegraphics[width=\linewidth,page=33]{graphics/eff/trackSplitting_CD.pdf}\\
		\includegraphics[width=\linewidth,page=36]{graphics/eff/trackSplitting_CD.pdf}\\
	}%
	\caption[$dE/dx$ of the closest track matched to true level particle.]{$dE/dx$ of the closest track matched to true level particle. Lines indicate Bichsel function prediction for each particle species. Only tracks matched to true level particles without end vertex  are shown.}\label{fig:trackSplittingNominaldEdx}
\end{figure}
\subsection{Method used in this analysis}\label{subsec:definitionTrueLevelMatching}
In this method, the definition of true level particle-track matching  is expanded. In addition to the requirement of the appropriate number of common hit points, the distance between true level particle and track is required to be smaller than $0.05$, $\delta^{2}\left(\eta,\phi\right)<0.0025$. Tracks, which do not satisfy this cut, are treated as background (even if they are matched to the true level particle in the standard way). In almost all cases, where the $\delta^{2}\left(\eta,\phi\right)<0.0025$, there is only one track matched to true level particle (Fig.~\ref{fig:trackSplittingetaPhi}). Additionally, the $dE/dx$ of the track is consistent with the true level PID (Fig.~\ref{fig:trackSplittingEtaPhidEdx}). Since that, the TPC acceptance and  reconstruction efficiency calculated with the extended matching definition is used as nominal. The sample TPC efficiency plot is shown in Fig.~\ref{fig:tpcEff_pion_sample}. All remaining TPC efficiency plots are contained in Appendix~\ref{appendix:tpcEff}. Figure~\ref{fig:trackTPCefficiencyComparisonEtaPhi} shows the difference between TPC efficiencies obtained with the standard and the extended definition of true particle-track matching.
\begin{figure}[ht]
	\centering
	\parbox{0.329\textwidth}{
		\centering
		\includegraphics[width=\linewidth,page=1]{graphics/eff/trackSplitting_QualityEtaPhiCD.pdf}\\
		\includegraphics[width=\linewidth,page=4]{graphics/eff/trackSplitting_QualityEtaPhiCD.pdf}\\
	}~
	\parbox{0.329\textwidth}{
		\centering
		\includegraphics[width=\linewidth,page=2]{graphics/eff/trackSplitting_QualityEtaPhiCD.pdf}\\
		\includegraphics[width=\linewidth,page=5]{graphics/eff/trackSplitting_QualityEtaPhiCD.pdf}\\
	}%
	\parbox{0.329\textwidth}{
		\centering
		\includegraphics[width=\linewidth,page=3]{graphics/eff/trackSplitting_QualityEtaPhiCD.pdf}\\
		\includegraphics[width=\linewidth,page=6]{graphics/eff/trackSplitting_QualityEtaPhiCD.pdf}\\
	}%
	\caption[Number of reconstructed global tracks, satisfying all quality criteria and $\delta^{2}\left(\eta,\phi\right)$ cut, matched with one true level primary particle.]{Number of reconstructed global tracks, satisfying all quality criteria (cuts~\ref{sec:TpcQualityCuts}) and $\delta^{2}\left(\eta,\phi\right)$ cut, matched with one true level primary particle.}\label{fig:trackSplittingetaPhi}
\end{figure}

\begin{figure}[ht]
	\centering
	\parbox{0.329\textwidth}{
		\centering
		\includegraphics[width=\linewidth,page=21]{graphics/eff/trackSplitting_QualityEtaPhiCD.pdf}\\
		\includegraphics[width=\linewidth,page=24]{graphics/eff/trackSplitting_QualityEtaPhiCD.pdf}\\
	}~
	\parbox{0.329\textwidth}{
		\centering
		\includegraphics[width=\linewidth,page=22]{graphics/eff/trackSplitting_QualityEtaPhiCD.pdf}\\
		\includegraphics[width=\linewidth,page=25]{graphics/eff/trackSplitting_QualityEtaPhiCD.pdf}\\
	}%
	\parbox{0.329\textwidth}{
		\centering
		\includegraphics[width=\linewidth,page=23]{graphics/eff/trackSplitting_QualityEtaPhiCD.pdf}\\
		\includegraphics[width=\linewidth,page=26]{graphics/eff/trackSplitting_QualityEtaPhiCD.pdf}\\
	}%
	\caption[$dE/dx$ of the closest track matched to true level particle passing the $\delta^{2}\left(\eta,\phi\right)$ cut.]{$dE/dx$ of the closest track matched to true level particle passing the $\delta^{2}\left(\eta,\phi\right)$ cut. Lines indicate Bichsel function prediction for each particle species. Only tracks matched to true level particles without end vertex  are shown.}\label{fig:trackSplittingEtaPhidEdx}
\end{figure}


\subsection{Sample of  efficiency plots}\label{subsec:sampleTpcEffPlots}

In Figure~\ref{fig:tpcEff_pion_sample} we present sample plots of the TPC track and reconstruction efficiency calculated with modified definition of reconstructed track and true-level particle matching (according to description in Sec.~\ref{subsec:definitionTrueLevelMatching}), used in our analyses. Plots for all analyzed particle types and all bins of true $z_{\text{vtx}}$ are contained in Appendix~\ref{appendix:tpcEff}.

In order to maximize the statistics available for the measurement (possibly wide range of accepted longitudinal vertex position $z_{\text{vtx}}$) with maximized probed phase-space in analyzed physics processes (wide range of track $p_{T}$ and $\eta$) and minimized systematic uncertainties related to the central detector (TPC and TOF), we have studied the efficiency plots like ones shown in Fig.~\ref{fig:tpcEff_pion_sample} and Fig.~\ref{fig:tofEff_pion_sample}. We thus decided to set the cut on $z_{\text{vtx}}$ at $\pm80~\text{cm}$, which corresponds to 89\% of the full integral of normal distribution with mean at 0 and standard deviation of 50~cm. At the same time we set the cuts on track $p_{T}$ and $\eta$ as listed in Sec.~\ref{sec:TpcKinematicCuts}. These cuts are represented with red dashed lines in Fig.~\ref{fig:tpcEff_pion_sample} and Fig.~\ref{fig:tofEff_pion_sample}. Our goal was to operate within cuboid ($z_{\text{vtx}}$, $p_{T}$, $\eta$) region of relatively high TPC and TOF efficiency ($\geq50\%$ of the maximum value). In other words, we required high acceptance and efficiency for a rectangular ($p_{T}$, $\eta$) space with limits independent from $z_{\text{vtx}}$. One can see that the red lines in Fig.~\ref{fig:tpcEff_pion_sample} and Fig.~\ref{fig:tofEff_pion_sample} always contain in their interior the region of relatively high acceptance.

%---------------------------
\begin{figure}[h!]%\vspace{-10pt}
	\centering
	\parbox{0.485\textwidth}{
		\centering
		\begin{subfigure}[b]{\linewidth}
			\subcaptionbox{\label{fig:tpcEff_pion_sample_a}}{\includegraphics[width=\linewidth,page=3]{graphics/eff/Eff2D_TPC_pion_Minus.pdf}\vspace*{-8pt}}
		\end{subfigure}\\[5pt]
		\begin{subfigure}[b]{\linewidth}\addtocounter{subfigure}{1}
			\subcaptionbox{\label{fig:tpcEff_pion_sample_c}}{\includegraphics[width=\linewidth,page=18]{graphics/eff/Eff2D_TPC_pion_Minus.pdf}\vspace*{-8pt}}
		\end{subfigure}
	}%
	\quad%
	\parbox{0.485\textwidth}{
		\centering
		\begin{subfigure}[b]{\linewidth}\addtocounter{subfigure}{-2}
			\subcaptionbox{\label{fig:tpcEff_pion_sample_b}}{\includegraphics[width=\linewidth,page=11]{graphics/eff/Eff2D_TPC_pion_Minus.pdf}\vspace*{-8pt}}
		\end{subfigure}\\[5pt]
		\begin{minipage}[t][0.78\linewidth][t]{\linewidth}\vspace{10pt}
		\caption[Sample TPC acceptance and reconstruction efficiency of $\pi^{-}$.]{Sample TPC acceptance and reconstruction efficiency of $\pi^{-}$ in 3 bins of true $z_{\text{vtx}}$. Plots represents the TPC efficiency $\epsilon_{\text{TPC}}$ ($z$-axis) as a function of true particle pseudorapidity $\eta$ ($x$-axis) and transverse momentum $p_{T}$ ($y$-axis) in single $z$-vertex bin whose range is given at the top. Red lines and arrows indicate region accepted in analyses.}\label{fig:tpcEff_pion_sample}
		\end{minipage}
	}
\end{figure}
%---------------------------

\begin{figure}[ht]%[hb]
	\centering
	\parbox{0.329\textwidth}{
		\centering
		\includegraphics[width=\linewidth,page=1]{graphics/eff/tpcEffi.pdf}\\
		\includegraphics[width=\linewidth,page=4]{graphics/eff/tpcEffi.pdf}\\
	}~
	\parbox{0.329\textwidth}{
		\centering
		\includegraphics[width=\linewidth,page=2]{graphics/eff/tpcEffi.pdf}\\
		\includegraphics[width=\linewidth,page=5]{graphics/eff/tpcEffi.pdf}\\
	}%
	\parbox{0.329\textwidth}{
		\centering
		\includegraphics[width=\linewidth,page=3]{graphics/eff/tpcEffi.pdf}\\
		\includegraphics[width=\linewidth,page=6]{graphics/eff/tpcEffi.pdf}\\
	}%
	\caption[TPC acceptance and reconstruction efficiency as a function of $p_T$ $\left(|V_z|<80\textrm{ cm}, |\eta|<0.7\right)$ obtained from two methods.]{TPC acceptance and reconstruction efficiency as a function of $p_T$ $\left(|V_z|<80\textrm{ cm}, |\eta|<0.7\right)$ obtained from two methods.}\label{fig:trackTPCefficiencyComparisonEtaPhi}
\end{figure}




\section{TOF acceptance, hit reconstruction and track matching efficiency}\label{sec:tofMatchEff}

Combined TOF acceptance, hit reconstruction efficiency and matching efficiency with TPC tracks, $\epsilon_{\textrm{\tiny TOF}}$, was defined as the probability that the global TPC track that satisfy quality criteria (cuts~\ref{sec:TpcQualityCuts}) is matched with hit in TOF (\ref{sec:TpcTofMatchingRequirement}). This quantity is generally referred as ``TOF efficiency''.

It was calculated in the very similiar way to TPC efficiency - single particle STARsim MC embedded into zero-bias triggers was used. Tracks belonging to $set~B$ from Sec.~\ref{sec:tpcAccAndEff} were utilized. From these tracks a sub-sample of tracks with non-zero TOF matching flag was extracted ($set~C$). The TOF efficiency was calculated as
\begin{equation}\label{eq:tofAccAndEffDefinition}
		\epsilon_{\textrm{\tiny TOF}}\left(p_{T}, \eta, z_{vtx};~\textrm{sign},\textrm{PID}\right) = \frac{(p_{T},\eta, z_{vtx})~\textrm{histogram for particles of given sign and ID from}~set~C}{(p_{T},\eta, z_{vtx})~\textrm{histogram for particles of given sign and ID from}~set~B}.
	\end{equation}

An additional note has to be made here about the correction which is applied to TOF matching flag in MC analysis. It was found that in embedded simulation the dead TOF elements were not masked. To correct for this effect (hence obtain more reliable TOF efficiency) a data-based map of TOF modules was created, separately for each RHIC fill. Map was filled with modules which were matched with TPC tracks in the data. In all MC sample analyses (including efficiency determination) each TPC track with non-zero TOF match flag was additionally checked if TOF module that it was matched with had any entries in the data-based map. If not - the TOF match flag was considered 0.

\subsection{Sample of  efficiency plots}

The sample TOF efficiency plot is shown in Fig.~\ref{fig:tofEff_pion_sample}. All remaining TOF efficiency plots are contained in Appendix~\ref{appendix:tofEff}.

%---------------------------
\begin{figure}[h!]%\vspace{-10pt}
	\centering
	\parbox{0.485\textwidth}{
		\centering
		\begin{subfigure}[b]{\linewidth}
			\subcaptionbox{\label{fig:tofEff_pion_sample_a}}{\includegraphics[width=\linewidth,page=3]{graphics/eff/Eff2D_TOF_pion_Minus.pdf}\vspace*{-8pt}}
		\end{subfigure}\\[5pt]
		\begin{subfigure}[b]{\linewidth}\addtocounter{subfigure}{1}
			\subcaptionbox{\label{fig:tofEff_pion_sample_c}}{\includegraphics[width=\linewidth,page=18]{graphics/eff/Eff2D_TOF_pion_Minus.pdf}\vspace*{-8pt}}
		\end{subfigure}
	}%
	\quad%
	\parbox{0.485\textwidth}{
		\centering
		\begin{subfigure}[b]{\linewidth}\addtocounter{subfigure}{-2}
			\subcaptionbox{\label{fig:tofEff_pion_sample_b}}{\includegraphics[width=\linewidth,page=11]{graphics/eff/Eff2D_TOF_pion_Minus.pdf}\vspace*{-8pt}}
		\end{subfigure}\\[5pt]
		\begin{minipage}[t][0.78\linewidth][t]{\linewidth}\vspace{10pt}
			\caption[Sample plotz of TOF acceptance, reconstruction and matching efficiency of $\pi^{-}$.]{Sample TOF acceptance, reconstruction and matching efficiency of $\pi^{-}$ in 3 bins of true $z_{\text{vtx}}$. Plots represents the TOF efficiency $\epsilon_{\text{TOF}}$ ($z$-axis) as a function of true particle pseudorapidity $\eta$ ($x$-axis) and transverse momentum $p_{T}$ ($y$-axis) in single $z$-vertex bin whose range is given at the top. Red lines and arrows indicate region accepted in analyses.}\label{fig:tofEff_pion_sample}
		\end{minipage}
	}
\end{figure}
%---------------------------

%---------------------------
%\begin{figure}[hb]%
%\centering\includegraphics[width=0.7\linewidth,page=11]{graphics/eff/Eff2D_TOF_pion_Minus.pdf}%
%\caption[Sample plot of TOF acceptance, reconstruction and matching efficiency of $\pi^{-}$.]{Sample plot of TOF acceptance, reconstruction and matching efficiency of $\pi^{-}$. Plot represents the TOF efficiency $\epsilon_{\text{TOF}}$ ($z$-axis) as a function of true particle pseudorapidity $\eta$ ($x$-axis) and transverse momentum $p_{T}$ ($y$-axis) in single $z$-vertex bin whose range is given at the top. Red lines and arrows indicate region accepted in analyses.}\label{fig:tofEff_pion_sample}
%\end{figure}
%---------------------------

As shown in Sec.~\ref{subsec:tofAbsEffSystAndCorr}, it was found during estimation of the systematic uncertainty of the TOF efficiency that the data-driven efficiency and MC effficiency (obtained with a method described in this section) differ significantly. Therefore the final TOF efficiency which is used to correct the data is a modified MC efficiency. Form of MC efficiency modification is given in ... .

% 
% \section{TPC vertex reconstruction efficiency}\label{sec:tpcVxRecoEff}
% 
% The definition of vertex reconstruction efficiency established in this analysis is the probability that two global tracks, both associated with true level primary particles from the kinematic region of the measurement, both satisfying kinematic and quality criteria (cuts~\ref{sec:TpcKinematicCuts} and ~\ref{sec:TpcQualityCuts}) and both matched with hits in TOF, form a vertex listed in the collection of reconstructed primary vertices and DCA(R) and DCA(z) of both global tracks calculated w.r.t. this vertex is contained within the limits of cut~\ref{sec:TpcDcaCuts}.