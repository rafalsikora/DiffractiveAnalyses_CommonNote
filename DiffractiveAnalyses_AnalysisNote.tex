\documentclass[a4paper,10pt,notitlepage]{report}
\usepackage[utf8]{inputenc}
\usepackage{authblk}
\usepackage{geometry}
\usepackage{graphicx}
\usepackage{grffile}
\usepackage{tabulary}
\usepackage{amsmath}
\usepackage{amssymb}
\usepackage{mathtools}
\usepackage{float}
\usepackage{cite}
\usepackage{color}
\usepackage{caption}
\usepackage{subcaption}
% \usepackage[pdfborder={0 0 0.6}]{hyperref}
\usepackage[linktocpage=true]{hyperref}
\usepackage{listliketab}
\usepackage{enumitem}
\usepackage{multirow}
\usepackage{multicol}
\usepackage{makecell}
\usepackage{bm}
\usepackage{wrapfig}
\usepackage{empheq}
\usepackage[titletoc]{appendix}
\usepackage{cleveref}
\usepackage[bottom]{footmisc}
%\usepackage{lineno}
%\linenumbers
%%%%%%%%%%%%%%%%%%%%%%%%%%%%%%%%%%%%%%% making figures in the subsection where placed (according to https://tex.stackexchange.com/questions/279/how-do-i-ensure-that-figures-appear-in-the-section-theyre-associated-with/235312#235312 )
\usepackage{placeins}

\let\Oldsection\section
\renewcommand{\section}{\FloatBarrier\Oldsection}

\let\Oldsubsection\subsection
\renewcommand{\subsection}{\FloatBarrier\Oldsubsection}

\let\Oldsubsubsection\subsubsection
\renewcommand{\subsubsection}{\FloatBarrier\Oldsubsubsection}
%%%%%%%%%%%%%%%%%%%%%%%%%%%%%%%%%%%%%%%5

\pdfinfo{%
  /Title    ()
  /Author   (Leszek Adamczyk, Łukasz Fulek, Rafał Sikora)
}


%colors
\definecolor{gray}{rgb}{0.5, 0.5, 0.5}
\definecolor{dgreen}{rgb}{0.1,0.6,0.1}  %zielony

%additional symbols
\newcommand{\Pom}{\texorpdfstring{I$\!$P}{P}}             % gives pomeron symbol
\newcommand{\Reg}{\texorpdfstring{I$\!$R}{R}}               % gives pomeron symbol
\newcommand{\DPE}{D\Pom E}
\newcommand{\Pomeron}{\Pom omeron}
\newcommand{\Reggeon}{\Reg eggeon}
\newcommand{\bis}{\prime\prime}

% ---- ---- additional commands ---- ----
\newcommand{\specialcell}[2][c]{%
  \begin{tabular}[#1]{@{}c@{}}#2\end{tabular}}

\makeatletter
\def\tagform@#1{\maketag@@@{(\ignorespaces#1\unskip\@@italiccorr)}}
\renewcommand{\eqref}[1]{\textup{{\normalfont(\ref{#1}}\normalfont)}}
\makeatother
  
\makeatletter
\newcommand{\Spvek}[2][l]{%
  \gdef\@VORNE{1}
  \left[\hskip-\arraycolsep%
    \begin{array}{#1}\vekSp@lten{#2}\end{array}%
  \hskip-\arraycolsep\right]}
\def\vekSp@lten#1{\xvekSp@lten#1;vekL@stLine;}
\def\vekL@stLine{vekL@stLine}
\def\xvekSp@lten#1;{\def\temp{#1}%
  \ifx\temp\vekL@stLine
  \else
    \ifnum\@VORNE=1\gdef\@VORNE{0}
    \else\@arraycr\fi%
    #1%
    \expandafter\xvekSp@lten
  \fi}
\makeatother
% ---- ---- ---- ---- ---- ---- ---- ----

%chapter heading
\makeatletter
\renewcommand{\@makechapterhead}[1]{%
 \vspace*{-18\p@}%
  {\parindent \z@ \raggedright
%     \LARGE \bfseries \thechapter. #1\par\nobreak
%     \vskip 40\p@
      \Huge \bfseries \thechapter. #1\par\nobreak
      \vskip 20\p@
  }}
\makeatother

\Crefname{figure}{Fig.}{Figs.}

% % Title Page
% \title{\textbf{Supplementary note on diffractive analyses\\of 2015 proton-proton data:\\[20pt]\large{\cite{AnalysisNoteRafal}~Measurement of Central Exclusive Production of $\bm{h\bar{h}}$ pairs ($\bm{h=\pi,K,p}$) with Roman Pot detectors in diffractive\\proton-proton interactions at~$\bm{\sqrt{s}=}$~200~GeV\\[10pt]\cite{AnalysisNoteLukasz}~Measurement of particle production with Roman Pot detectors\\in diffractive proton-proton interactions at~$\bm{\sqrt{s}=}$~200~GeV\\}}\vspace*{10pt}}
% \author[1]{\href{mailto:leszek.adamczyk@agh.edu.pl}{Leszek Adamczyk}}
% \author[1]{\href{mailto:lukasz.fulek@fis.agh.edu.pl}{Łukasz Fulek}}
% \author[1]{\href{mailto:rafal.sikora@fis.agh.edu.pl}{Rafał Sikora}}
% \affil[1]{AGH University of Science and Technology, FPACS, Kraków, Poland}

% Title Page
\title{\textbf{Supplementary note on diffractive analyses\\of 2015 proton-proton data:\\[20pt]\large{\cite{AnalysisNoteRafal}~Measurement of Central Exclusive Production of $\bm{h\bar{h}}$ pairs ($\bm{h=\pi,K,p}$) with Roman Pot detectors in diffractive\\proton-proton interactions at~$\bm{\sqrt{s}=}$~200~GeV\\[10pt]\cite{AnalysisNoteLukasz}~Measurement of particle production with Roman Pot detectors\\in diffractive proton-proton interactions at~$\bm{\sqrt{s}=}$~200~GeV\\}}\vspace*{10pt}}
\author[ ]{\href{mailto:leszek.adamczyk@agh.edu.pl}{Leszek Adamczyk}}
\author[ ]{\href{mailto:lukasz.fulek@fis.agh.edu.pl}{Łukasz Fulek}}
\author[ ]{\href{mailto:rafal.sikora@fis.agh.edu.pl}{Rafał Sikora}}
\affil[ ]{AGH University of Science and Technology, FPACS, Kraków, Poland}

\setcounter{Maxaffil}{0}
\renewcommand\Affilfont{\itshape\small}
\renewcommand{\bibname}{References}

\begin{document}

\begin{center}
\begin{minipage}[c]{0.12\linewidth}%
\vspace{5.5pt}\textbf{\LARGE{of the}}
\end{minipage}
\begin{minipage}[c]{0.15\linewidth}%
\hspace*{-8pt}\includegraphics[width=\linewidth]{graphics/STAR_logo.pdf}
\end{minipage}~
\begin{minipage}[c]{0.24\linewidth}%
\vspace{9pt}\hspace*{-8pt}\textbf{\LARGE{Experiment}}
\end{minipage}\\[-50pt]
\textbf{\LARGE{Analysis Note}}

\vspace*{125pt}
\begin{minipage}{\linewidth}
\maketitle
\begin{abstract}
In this note we present supplementary material related to analyses of diffractive processes based on 2015 data from proton-proton collisions at $\sqrt{s}=200$~GeV. This dataset was collected with newly installed Roman Pot detectors in Phase II* configuration which ensured efficient triggering and measuring diffractively scattered protons.\\\indent We focus on issues which are common in aforementioned analyses: calculation of efficiencies, corrections to Monte Carlo simulations and derivation of systematic uncertainties of our measurements.
\end{abstract}
\thispagestyle{empty}
\end{minipage}

% \vspace{50pt}

%  \Huge{\textbf{\textit{DRAFT}}}
\end{center}


\clearpage
\thispagestyle{empty}
\newgeometry{hmargin={2cm, 2cm}, vmargin={-0.68cm, 2.6cm} }

\setcounter{secnumdepth}{3}
\setcounter{tocdepth}{4}
\addtocontents{toc}{\protect\enlargethispage{\baselineskip}}
\tableofcontents
\addtocontents{toc}{\vspace{-20pt}}

\clearpage
\thispagestyle{empty}
\newgeometry{hmargin={2cm, 2cm}, height=10.0in}

%% =====  BAD RUNS ====
%%===========================================================%%
%%                                                                                                                      %%
%%                                                    BAD RUNS                                                 %%
%%                                                                                                                      %%
%%===========================================================%%


\chapter{Bad run list}\label{chap:badRunList}

%\subsection{RP distance from the beamline}

16065025
16065026
16065027
16065028
16072057
16072058
16077055
16083006
16083007
16106031

%---------------------------
\begin{figure}[hb]
\centering
\parbox{0.4\textwidth}{
  \centering
  \begin{subfigure}[b]{\linewidth}{
                \subcaptionbox{\label{fig:positionHistograms}}{\includegraphics[width=\linewidth]{graphics/badRuns/positionHistograms.pdf}}}
  \end{subfigure}
}
\quad
\parbox{0.545\textwidth}{
  \centering
  \begin{subfigure}[b]{\linewidth}{
                \subcaptionbox{\label{fig:positionVsRunGraph}}{\includegraphics[width=\linewidth]{graphics/badRuns/positionVsRunGraph.pdf}}}
  \end{subfigure}
}%
\caption[Beam-detector distance of the Roman Pots in run 15.]{Histogram of beam-detector distance $|y|$ (\ref{fig:positionHistograms}) and graph showing run-dependence of $|y|$ (\ref{fig:positionVsRunGraph}) for all Roman Pots.}%\label{fig:xy_recoEff}
\end{figure}
%---------------------------
 
%% =====  TPC TRACK QUALITY CUTS ====
%%===========================================================%%
%%                                                           %%
%%                   TPC TRACK QUALITY CUTS                  %%
%%                                                           %%
%%===========================================================%%


\chapter{TPC track quality cuts}\label{chap:TpcTrackQualityCuts}

Charged particle tracks reconstructed in the TPC are selected in our analyses with the set of cuts listed below. The goal of these criteria is to 
\begin{enumerate}
\item Both tracks are contained within the kinematic range:\label{enum:TpcKinematicCuts}\\[2pt]
$|\eta|<1$,~~~~$p_{T}>0.2~\textrm{GeV}/c$,
\item Both tracks satisfy quality criteria:\label{enum:TpcQualityCuts}\\[2pt]
$N_{\textrm{hits}}^{\textrm{fit}}\geq25$,~~~~$N_{\textrm{hits}}^{\textrm{dE/dx}}\geq15$,~~~~$N_{\textrm{hits}}^{\textrm{fit}}/N_{\textrm{hits}}^{\textrm{poss}}\geq0.52$,~~~~$|d_{0}|<2$~cm
\item Both tracks match well to the primary vertex:\label{enum:TpcDcaCuts}\\[2pt]
$\textrm{DCA}(R)<1.5$~cm,~~~~$\textrm{DCA}(z)<1$~cm.
\end{enumerate}
%---------------------------

%% =====  EFFICIENCIES ====
%%===========================================================%%
%%                                                           %%
%%                       EFFICIENCIES                        %%
%%                                                           %%
%%===========================================================%%


\chapter{Efficiencies}\label{chap:efficiencies}
\section{General overview}\label{sec:effGeneralOverview}
It is the best to apply the corrections to the measured distributions  that are based on real data analysis. However, due to the complexity of the processes we are studying, it is necessary to obtain corrections from the MC simulation. Whenever possible, we compare efficiencies obtained from MC with these from the data and introduce appropriate corrections to the MC if necessary.

It is prefered to get the detector to true-level corrections from the MC, which is dedicated to the studied physics process. However, for this purpose, the~statistics in the MC should be several times greater (preferably an order of magnitude) than we have in the data for analysis. This is not possible with generally low total efficiency $\left(\textrm{TPC \& TOF \& RP}\right)$. On the other hand, such efficiencies depend on the modeling of the studied process and require a good description of all generally multidimensional distributions  by MC. Since that, the basic method of corrections that we use in the analyses is the method of factorization of global efficiency into the product of single-particle efficiencies (separate for each central particle and forward protons). Thereby,  statistically precise multidimensional corrections on TPC, TOF and RP are obtained. Nevertheless, this method does not correct on the migration effects caused by the finite detector resolution and is sensitive to the binning. Whenever possible, we use both methods of correction (at least in the consistency tests for MC samples). In this note, only the single-particle corrections are described, which are common for both analyses.

\section{TPC track acceptance and reconstruction efficiency}\label{sec:tpcAccAndEff}
We defined joint acceptance and efficiency of reconstruction of a track in the TPC, $\epsilon_{\textrm{\tiny TPC}}$, as the probability that particle from the primary interaction generates signal in the detector which is reconstructed as a global track that satisfies all quality criteria (cuts~\ref{sec:TpcQualityCuts}).
To derive this efficiency the single particle STARsim MC embedded into zero-bias trigger data taken simultaneously with physics triggers 
 was used.
\subsection{STAR nominal method}
Technically, the common method used by the STAR to obtain  $\epsilon_{\textrm{\tiny TPC}}$ is the following procedure:
\begin{enumerate}
	\item True-level primary particles of given ID and charge were selected ($set~A$).
	\item Each particle from $set~A$ was checked if any global TPC track with more than half of hit points associated to it was generated by true-level particle  (definition of true level particle-track matching, idTruth from StMuTrack collection). All particles from $set~A$ which have associated global track satisfying quality criteria (cut~\ref{sec:TpcQualityCuts}) formed $set~B$.
	\item The joint TPC acceptance and efficiency was calculated as the ratio of the histograms of true level quantities (such as $p_{T}$, $\eta$, $z_{\textrm{vtx}}$) for particles from $set~B$ and particles from $set~A$:
	\begin{equation}\label{eq:tpcAccAndEffDefinition}
	\epsilon_{\textrm{\tiny TPC}}\left(p_{T}, \eta, z_{vtx};~\textrm{sign},\textrm{PID}\right) = \frac{(p_{T},\eta, z_{vtx})~\textrm{histogram for particles of given sign and ID from}~set~B}{(p_{T},\eta, z_{vtx})~\textrm{histogram for particles of given sign and ID from}~set~A}.
	\end{equation}
	
\end{enumerate}
\subsubsection{Global track to true level particle matching}
It was found during the analysis that in about $1\%$ events there are more than one reconstructed global track matched with the same true level particle, which is shown in Fig.~\ref{fig:trackSplittingNominal}, separatelly for true level pions, kaons and protons. This generally leads to few percent overestimation of corrected number of tracks since each reconstructed tracks is weigthed by TPC efficiency. Therefore only one reconstructed tracks should be considered as good track and the rest should be treated as fake tracks. 
\begin{figure}[ht]
	\centering
	\parbox{0.329\textwidth}{
		\centering
		\includegraphics[width=\linewidth,page=1]{graphics/eff/trackSplitting_CD.pdf}\\
		\includegraphics[width=\linewidth,page=4]{graphics/eff/trackSplitting_CD.pdf}\\
	}~
	\parbox{0.329\textwidth}{
		\centering
		\includegraphics[width=\linewidth,page=2]{graphics/eff/trackSplitting_CD.pdf}\\
		\includegraphics[width=\linewidth,page=5]{graphics/eff/trackSplitting_CD.pdf}\\
	}%
	\parbox{0.329\textwidth}{
		\centering
		\includegraphics[width=\linewidth,page=3]{graphics/eff/trackSplitting_CD.pdf}\\
		\includegraphics[width=\linewidth,page=6]{graphics/eff/trackSplitting_CD.pdf}\\
	}%
	\caption[Number of reconstructed global tracks, satisfying all quality criteria, matched with the same true level primary particle.]{Number of reconstructed global tracks, satisfying all quality criteria (cuts~\ref{sec:TpcQualityCuts}), matched with the~same true level primary particle. The type of true level particle is indicated in the figure.}\label{fig:trackSplittingNominal}
\end{figure}

The true level particle end vertex $V_r^{end}$ is not specified if the particle neither interacted with the dead material nor decayed. The analysis showed that $1\%$ of the reconstructed tracks are matched to the true particle which lost identitiy ($V_r^{end}<48$~cm) before entering TPC. We interprate such situation as matching of reconstructed daugther particle to true level parent particle, which is clear bug in the STAR software rather than intentional feature. It is potentially dangerous since  momentum and type of daugther particle might be significantly different from parent particle.  Problem with wrong true level matching is also present for tracks with only one track matched to true level particle which do not decay or interact with material (no end vertex assosiated to true  level particle). 
It is visible on Fig.~\ref{fig:trackSplittingNominaldEdx} where $dE/dx$ is shown that some reconstracted tracks 
have different PID than true level particle matched to it.
 Also, there are problems in the closure tests, where  the  reconstructed-level distributions of rapidity and transverse momenta weighted by the nominal efficiency corrections do not describe the true level distributions. Main reason for failing of closure tests is non-negligible (and significant   for anti-protons) amount of global tracks matched with primary particles but no or little correlation in $\eta-\phi$ space beetwen
matched pair. 
This correlation can be represented by the the~distance 
\begin{equation}\label{eq:tpcMatchingDeltaSquare}
\delta^{2}\left(\eta,\phi\right)=\left(\eta^{true}-\eta^{reco}\right)^2+\left(\phi^{true}-\phi^{reco}\right)^2
\end{equation}
between the true level particle and global track assigned to it, shown in Fig.~\ref{fig:trackSplittingNominalDelta_1} for particles with only one  matched global track and Fig.~\ref{fig:trackSplittingNominalDelta_2} for particles with at least two  matched global tracks, indicates that some part of the tracks taken for the efficiency calculation are measured very badly ($\delta^{2}\left(\eta,\phi\right)$ is very large), even if there is only one global track matched to the true-particle.
\begin{figure}[!h]
	\vspace{-0.1cm}
	\centering
	\parbox{0.329\textwidth}{
		\centering
		\includegraphics[width=\linewidth,page=31]{graphics/eff/trackSplitting_CD.pdf}\\
		\includegraphics[width=\linewidth,page=34]{graphics/eff/trackSplitting_CD.pdf}\\
	}~
	\parbox{0.329\textwidth}{
		\centering
		\includegraphics[width=\linewidth,page=32]{graphics/eff/trackSplitting_CD.pdf}\\
		\includegraphics[width=\linewidth,page=35]{graphics/eff/trackSplitting_CD.pdf}\\
	}%
	\parbox{0.329\textwidth}{
		\centering
		\includegraphics[width=\linewidth,page=33]{graphics/eff/trackSplitting_CD.pdf}\\
		\includegraphics[width=\linewidth,page=36]{graphics/eff/trackSplitting_CD.pdf}\\
	}%
	\caption[$dE/dx$ of the track matched to true level particle.]{$dE/dx$ of the track matched to true level particle. Lines indicate Bichsel function prediction for each particle species. Only tracks matched to non-interacting true level particles without end vertex  are shown. Black line with arrow indicates region accepted in the analysis.}\label{fig:trackSplittingNominaldEdx}
\end{figure}

Because of several above mentioned  problems with  nominal STAR definition of matching between reconstructed tracks and true  level particles we decided to use in the analysis modified matching            
definition by taking into the account the difference    between reconstructed tracks and true particles in $\eta-\phi$ space.




\begin{figure}[hb]
	\centering
	\parbox{0.329\textwidth}{
		\centering
		\includegraphics[width=\linewidth,page=25]{graphics/eff/trackSplitting_CD.pdf}\\
		\includegraphics[width=\linewidth,page=28]{graphics/eff/trackSplitting_CD.pdf}\\
	}~
	\parbox{0.329\textwidth}{
		\centering
		\includegraphics[width=\linewidth,page=26]{graphics/eff/trackSplitting_CD.pdf}\\
		\includegraphics[width=\linewidth,page=29]{graphics/eff/trackSplitting_CD.pdf}\\
	}%
	\parbox{0.329\textwidth}{
		\centering
		\includegraphics[width=\linewidth,page=27]{graphics/eff/trackSplitting_CD.pdf}\\
		\includegraphics[width=\linewidth,page=30]{graphics/eff/trackSplitting_CD.pdf}\\
	}%
	\caption[$\delta^{2}\left(\eta,\phi\right)$ distributions between true level particles and tracks assigned to them.]{$\delta^{2}\left(\eta,\phi\right)$ distributions  between true level particles and tracks assigned to them. Only true level particles with only one reconstructed track matched to them were selected. Red lines and arrows indicate  the~cut value of $0.15^2$, which is used in the modified true level particle-track matching definition.}\label{fig:trackSplittingNominalDelta_1}
\end{figure}

\begin{figure}[h!]\vspace{-10pt}
	\centering
	\parbox{0.329\textwidth}{
		\centering
		\includegraphics[width=\linewidth,page=19]{graphics/eff/trackSplitting_CD.pdf}\\
		\includegraphics[width=\linewidth,page=22]{graphics/eff/trackSplitting_CD.pdf}\\
	}~
	\parbox{0.329\textwidth}{
		\centering
		\includegraphics[width=\linewidth,page=20]{graphics/eff/trackSplitting_CD.pdf}\\
		\includegraphics[width=\linewidth,page=23]{graphics/eff/trackSplitting_CD.pdf}\\
	}%
	\parbox{0.329\textwidth}{
		\centering
		\includegraphics[width=\linewidth,page=21]{graphics/eff/trackSplitting_CD.pdf}\\
		\includegraphics[width=\linewidth,page=24]{graphics/eff/trackSplitting_CD.pdf}\\
	}%
	\caption[$\delta^{2}\left(\eta,\phi\right)$ distributions between true level particles and tracks assigned to them.]{$\delta^{2}\left(\eta,\phi\right)$ distributions between true level particles and tracks assigned to them. Only true level particles with at least two reconstructed tracks matched to them were selected. Red lines and arrows indicate  the~cut value of $0.15^2$, which is used in the modified true level particle-track matching definition.}\label{fig:trackSplittingNominalDelta_2}
\end{figure}




\subsection{Method used in this analysis}\label{subsec:definitionTrueLevelMatching}
In this method, the definition of true level particle-track matching is modified. In addition to the requirement of the appropriate number of common hit points, the distance between true level particle and track is required to be smaller than $0.15$, $\delta^{2}\left(\eta,\phi\right)<\left(0.15\right)^2$. It is quite an arbitrary value which should be small but not too small  to loose good events. The value of $\delta^2$ cut was chosen by the requirement that only acceptable small amount of CEP events which passed all selection criteria will not satisfy matching criteria. It was verified with the CEP MC embedded into zero-bias triggers that with quoted value of cut on $\delta^{2}\left(\eta,\phi\right)$ less than $0.3\%$ of CEP events have at least one track which is not considered to be matched with true-level pion despite the standard matching (Fig.~\ref{fig:deltaSqCEP}). We consider this an acceptably low effect.

Tracks, which do not satisfy the above criterion, are treated as fake tracks (even if they are matched to the true level particle in the standard way). In $99.99\%$ cases, where the $\delta^{2}\left(\eta,\phi\right)<\left(0.15\right)^2$, there is only one track matched to true level particle (Fig.~\ref{fig:trackSplittingetaPhi}). Additionally, the $dE/dx$ of the track is mostly consistent with the true level PID (Fig.~\ref{fig:trackSplittingEtaPhidEdx}). Figure~\ref{fig:trackTPCefficiencyComparisonEtaPhi} shows the difference between TPC efficiencies obtained with the STAR standard and the modified definition of true particle-track matching. The maximum differences between TPC efficiencies in the analyzed $p_T$ range are about $2\%$ for pions, $3\%$ for kaons, $2\%$ and $4\%$ for protons and antiprotons, respectively.


%---------------------------
\begin{figure}[h!]%\vspace{-10pt}
	\centering
	\parbox{0.685\textwidth}{
		\centering
		\begin{subfigure}[b]{\linewidth}
			\includegraphics[width=\linewidth]{graphics/eff/deltaEtaSqDeltaPhiSqMatchedExclusive.pdf}
		\end{subfigure}
	}%
	\quad%
	\parbox{0.285\textwidth}{
		\centering
		\begin{minipage}[t][0.78\linewidth][t]{\linewidth}\vspace{-60pt}
			\caption[Distribution of $\delta^{2}\left(\eta,\phi\right)$ in CEP MC.]%
			{Distribution of $\delta^{2}\left(\eta,\phi\right)$ for tracks matched with true-level pions (using standard matching) in CEP MC embedded into zero-bias triggers. Tracks were taken from events passing full CEP event selection, recognized as exclusive $\pi^{+}\pi^{-}$. The vertical red dashed line indicates the cut value of $0.15^{2} \approx 0.023$, above which less than 0.14\% of tracks is contained.}%
			\label{fig:deltaSqCEP}
		\end{minipage}
	}
\end{figure}
%---------------------------


\begin{figure}[ht]
	\centering
	\parbox{0.329\textwidth}{
		\centering
		\includegraphics[width=\linewidth,page=1]{graphics/eff/trackSplitting_QualityEtaPhiCD.pdf}\\
		\includegraphics[width=\linewidth,page=4]{graphics/eff/trackSplitting_QualityEtaPhiCD.pdf}\\
	}~
	\parbox{0.329\textwidth}{
		\centering
		\includegraphics[width=\linewidth,page=2]{graphics/eff/trackSplitting_QualityEtaPhiCD.pdf}\\
		\includegraphics[width=\linewidth,page=5]{graphics/eff/trackSplitting_QualityEtaPhiCD.pdf}\\
	}%
	\parbox{0.329\textwidth}{
		\centering
		\includegraphics[width=\linewidth,page=3]{graphics/eff/trackSplitting_QualityEtaPhiCD.pdf}\\
		\includegraphics[width=\linewidth,page=6]{graphics/eff/trackSplitting_QualityEtaPhiCD.pdf}\\
	}%
	\caption[Number of reconstructed global tracks, satisfying all quality criteria and $\delta^{2}\left(\eta,\phi\right)$ cut, matched with the same true level primary particle.]{Number of reconstructed global tracks, satisfying all quality criteria (cuts~\ref{sec:TpcQualityCuts}) and $\delta^{2}\left(\eta,\phi\right)$ cut, matched with the same true level primary particle.}\label{fig:trackSplittingetaPhi}
\end{figure}


\begin{figure}[h!]%\vspace{-5pt}
	\centering
	\parbox{0.48\textwidth}{
		\centering
		\includegraphics[width=\linewidth,page=21]{graphics/eff/trackSplitting_QualityEtaPhiCD.pdf}\\
		\includegraphics[width=\linewidth,page=22]{graphics/eff/trackSplitting_QualityEtaPhiCD.pdf}\\
		\includegraphics[width=\linewidth,page=23]{graphics/eff/trackSplitting_QualityEtaPhiCD.pdf}\\
	}~
	\parbox{0.48\textwidth}{
		\centering
		\includegraphics[width=\linewidth,page=24]{graphics/eff/trackSplitting_QualityEtaPhiCD.pdf}\\
		\includegraphics[width=\linewidth,page=25]{graphics/eff/trackSplitting_QualityEtaPhiCD.pdf}\\
		\includegraphics[width=\linewidth,page=26]{graphics/eff/trackSplitting_QualityEtaPhiCD.pdf}\\
	}%
	\caption[$dE/dx$ of the closest track matched to true level particle passing the $\delta^{2}\left(\eta,\phi\right)$ cut.]{$dE/dx$ of the closest track matched to true level particle passing the $\delta^{2}\left(\eta,\phi\right)$ cut. Lines indicate Bichsel function prediction for each particle species. Black line with arrow indicates region accepted in the analysis.}\label{fig:trackSplittingEtaPhidEdx}
\end{figure}

\begin{figure}[ht]%[hb]
	\centering
	\parbox{0.48\textwidth}{
		\centering
		\includegraphics[width=\linewidth,page=1]{graphics/eff/tpcEffi.pdf}\\
		\includegraphics[width=\linewidth,page=2]{graphics/eff/tpcEffi.pdf}\\
		\includegraphics[width=\linewidth,page=3]{graphics/eff/tpcEffi.pdf}\\
	}~
	\parbox{0.48\textwidth}{
		\centering
		\includegraphics[width=\linewidth,page=4]{graphics/eff/tpcEffi.pdf}\\
		\includegraphics[width=\linewidth,page=5]{graphics/eff/tpcEffi.pdf}\\
		\includegraphics[width=\linewidth,page=6]{graphics/eff/tpcEffi.pdf}\\
	}%
	\caption[TPC acceptance and reconstruction efficiency as a function of $p_T$ $\left(|V_z|<80\textrm{ cm}, |\eta|<0.7\right)$ obtained from two methods.]{TPC acceptance and reconstruction efficiency as a function of $p_T$ $\left(|V_z|<80\textrm{ cm}, |\eta|<0.7\right)$ obtained from two methods. Black line with arrow indicates region accepted in the analysis.}\label{fig:trackTPCefficiencyComparisonEtaPhi}
\end{figure}


\subsection{Sample of  efficiency plots}\label{subsec:sampleTpcEffPlots}

In Figure~\ref{fig:tpcEff_pion_sample} we present sample plots of the TPC track acceptance and reconstruction efficiency calculated with modified definition of reconstructed track and true-level particle matching (according to description in Sec.~\ref{subsec:definitionTrueLevelMatching}), used in our analyses. Plots for all analyzed particle types and all bins of true $z_{\text{vtx}}$ are contained in Appendix~\ref{appendix:tpcEff}.

In order to maximize the statistics available for the measurement (possibly wide range of accepted longitudinal vertex position $z_{\text{vtx}}$) with maximized probed phase-space in analyzed physics processes (wide range of track $p_{T}$ and $\eta$) and minimized systematic uncertainties related to the central detector (TPC and TOF), we have studied the efficiency plots like ones shown in Fig.~\ref{fig:tpcEff_pion_sample} and Fig.~\ref{fig:tofEff_pion_sample}. We thus decided to set the cut on $z_{\text{vtx}}$ at $\pm80~\text{cm}$, which corresponds to 89\% of the full integral of normal distribution with mean at 0 and standard deviation of 50~cm. At the same time we set the cuts on track $p_{T}$ and $\eta$ as listed in Sec.~\ref{sec:TpcKinematicCuts}. These cuts are represented with red dashed lines in Fig.~\ref{fig:tpcEff_pion_sample} and Fig.~\ref{fig:tofEff_pion_sample}. Our goal was to operate within cuboid ($z_{\text{vtx}}$, $p_{T}$, $\eta$) region of relatively high TPC and TOF efficiency ($\geq50\%$ of the maximum value). In other words, we required high acceptance and efficiency for a rectangular ($p_{T}$, $\eta$) space with limits independent from $z_{\text{vtx}}$. One can see that the red lines in Fig.~\ref{fig:tpcEff_pion_sample} and Fig.~\ref{fig:tofEff_pion_sample} always contain in their interior the region of relatively high acceptance.

%---------------------------
\begin{figure}[h!]%\vspace{-10pt}
	\centering
	\parbox{0.485\textwidth}{
		\centering
		\begin{subfigure}[b]{\linewidth}
			\subcaptionbox{\label{fig:tpcEff_pion_sample_a}}{\includegraphics[width=\linewidth,page=3]{graphics/eff/Eff2D_TPC_pion_Minus.pdf}\vspace*{-8pt}}
		\end{subfigure}\\[5pt]
		\begin{subfigure}[b]{\linewidth}\addtocounter{subfigure}{1}
			\subcaptionbox{\label{fig:tpcEff_pion_sample_c}}{\includegraphics[width=\linewidth,page=18]{graphics/eff/Eff2D_TPC_pion_Minus.pdf}\vspace*{-8pt}}
		\end{subfigure}
	}%
	\quad%
	\parbox{0.485\textwidth}{
		\centering
		\begin{subfigure}[b]{\linewidth}\addtocounter{subfigure}{-2}
			\subcaptionbox{\label{fig:tpcEff_pion_sample_b}}{\includegraphics[width=\linewidth,page=11]{graphics/eff/Eff2D_TPC_pion_Minus.pdf}\vspace*{-8pt}}
		\end{subfigure}\\[5pt]
		\begin{minipage}[t][0.78\linewidth][t]{\linewidth}\vspace{10pt}
		\caption[Sample TPC acceptance and reconstruction efficiency of $\pi^{-}$.]{Sample TPC acceptance and reconstruction efficiency of $\pi^{-}$ in 3 bins of true $z_{\text{vtx}}$. Plots represent the TPC efficiency $\epsilon_{\text{TPC}}$ ($z$-axis) as a function of true particle pseudorapidity $\eta$ ($x$-axis) and transverse momentum $p_{T}$ ($y$-axis) in single $z$-vertex bin whose range is given at the top. Red lines and arrows indicate region accepted in analyses.}\label{fig:tpcEff_pion_sample}
		\end{minipage}
	}
\end{figure}
%---------------------------





\section{TOF acceptance, hit reconstruction and track matching efficiency}\label{sec:tofMatchEff}

Combined TOF acceptance, hit reconstruction efficiency and matching efficiency with TPC tracks, $\epsilon_{\textrm{\tiny TOF}}$, was defined as the probability that the global TPC track that satisfy quality criteria (cuts~\ref{sec:TpcQualityCuts}) is matched with hit in TOF (\ref{sec:TpcTofMatchingRequirement}). This quantity is generally referred as ``TOF efficiency''.

It was calculated in the very similiar way to TPC efficiency - single particle STARsim MC embedded into zero-bias triggers was used. Tracks belonging to $set~B$ from Sec.~\ref{sec:tpcAccAndEff} were utilized. From these tracks a sub-sample of tracks with non-zero TOF matching flag (StMuBTofPidTraits.mMatchFlag $>0$) was extracted ($set~C$). The TOF efficiency was calculated as
\begin{equation}\label{eq:tofAccAndEffDefinition}
		\epsilon_{\textrm{\tiny TOF}}\left(p_{T}, \eta, z_{vtx};~\textrm{sign},\textrm{PID}\right) = \frac{(p_{T},\eta, z_{vtx})~\textrm{histogram for particles of given sign and ID from}~set~C}{(p_{T},\eta, z_{vtx})~\textrm{histogram for particles of given sign and ID from}~set~B}.
	\end{equation}

An additional note has to be made here about the correction which is applied to TOF matching flag in MC analysis. It was found that in embedded simulation the dead TOF elements were not masked. To correct for this effect (hence obtain more reliable TOF efficiency) a data-based map of modules was created, separately for each RHIC fill. Map was filled with modules which were matched with TPC tracks in the data. In all MC sample analyses (including efficiency determination) each TPC track with non-zero TOF match flag was additionally checked if TOF module that track was matched with had any entries in the data-based map. If not - the TOF match flag was considered 0. All dead TOF modules masked in the MC are listed in Appendix~\ref{appendix:tofDeadTraysModules}.

\subsection{Sample of  efficiency plots}

The sample TOF efficiency plot is shown in Fig.~\ref{fig:tofEff_pion_sample}. All remaining TOF efficiency plots are contained in Appendix~\ref{appendix:tofEff}.

As shown in Sec.~\ref{sec:tofAbsEffCorr} the data-driven efficiency and MC effficiency differ significantly, therefore the final TOF efficiency which is used to correct the data is the one presented here (taken from single particle embedded MC) but additionally modified according to correction derived in the reffered section.

%---------------------------
\begin{figure}[H]%\vspace{-10pt}
	\centering
	\parbox{0.485\textwidth}{
		\centering
		\begin{subfigure}[b]{\linewidth}
			\subcaptionbox{\label{fig:tofEff_pion_sample_a}}{\includegraphics[width=\linewidth,page=3]{graphics/eff/Eff2D_TOF_pion_Minus.pdf}\vspace*{-8pt}}
		\end{subfigure}\\[5pt]
		\begin{subfigure}[b]{\linewidth}\addtocounter{subfigure}{1}
			\subcaptionbox{\label{fig:tofEff_pion_sample_c}}{\includegraphics[width=\linewidth,page=18]{graphics/eff/Eff2D_TOF_pion_Minus.pdf}\vspace*{-8pt}}
		\end{subfigure}
	}%
	\quad%
	\parbox{0.485\textwidth}{
		\centering
		\begin{subfigure}[b]{\linewidth}\addtocounter{subfigure}{-2}
			\subcaptionbox{\label{fig:tofEff_pion_sample_b}}{\includegraphics[width=\linewidth,page=11]{graphics/eff/Eff2D_TOF_pion_Minus.pdf}\vspace*{-8pt}}
		\end{subfigure}\\[5pt]
		\begin{minipage}[t][0.78\linewidth][t]{\linewidth}\vspace{10pt}
			\caption[Sample plotz of TOF acceptance, reconstruction and matching efficiency of $\pi^{-}$.]{Sample TOF acceptance, reconstruction and matching efficiency of $\pi^{-}$ in 3 bins of true $z_{\text{vtx}}$. Plots represent the TOF efficiency $\epsilon_{\text{TOF}}$ ($z$-axis) as a function of true particle pseudorapidity $\eta$ ($x$-axis) and transverse momentum $p_{T}$ ($y$-axis) in single $z$-vertex bin whose range is given at the top. Red lines and arrows indicate region accepted in analyses.}\label{fig:tofEff_pion_sample}
		\end{minipage}
	}
\end{figure}
%---------------------------

%---------------------------
%\begin{figure}[hb]%
%\centering\includegraphics[width=0.7\linewidth,page=11]{graphics/eff/Eff2D_TOF_pion_Minus.pdf}%
%\caption[Sample plot of TOF acceptance, reconstruction and matching efficiency of $\pi^{-}$.]{Sample plot of TOF acceptance, reconstruction and matching efficiency of $\pi^{-}$. Plot represents the TOF efficiency $\epsilon_{\text{TOF}}$ ($z$-axis) as a function of true particle pseudorapidity $\eta$ ($x$-axis) and transverse momentum $p_{T}$ ($y$-axis) in single $z$-vertex bin whose range is given at the top. Red lines and arrows indicate region accepted in analyses.}\label{fig:tofEff_pion_sample}
%\end{figure}
%---------------------------


% 
% \section{TPC vertex reconstruction efficiency}\label{sec:tpcVxRecoEff}
% 
% The definition of vertex reconstruction efficiency established in this analysis is the probability that two global tracks, both associated with true level primary particles from the kinematic region of the measurement, both satisfying kinematic and quality criteria (cuts~\ref{sec:TpcKinematicCuts} and ~\ref{sec:TpcQualityCuts}) and both matched with hits in TOF, form a vertex listed in the collection of reconstructed primary vertices and DCA(R) and DCA(z) of both global tracks calculated w.r.t. this vertex is contained within the limits of cut~\ref{sec:TpcDcaCuts}.


%% =====  ENERGY LOSS CORRECTION ====
%%===========================================================%%
%%                                                           %%
%%                  ENERGY LOSS  CORRECTION                   %%
%%                                                           %%
%%===========================================================%%


\chapter{Energy Loss Correction}\label{chap:energyLossCorrection}
Particles passing through the detector material loose energy as they travel. The track momentum $p_T^{true}$ is reconstructed by fitting a helical path to the track points left in the detector. Fitting the track points to an ideal
helical track tends to underestimate the momentum due to these energy loss effects. To minimize biases due to this effect, correction procedure is applied 
during standard track momentum reconstruction procedure for both data and MC simulation. For this procedure all particles are assumed to be pions and the reconstructed momentum $p_T^{meas}$ is corrected by the amount of energy loss for a pion.  For anything that is not a pion some rest bias  is still present since on average energy loss is specific for each particle type. These biases  can be determined from simulated tracks run through GEANT.
The correction $p_T^{meas}-p_T^{true}$ was calculated for each particle species as a function of $p_T^{meas}$, $\eta$ and $z$-vertex. The sample energy loss correction averaged over $|\eta|<0.7$ for $K^-$ is shown in Fig. \ref{fig:energyLossPrimaryK_minus_sample}. 
\noindent The energy loss corrections for other particle species are shown in \Cref{fig:energyLossPrimaryPi_minus,fig:energyLossPrimaryPi_plus,fig:energyLossPrimaryK_minus,fig:energyLossPrimaryK_plus,fig:energyLossPrimaryP_bar,fig:energyLossPrimaryP,fig:energyLossPrimaryNegative,fig:energyLossPrimaryPositive,fig:energyLossPrimaryP_barGlobal,fig:energyLossPrimaryPGlobal} in Appendix \ref{appendix:energyLoss}.\newline



\begin{figure}[hb]

\centering
 \includegraphics[width=0.9\linewidth,page=30]{graphics/energyLoss/energyLoss3D_OnePrtAlso.pdf}
 \caption[Sample energy loss correction for $K^-$ as a function of reconstructed transverse momentum $p_T^{meas}$.]{Sample energy loss correction $p_T^{meas}-p_T^{true}$ for $K^-$ as a function of reconstructed transverse momentum $p_T^{meas}$ $\left(|\eta|<0.7\right)$ in single $z$-vertex bin, $-10<V_z <0$~cm. Red lines and arrows indicate region accepted in analyses.}\label{fig:energyLossPrimaryK_minus_sample}

\end{figure}

%% =====  ROMAN POT SIMULATION ====
%%===========================================================%%
%%                                                           %%
%%                       FORWARD PROTONS                     %%
%%                                                           %%
%%===========================================================%%


\chapter{Forward protons}\label{chap:forwardProtons}

\section{Roman Pot track reconstruction}

\subsection{Alignment}

\section{Roman Pot simulation}

\begin{figure}[hb]%
\caption[Apertures.]{Apertures.}\label{fig:aperturesWithFit}%
\centering
\parbox{0.495\textwidth}{
  \centering
  \includegraphics[width=\linewidth,page=1]{graphics/rpSim/Apertures_swapedAxes_withFit_beforeDxShift.pdf}\\
  \includegraphics[width=\linewidth,page=2]{graphics/rpSim/Apertures_swapedAxes_withFit_beforeDxShift.pdf}\\
  \includegraphics[width=\linewidth,page=3]{graphics/rpSim/Apertures_swapedAxes_withFit_beforeDxShift.pdf}
}~
\parbox{0.495\textwidth}{
  \centering
  \includegraphics[width=\linewidth,page=1]{graphics/rpSim/Apertures_swapedAxes_withFit.pdf}\\
  \includegraphics[width=\linewidth,page=2]{graphics/rpSim/Apertures_swapedAxes_withFit.pdf}\\
  \includegraphics[width=\linewidth,page=3]{graphics/rpSim/Apertures_swapedAxes_withFit.pdf}
}%
\end{figure}
\begin{figure}[ht!]\ContinuedFloat
% ~\\[32pt]
\centering
\parbox{0.495\textwidth}{
  \centering
  \includegraphics[width=\linewidth,page=4]{graphics/rpSim/Apertures_swapedAxes_withFit_beforeDxShift.pdf}
}~
\parbox{0.495\textwidth}{
  \centering
  \includegraphics[width=\linewidth,page=4]{graphics/rpSim/Apertures_swapedAxes_withFit.pdf}
}%
\end{figure}
%---------------------------

%% ===== DE/DX ADJUSTMENT ====
%%===========================================================%%
%%                                                           %%
%%                     DE/DX ADJUSTMENT                      %%
%%                                                           %%
%%===========================================================%%


\chapter{\texorpdfstring{d$\bm{E}$/d$\bm{x}$}{dE/dx} adjustment}\label{chap:dEdxCorrection}

Particle identification in our analyses is done using merged information from the TPC (specific energy loss of tracks $dE/dx$) and from the TOF (time of hit macthed to TPC track). As can be seen in Fig.~\ref{fig:dEdxDataVsMC}, $dE/dx$ information from the MC events simulated in STARsim (in red) poorely matches the data points (black). This results e.g. in large systematic error of estimate of particle identification efficiency.

This problem was discussed under ticket \#3272~(Ref.~\cite{dedxTicket}). There were trials to improve the TPC calibration in simulation, but the problem remained. It was finally concluded that the origin of the problem lies in the model of energy loss used in the STARsim, therefore any further action was postponed. 

In order to tune simulated reposponse of the TPC in terms of $dE/dx$, hence also reduce the systematic uncertainty related to particle identification, a correction method was developed based on proper transformation (recalculation) of simulated $dE/dx$ to obtain new $dE/dx$ whose distribution matches the data.
% It is possible to transform dE/dx in MC to make it follow the shape of dE/dx in the data. 
We know that $n^{\sigma}_{X}$ (where $X=\pi, K, p$, ...) variable for particle $X$ follows a gaussian distribution
\begin{equation}n^{\sigma}_{X} = \Big( \ln{\frac{dE/dx}{\langle dE/dx\rangle_{X}}} \Big) / \sigma_{dE/dx},~~~~~f(n^{\sigma}_{X}) = \mathcal{N}(n^{\sigma}_{X}; \mu=0,\sigma=1),\end{equation}
therefore $dE/dx$ itself by definition follows log-normal distribution:
\begin{equation}f(dE/dx) = \mathcal{L}og\mathcal{N}(dE/dx; \mu=\langle dE/dx\rangle,\sigma=\sigma_{dE/dx}) = \frac{1}{\sqrt{2\pi}\cdot \sigma\cdot dE/dx}e^{-\frac{\ln^{2}{\frac{dE/dx}{\langle dE/dx\rangle}}}{2\sigma^{2}}}.\end{equation}
The desired transformation should preserve the shape of $dE/dx$ distribution so that it is still described by $\mathcal{L}og\mathcal{N}$, however it should change $\mu$ and $\sigma$ so that these values are euqal to ones in the data. The transformation that satisfies above postulate is
\begin{equation}dE/dx' = c\cdot (dE/dx)^{a}.\label{eq:dedxTranformation}\end{equation}
Parameters of the distribution $\mathcal{L}og\mathcal{N}(dE/dx')$ are then
\begin{equation}\mu' = c\cdot\mu^{a},~~~~\sigma' = a\cdot\sigma.\end{equation}
From above we get formulae for parameters of the transformation:
\begin{equation}a=\sigma'/\sigma,~~~~c = \frac{\mu'}{\mu^{a}}.\label{eq:dedxPatameters}\end{equation}%
%
To sum up, one has to find the MPV and width parameter of the $dE/dx$ spectrum of each particle in the data and MC, and use relations \eqref{eq:dedxPatameters} in order to find parameters of the transformation introduced in Eq.~\eqref{eq:dedxTranformation}.

The most challenging part of the task was extraction of the $\langle dE/dx\rangle$ and $\sigma_{dE/dx}$ from the data. In case of MC one can select tracks matched to true-level particles of given ID and thus separate $dE/dx$ of different particles, which makes extraction of the distribution shape straightforward. Unfortunately, it is not possible to apply the same method to the data - here one has to deal with overlapping of the reconstructed $dE/dx$ from different particles. Therefore fits of sum of $f(dE/dx)$ corresponding to different particles were performed to reconstructed track $dE/dx$ in narrow momentum bins. The width of momentum bins was chosen to compromize statistics and validity of assumption of constant parameters of $dE/dx$ distribution over bin range.

It was found during the fitting that log-normal distribution is not a perfect model of the reconstructed $dE/dx$. The problems with description of the data were mainly in the tail-part of the distribution from single particle. Precise model was necessary to obtain satistactory quality of fits and trustworthy values of parameters. After some reasearch the best model of $dE/dx$ distribution from single particle was found to be%  Eq.~\eqref{eq:expTail}:%
%
\begin{equation}\label{eq:expTail}
	f(dE/dx)=\left\{
                \begin{array}{ll}
                  \frac{A}{\sqrt{2\pi}\cdot \sigma\cdot dE/dx}\exp{\Bigg(-\frac{1}{2}\Big(\frac{\ln{\frac{dE/dx}{\langle dE/dx\rangle}}}{\sigma}\Big)^{2}\Bigg)} & \textrm{for}~\frac{\ln{\frac{dE/dx}{\langle dE/dx\rangle}}}{\sigma} \leq k \\
                  \frac{A}{\sqrt{2\pi}\cdot \sigma\cdot dE/dx}\exp{\Bigg(-k\cdot \frac{\ln{\frac{dE/dx}{\langle dE/dx\rangle}}}{\sigma} + \frac{1}{2}k^{2} \Bigg)} & \textrm{for}~\frac{\ln{\frac{dE/dx}{\langle dE/dx\rangle}}}{\sigma} > k.
                \end{array}
              \right.
\end{equation}%
%
Such form was motivated by the function presented in Ref.~\cite{AlternativeToCrystallBall}, here adopted for the log-normal instead of normal distribution. Because the modification of the log-normal distribution is introduced only at high-end tail, the validity of the transformation discussed above still holds. To reduce fit complexity the $k$ parameter was set the same for all particle species and fixed  to value equal $2.2$, which worked well for both data and embedded MC. Particles and their anti-particles were assumed to have the same $dE/dx$ distributions for a given momentum and were analyzed together. The sample fit in a single momentum bin can be found in Fig.~\ref{fig:dEdxFit}. Fits in all momentum bins can be found in Appendix~\ref{appendix:dEdxAdjustment}.
%
%
%---------------------------
\begin{figure}[ht!]%
\centering%
\includegraphics[width=\linewidth,page=12]{graphics/dedx/dEdx_fitPerMomentumBin_4thIteration.pdf}\vspace{-5pt}%
\caption[Sample fit to dE/dx spectrum in the data in single momentum bin.]%
{Sample fit of sum of functions from Eq.~\eqref{eq:expTail} corresponding to different particle species to dE/dx spectra in the data in a single momentum bin.}\label{fig:dEdxFit}\vspace{10pt}
\end{figure}
%---------------------------
%
%
%---------------------------
\begin{figure}[hb!]\vspace{-45pt}
\centering
\parbox{0.4725\textwidth}{
  \centering
  \begin{subfigure}[b]{\linewidth}{
                \subcaptionbox{\label{fig:dEdxMeanOffsetMC}}{\includegraphics[width=\linewidth]{graphics/dedx/dEdxMeanOffset_allPIDs.pdf}\vspace*{-10pt}}}
  \end{subfigure}\\[-3pt]
  \begin{subfigure}[b]{\linewidth}\addtocounter{subfigure}{1}{
                \subcaptionbox{\label{fig:dEdxMeanOffsetData}}{\includegraphics[width=\linewidth]{graphics/dedx/dEdxMeanOffset_allPIDs_data.pdf}\vspace*{-10pt}}}
  \end{subfigure}
}
\quad
\parbox{0.4725\textwidth}{
  \centering
  \begin{subfigure}[b]{\linewidth}\addtocounter{subfigure}{-2}{
                \subcaptionbox{\label{fig:dEdxWidthMC}}{\includegraphics[width=\linewidth]{graphics/dedx/dEdxWidth_allPIDs.pdf}\vspace*{-10pt}}}
  \end{subfigure}\\[-3pt]
  \begin{subfigure}[b]{\linewidth}\addtocounter{subfigure}{1}{
                \subcaptionbox{\label{fig:dEdxWidthData}}{\includegraphics[width=\linewidth]{graphics/dedx/dEdxWidth_allPIDs_data.pdf}\vspace*{-10pt}}}
  \end{subfigure}
}\vspace{-5pt}%
\caption[Parameters of reconstructed track dE/dx as a function of reconstructed momentum for a few particle species.]{Difference between MPV of dE/dx predicted by Bichsel parametrization and obtained from the fit of Eq.~\eqref{eq:expTail} to dE/dx distribution in the data (\ref{fig:dEdxMeanOffsetData}) and MC sample (\ref{fig:dEdxMeanOffsetMC}) and dE/dx width parameter in data (\ref{fig:dEdxWidthData}) and MC (\ref{fig:dEdxWidthMC}) as a function of reconstructed particle momentum for a few particle species. Solid lines represent fits to points of corresponding color. Only statistical errors are shown.}\label{fig:dEdxParametersMC}
\end{figure}
%---------------------------


Results of the fits for all considered particle species (pions, kaons, protons, electrons and deuterons) are commonly presented in Fig.~\ref{fig:dEdxParametersMC} with color markers. Figures \ref{fig:dEdxMeanOffsetData} and~\ref{fig:dEdxMeanOffsetMC} show the offset of the MPV of reconstructed $dE/dx$ relative to the Bichsel parametrization in the data and embedded MC, respectively, and Fig.~\ref{fig:dEdxWidthData} and~\ref{fig:dEdxWidthMC} show the width of reconstructed $dE/dx$ (in the same order). Function able to qualitatively describe dependence of the parameters as a function of track momentum was empirically found to be given by Eq.~\eqref{eq:dEdxParametrization}:
%
\begin{equation}\label{eq:dEdxParametrization}
	g(p) = P_{1} + P_{2}\cdot \exp{\left(-P_{3}\cdot p\right)} + P_{4}\cdot \arctan{\big(P_{5}\cdot(p-P_{6})\big)}
\end{equation}
%
This function was fitted to points corresponding to each particle type and fit result is shown in Fig.~\ref{fig:dEdxParametersMC} with lines colored in accordance to markers. Values of parameters of above function are tabulated in Tab.~\ref{tab:dEdxParametersMC}.

\begin{table}[hb!]\vspace{5pt}\centering%
\subcaptionbox{\label{tab:dEdxParametersMC}}{\centering%
 \begin{tabular}{r||c|c|c|c|c|c||c|c|c|c|c|c}%\hline
 \multirow{2}{*}{\textbf{PID}} &  \multicolumn{6}{c||}{\bm{$\langle dE/dx\rangle_{\textrm{\textbf{Bichsel}}} - \langle dE/dx\rangle_{\textrm{\textbf{MC}}}$}} & \multicolumn{6}{c}{\bm{$\sigma(dE/dx)_{\textrm{\textbf{MC}}}$}} \\ \cline{2-13}%
  & $P_{1}$ & $P_{2}$ & $P_{3}$ & $P_{4}$ & $P_{5}$ & $P_{6}$ & $P_{1}$ & $P_{2}$ & $P_{3}$ & $P_{4}$ & $P_{5}$ & $P_{6}$ \\ \Xhline{2\arrayrulewidth}
$\bm{\pi^{\pm}}$ &\scriptsize 7.183e-8 &\scriptsize -1.647e-4 &\scriptsize 41.68 &\scriptsize          &\scriptsize         &\scriptsize          &\scriptsize 0.0705 &\scriptsize      &\scriptsize      &\scriptsize -1.42e-3 &\scriptsize 9.860 &\scriptsize 0.951 \\ \hline
$\bm{K^{\pm}}$ &\scriptsize 4.359e-8 &\scriptsize -9.285e-6 &\scriptsize 7.697 &\scriptsize          &\scriptsize         &\scriptsize         &\scriptsize 0.0511 &\scriptsize 0.034 &\scriptsize 1.675 &\scriptsize 1.01e-2 &\scriptsize 4.934 &\scriptsize 0.528 \\ \hline
$\bm{p,\bar{p}}$ &\scriptsize 3.556e-8 &\scriptsize -8.621e-6 &\scriptsize 3.980 &\scriptsize          &\scriptsize         &\scriptsize         &\scriptsize 0.0630 &\scriptsize -7.725 &\scriptsize 27.17 &\scriptsize 3.37e-3 &\scriptsize 5.245 &\scriptsize 0.670 \\ \hline
$\bm{e^{\pm}}$ &\scriptsize -6.219e-8 &\scriptsize 2.065e-7 &\scriptsize 3.241 &\scriptsize          &\scriptsize         &\scriptsize           &\scriptsize 0.0354 &\scriptsize 0.982 &\scriptsize 26.58 &\scriptsize 1.79e-2 &\scriptsize 41.515 &\scriptsize 0.095 \\ \hline
$\bm{d,\bar{d}}$ &\scriptsize -1.305e-6 &\scriptsize -5.268e-6 &\scriptsize 3.486 &\scriptsize ~~~~~~~~~~~ &\scriptsize ~~~~~~~ &\scriptsize ~~~~~~ &\scriptsize 0.0967 &\scriptsize -1526 &\scriptsize 18.75 &\scriptsize          &\scriptsize         &\scriptsize         
\end{tabular}%
}\\ \centering%
\subcaptionbox{\label{tab:dEdxParametersData}}{%
\begin{tabular}{r||c|c|c|c|c|c||c|c|c|c|c|c}%\hline
 \multirow{2}{*}{\textbf{PID}} &  \multicolumn{6}{c||}{\bm{$\langle dE/dx\rangle_{\textrm{\textbf{Bichsel}}} - \langle dE/dx\rangle_{\textrm{\textbf{Data}}}$}} & \multicolumn{6}{c}{\bm{$\sigma(dE/dx)_{\textrm{\textbf{Data}}}$}} \\ \cline{2-13}
  & $P_{1}$ & $P_{2}$ & $P_{3}$ & $P_{4}$ & $P_{5}$ & $P_{6}$ & $P_{1}$ & $P_{2}$ & $P_{3}$ & $P_{4}$ & $P_{5}$ & $P_{6}$ \\ \Xhline{2\arrayrulewidth}
 $\bm{\pi^{\pm}}$ & \scriptsize -1.399e-8 & \scriptsize 2.012e-7 & \scriptsize10.39 & \scriptsize&             \scriptsize&                \scriptsize&         \scriptsize 0.0734 & \scriptsize1.907 & \scriptsize31.86 & \scriptsize-8.20e-4 & \scriptsize 22.788 & \scriptsize 0.653\\ \hline
 $\bm{K^{\pm}}$ & \scriptsize    2.325e-9 & \scriptsize -3.690e-6 & \scriptsize 8.712 & \scriptsize&           \scriptsize&               \scriptsize&          \scriptsize 0.0808 &  \scriptsize -0.040 & \scriptsize7.951 & \scriptsize 5.62e-3 & \scriptsize -17.08 &  \scriptsize 0.269\\ \hline
 $\bm{p,\bar{p}}$ & \scriptsize -1.458e-7 & \scriptsize 0.6655 &   \scriptsize59.06 & \scriptsize 1.171e-7 &    \scriptsize 4.660 &       \scriptsize 0.644 &   \scriptsize0.0795 & \scriptsize 0.181 & \scriptsize12.12 & \scriptsize& \scriptsize& \scriptsize\\ \hline
 $\bm{e^{\pm}}$ & \scriptsize   9.005e-8 &  \scriptsize 2.494e-7 & \scriptsize 8.834 & \scriptsize&              \scriptsize&              \scriptsize&          \scriptsize 0.0680 & \scriptsize 8.8e-4 & \scriptsize 1.549 & \scriptsize& \scriptsize& \scriptsize\\ \hline
 $\bm{d,\bar{d}}$ & \scriptsize -1.910e-7 & \scriptsize 5.637e-3 &  \scriptsize 14.48 & \scriptsize&               \scriptsize&              \scriptsize&          \scriptsize 0.1161 & \scriptsize -0.147 & \scriptsize 2.890 & \scriptsize& \scriptsize& \scriptsize%\\ \hline
\end{tabular}%
}\vspace{-7pt}\caption[Parameters of functions from Fig.~\ref{fig:dEdxParametersMC} describing reconstructed track dE/dx as a function of reconstructed momentum for a few particle species (STARsim MC).]{Parameters of functions from Fig.~\ref{fig:dEdxParametersMC} describing reconstructed track dE/dx as a function of reconstructed momentum for a few particle species. Blank cells denote parameters equal 0. Units of parameters $P_{i}$ are such that if one provides momentum in Eq.~\eqref{eq:dEdxParametrization} in GeV/$c$ the resultant offset of dE/dx MPV with respect to Bichsel parametrization is in GeV/cm, and the resultant $\sigma$ parameter is unitless.}\label{tab:dEdxParameters}
\end{table}%
%
%
%---------------------------
\begin{figure}[hb!]\vspace{-5pt}%
\centering%
\includegraphics[width=\linewidth,page=13]{graphics/dedx/dEdx_DataVsMC.pdf}\vspace{-5pt}%
\caption[Sample comparison of dE/dx spectrum between data and embedded MC in single momentum bin.]{Sample comparison of dE/dx spectrum between data and embedded MC (before and after $dE/dx$ adjustment) in a single momentum bin. Lower pad shows the ratio between embedded MC and data before and after $dE/dx$ adjustment. In both upper and lower padt the same color code is used. Only statistical errors are shown.}\label{fig:dEdxDataVsMCSingleBin}
\end{figure}
%---------------------------
%
The correctness of the entire procedure described in this section was verified by comparing the reconstructed track $dE/dx$ between the data and embedded MC without and with the $dE/dx$ transformed using Eq.~\eqref{eq:dedxTranformation} and parameters from Tab.~\ref{tab:dEdxParameters}. Some difficulty arised in this comparison due to inconsistent relative content of different particle species in the data and embedded MC sample. Problem was ressolved by separating $dE/dx$ histograms of different particle species (in the same way as it was done for extraction of $dE/dx$ MPV and $\sigma$ for each particle ID) and fitting the sum of histograms from different particle types to the data histogram (in momentum bins). The only free parameters in the fit were relative contents of histogram from singe particle type to the data histogram. A sample comparison between the $dE/dx$ in data and embedded MC is presented in Fig.~\ref{fig:dEdxDataVsMCSingleBin}. Comparison in all other momentum bins is contained in Appendix~\ref{appendix:dEdxAdjustment} (Fig.~\ref{fig:dEdxDataVsMC}). Fits were done for adjusted $dE/dx$ (filled green). Histograms for unadjusted $dE/dx$ (hashed red) were composed using the same relative content of particles as obtained from the fit of adjusted $dE/dx$. The ratio of the MC to the data shown in the lower pad of Fig.~\ref{fig:dEdxDataVsMCSingleBin} and Fig.~\ref{fig:dEdxDataVsMC} clearly demonstrates better agreement of the MC and the data after the adjustent in terms of position and width of peaks in $dE/dx$ spectrum.

%% =====  TPC TRACK POINTING RESOLUTION ADJUSTMENT ====
%%===========================================================%%
%%                                                           %%
%%          TPC TRACK POINTING RESOLUTION ADJUSTMENT         %%
%%                                                           %%
%%===========================================================%%


\chapter{TPC track pointing resolution adjustment}\label{chap:tpcTrackPointingRes}

It was found during the analysis that distributions of quantities which describe the pointing resolution of the TPC tracks do not agree well between the data and embedded MC. Namely, the resolutions of the global helices associated with the tracks were found to be significantly better in the STAR simulation than in the data, what manifests as narrower DCA and $d_{0}$ distribution in the embedded MC, comparing to corresponding distribution in the data (Fig.~\ref{fig:pointingResComp}). This issue was discussed under ticket \#3332~(Ref.~\cite{dcaTicket}).

This problem could affect the momentum resolution and thus all other resolutions and reponse matrices used in data unfolding. Therefore the resolution adjustment procedure was performed to find appropriate parameters of the ``artificial'' helix deterioration and finally obtain agreement between DCA and $d_{0}$ distributions (and all related resolutions) in the data and embedded MC.

In order to reduce pointing resolution in the MC an additional smearing of the helix radius $\sigma(R)$ was introduced. Based on $d_{0}$ comparison in~Fig.~\ref{fig:d0} it was decided to account also for the systematic bias of the helix radius $\Delta\mu(R)$\footnote{Transverse impact parameter $d_{0}$ takes positive value if the beamline is contained inside the helix (in the $yz$-plane projection), otherwise it is negative. Any asymmetry in the $d_{0}$ distribution in the MC with respect to the data indicates presence of systematic difference in reconstructed $d_{0}$, hence also in reconstructed $R$.}, which may be present e.g. due to differences in the material budget used the simulation and reconstruction. Both smearing and bias of the helix radius were introduced only for MC tracks which were matched with the true-level particles since only simulated tracks require adjustment (tracks from zero-bias event used in embedding already contain all detector effects).

%---------------------------
\begin{wrapfigure}{i}{0.365\textwidth}\vspace*{-9pt}
  \centering
  \includegraphics[width=0.365\textwidth]{graphics/tpcHelixAdj/trackSmearing.pdf}
  \caption[Sketch of helix modification procedure and $d_{0}$ calculation.]
   {Sketch of helix modification procedure and $d_{0}$ calculation.}
   \label{fig:d0sketch}%\vspace*{-29pt}
\end{wrapfigure}
%---------------------------

Extraction of $\Delta\mu(R)$ and $\sigma(R)$ parameter required to achieve agreement of pointing resolution between embedded MC and the data involved a few steps, as listed below:
  \begin{enumerate}
   \item Series of $d_{0}$ histograms in bins of $p_{T}$ (100~MeV/$c$ wide) was prepared, each for different size of distortion (different $\Delta\mu(R)$ and $\sigma(R)$) of global helix of the TPC tracks matched with true-level particles (example plot in single $p_{T}$ bin is shown in Fig.~\ref{fig:d0ForChiSqMin}):
   \begin{enumerate}
   \item for each set of parameters $\Delta\mu(R)$ and $\sigma(R)$ the helix radius $R$ was recalculated independently for each track following the Eq.~\eqref{eq:radiusRecalc}:\vspace{-5pt}
   \begin{equation}\label{eq:radiusRecalc}R'=R\times \mathcal{N}\Big(1+\Delta\mu(R),~\sigma(R)\Big),\vspace{-5pt}\end{equation}
   \item new helix of a radius $R'$ was assigned to a track and used to calculate $d_{0}$. The modified helix was obtained by changing the radius of original helix from $R$ to $R'$ with a fixed middle point between the first and last TPC hit of a global track represented by the helix (Fig.~\ref{fig:d0sketch}). The momentum of the track was also recalculated:\vspace{-5pt}
   \begin{equation}p_{T}'=p_{T}\times \frac{R'}{R},~~~~~~~\eta'=\eta\times \frac{R'}{R}.\vspace{-5pt}\end{equation}
   \end{enumerate}%
  \end{enumerate}%
 
  \begin{enumerate}\setcounter{enumi}{1}
   \item In each $p_{T}$ bin the $\chi^{2}/\text{NDF}$ was calculated between the data and MC $d_{0}$ histogram in a range -1.5~cm~$<d_{0}<$~1.5~cm (corresponding to $d_{0}$ cut used in analyses), for every point in parameter space of radius distortion (for every set of $\Delta\mu(R)$ and $\sigma(R)$). An example (single $p_{T}$ bin) of map of $-\chi^{2}/\text{NDF}$ in a~parameter space is presented in~Fig~\ref{fig:chiSqPerNdfTpcResAdj}.
   \item In each bin of recalculated $p_{T}$ the 2-dim parabola $z\left(x,y;~a,b,x_{0},y_{0},z_{0}\right)$ given in Eq.~\eqref{eq:parabolaChiSq} ($z=\chi^{2}/\text{NDF},~x=\Delta\mu(R),~y=\sigma(R)$) was fitted to $-\chi^{2}/\text{NDF}$ in the global minimum region to obtain the best-fit distortion parameters.\vspace{-5pt}
   \begin{equation}\label{eq:parabolaChiSq}  z=z_{0}-a(x-x_{0})^{2}-b(y-y_{0})^{2}.\vspace{-5pt}\end{equation}
   \item The best-fit smearing $\sigma(R)$ (equal to parabola parameter $y_{0}$) and best-fit bias $\Delta\mu(R)$ ($x_{0}$) from individual $p_{T}$ bins was plotted as a function of global track $p_{T}$ (Fig.~\ref{fig:distortionVsPt}). Each point was assigned with an error being a quadratic sum of two components: the error on $x_{0}$ ($y_{0}$) resulting from the parabola fit to $-\chi^{2}/\text{NDF}$, and length of corresponding semi-axis of ellipsis formed by the intersection of fitted parabola with the $xy$-plane at $z=z_{0}-1/\text{NDF}$ (from definition of the parameter uncertainty given by the change of overall $\chi^{2}$ by 1 unit). Resultant formulae for the error of each individual point in Fig.~\ref{fig:distortionVsPt} are%\vspace{-5pt}
   \begin{multicols}{2}~\\[-30pt]
    \begin{equation}\label{eq:errorDeltaMuR}  \delta\left(\Delta\mu(R)\right)=\sqrt{\delta_{\text{fit}}^{2}(x_{0}) + \frac{1}{2a\text{NDF}}},\end{equation}
   \break\\[-60pt]
    \begin{equation}\label{eq:errorSigmaR} \delta\left(\sigma(R)\right)=\sqrt{\delta_{\text{fit}}^{2}(y_{0}) + \frac{1}{2b\text{NDF}}}.\vspace{-5pt}\end{equation}
    \end{multicols}\vspace*{-7pt}
    From Fig.~\ref{fig:d0ForChiSqMin} one can read that $\text{NDF}=14$. In calculation of uncertainties correlation of $\Delta\mu(R)$ and $\sigma(R)$ have not been accounted.
   \item The empirically determined functions were fitted to points representing $\Delta\mu(R)$ and $\sigma(R)$ dependence on the global track $p_{T}$. Their form and values of parameters are given in Fig.~\ref{fig:distortionVsPt}.
  \end{enumerate}%
  
%---------------------------
\begin{figure}[t!]%
\centering%
\begin{minipage}{.4725\textwidth}%
  \centering%
  \includegraphics[width=\linewidth,page=5]{graphics/tpcHelixAdj/D0ComparisonForChiSqMinimization.pdf}%
  \caption[Example of comparison of $d_{0}$ histograms in the data and embedded MC in the procedure of TPC pointing resolution adjustment.]{Example of comparison of $d_{0}$ histograms in single $p_{T}$ bin in the data (black points) and embedded MC (colored lines) in the procedure of TPC pointing resolution adjustment. MC histograms only for $\Delta\mu(R)=0$ and $\sigma(R)=0$, $5\times10^{-3}$ and $10^{-2}$ were shown for explanatory purposes.}\label{fig:d0ForChiSqMin}
\end{minipage}%
\quad\quad%
\begin{minipage}{.4725\textwidth}%
  \centering
  \includegraphics[width=\linewidth,page=3]{graphics/tpcHelixAdj/ChiSqVsSmearingVsBias.pdf}%
  \caption[Example of $-\chi^{2}/\text{NDF}$ map in a parameter space in the procedure of TPC pointing resolution adjustment.]{Example of $-\chi^{2}/\text{NDF}$ map in a parameter space in the procedure of TPC pointing resolution adjustment. The red surface represents parabola fitted in the vicinity of the global minimum.\newline\newline}\label{fig:chiSqPerNdfTpcResAdj}
\end{minipage}%
\end{figure}%
%---------------------------

  
%---------------------------
\begin{wrapfigure}{o}{0.465\textwidth}\vspace*{-15pt}
  \centering
  ~\includegraphics[width=0.465\textwidth]{graphics/tpcHelixAdj/DistortionVsPt.pdf}\vspace*{-5pt}
  \caption[Best-fit parameters obtained in the procedure of the TPC track pointing resolution adjustment.]
   {Best-fit parameters obtained in the procedure of the TPC track pointing resolution adjustment. Uncertainties on parameters resulting solely from the fit of Eq.~\eqref{eq:parabolaChiSq} to $-\chi^{2}/\text{NDF}$ are represented by the lines with perpendicular endings. Total uncertainties (Eqs.~\eqref{eq:errorDeltaMuR},~\eqref{eq:errorSigmaR}) extend beyond. The empirical functions fitted to points are drawn with corresponding colors, and formula of each is written aside.}
   \label{fig:distortionVsPt}%\vspace*{-29pt}
\end{wrapfigure}
%---------------------------
Helices of global TPC tracks were deteriorated according to Eq.~\eqref{eq:radiusRecalc} and the parametrizations of global track $p_{T}$-dependence of $\Delta\mu(R)$ and $\sigma(R)$ from Fig.~\ref{fig:distortionVsPt}, to verify if better agreement between the data and embedded MC is found after the adjustment. Filled histograms in Fig.~\ref{fig:pointingResComp} show $d_{0}$ and DCA distributions after the described adjustment, and filled circles in the bottom pad show their ratio to the data points. Clearly, there is much better agreement between embedded MC and the data after the pointing resolution adjustment. Remaining differences may arise from incomplete theoretical model of the CEP process impleneted in GenEx leading to different $p_{T}$ spectra of the data and the model (e.g. model does not contain resonant $\pi^{+}\pi^{-}$ production).





%---------------------------
\begin{figure}[ht]
\centering
\parbox{0.4725\textwidth}{
  \centering
  \begin{subfigure}[b]{\linewidth}
                \subcaptionbox{\label{fig:d0}}{\includegraphics[width=\linewidth]{graphics/tpcHelixAdj/d0_beforeAfterCorrection.pdf}}
  \end{subfigure}\\
  \begin{subfigure}[b]{\linewidth}\addtocounter{subfigure}{1}
                \subcaptionbox{\label{fig:dcaX}}{\includegraphics[width=\linewidth]{graphics/tpcHelixAdj/DcaX_beforeAfterCorrection.pdf}}
  \end{subfigure}\\
  \begin{subfigure}[b]{\linewidth}\addtocounter{subfigure}{1}
                \subcaptionbox{\label{fig:dcaZ}}{\includegraphics[width=\linewidth]{graphics/tpcHelixAdj/LongitudinalDCA_beforeAfterCorrection.pdf}}
  \end{subfigure}
}%
\quad\quad%
\parbox{0.4725\textwidth}{
  \centering
  \begin{subfigure}[b]{\linewidth}\addtocounter{subfigure}{-4}
                \subcaptionbox{\label{fig:dcaR}}{\includegraphics[width=\linewidth]{graphics/tpcHelixAdj/RadialDCA_beforeAfterCorrection.pdf}}
  \end{subfigure}\\
  \begin{subfigure}[b]{\linewidth}\addtocounter{subfigure}{1}
                \subcaptionbox{\label{fig:dcaY}}{\includegraphics[width=\linewidth]{graphics/tpcHelixAdj/DcaY_beforeAfterCorrection.pdf}}
  \end{subfigure}
  \begin{minipage}[t][1.042\linewidth][t]{\linewidth}\vspace{10pt}
    \caption[Comparison of distribution of pion $d_{0}$ and components of DCA vector in the data and embedded MC, before and after adjustment of TPC pointing resolution.]
    {Comparison of distribution of pion transverse impact parameter $d_{0}$~(\ref{fig:d0}) and transverse~(\ref{fig:dcaR}), $x$-~(\ref{fig:dcaX}), $y$-~(\ref{fig:dcaY}) and $z$-component (\ref{fig:dcaZ}) of the DCA vector between the global helix and primary vertex in the data (CEP) and embedded MC (GenEx). Distributions for unadjusted helices are drawn as hashed histograms, while filled histograms are for adjusted helices. Normalizations of the signal and backgrounds were established from the comparison of $p_{T}^{\textrm{miss}}$ and $\Delta\theta$ distributions after full selection (without cut on the presented quantity and without exclusivity cut), as described in Sec.~XXX of~Ref.~\cite{AnalysisNoteRafal}. Red dashed lines and red arrows indicate the range of each quantity which is accepted in analyses.}\label{fig:pointingResComp}
  \end{minipage}
}%

\end{figure}
%---------------------------


% %% =====  BACKGROUNDS ====
%%===========================================================%%
%%                                                           %%
%%              DEAD MATERIAL IN FORNT OF TPC                %%
%%                                                           %%
%%===========================================================%%


\newcommand{\itemm}{\item\hspace*{-5pt}.\hspace*{-1pt}~}

\chapter{Dead material in front of TPC}\label{chap:deadMaterial}
 
% %% =====  SYSTEMATIC ERRORS ====
%%===========================================================%%
%%                                                           %%
%%                   SYSTEMATIC ERRORS                       %%
%%                                                           %%
%%===========================================================%%


\chapter{Systematic errors}\label{chap:systematicErrors}
\section{TPC track reconstruction efficiency systematics}\label{sec:tpcSystematics}
\section{TOF matching efficiency systematics}\label{sec:tofSystematics}
\section{Dead material correction to TPC track reconstruction efficiency}\label{sec:deadMaterialSystematics}

\begin{figure}[hb]
\caption[...]{...}\label{fig:dead_materialCD3D}
\centering
\parbox{0.495\textwidth}{
  \centering
  \includegraphics[width=\linewidth,page=1]{graphics/systematicsEfficiency/deadMaterial/secondaries_Unbinned_CD_.pdf}\\
  \includegraphics[width=\linewidth,page=3]{graphics/systematicsEfficiency/deadMaterial/secondaries_Unbinned_CD_.pdf}\\
  \includegraphics[width=\linewidth,page=5]{graphics/systematicsEfficiency/deadMaterial/secondaries_Unbinned_CD_.pdf}\\
  \includegraphics[width=\linewidth,page=7]{graphics/systematicsEfficiency/deadMaterial/secondaries_Unbinned_CD_.pdf}\\
}~
\parbox{0.495\textwidth}{
  \centering
  \includegraphics[width=\linewidth,page=2]{graphics/systematicsEfficiency/deadMaterial/secondaries_Unbinned_CD_.pdf}\\
  \includegraphics[width=\linewidth,page=4]{graphics/systematicsEfficiency/deadMaterial/secondaries_Unbinned_CD_.pdf}\\
  \includegraphics[width=\linewidth,page=6]{graphics/systematicsEfficiency/deadMaterial/secondaries_Unbinned_CD_.pdf}\\
  \includegraphics[width=\linewidth,page=8]{graphics/systematicsEfficiency/deadMaterial/secondaries_Unbinned_CD_.pdf}
}%
\end{figure}
\begin{figure}[hb]\ContinuedFloat
% ~\\[32pt]
\centering
\parbox{0.495\textwidth}{
  \centering
  \includegraphics[width=\linewidth,page=9]{graphics/systematicsEfficiency/deadMaterial/secondaries_Unbinned_CD_.pdf}\\
  \includegraphics[width=\linewidth,page=11]{graphics/systematicsEfficiency/deadMaterial/secondaries_Unbinned_CD_.pdf}\\
  \includegraphics[width=\linewidth,page=13]{graphics/systematicsEfficiency/deadMaterial/secondaries_Unbinned_CD_.pdf}\\
  \includegraphics[width=\linewidth,page=15]{graphics/systematicsEfficiency/deadMaterial/secondaries_Unbinned_CD_.pdf}\\
}~
\parbox{0.495\textwidth}{
  \centering
  \includegraphics[width=\linewidth,page=10]{graphics/systematicsEfficiency/deadMaterial/secondaries_Unbinned_CD_.pdf}\\
  \includegraphics[width=\linewidth,page=12]{graphics/systematicsEfficiency/deadMaterial/secondaries_Unbinned_CD_.pdf}\\
  \includegraphics[width=\linewidth,page=14]{graphics/systematicsEfficiency/deadMaterial/secondaries_Unbinned_CD_.pdf}\\
  \includegraphics[width=\linewidth,page=16]{graphics/systematicsEfficiency/deadMaterial/secondaries_Unbinned_CD_.pdf}
}%
\end{figure}
\begin{figure}[hb]
\caption[...]{...}\label{fig:dead_materialCD1D}
\centering
\parbox{0.495\textwidth}{
  \centering
  \includegraphics[width=\linewidth,page=1]{graphics/systematicsEfficiency/deadMaterial/secondaries_Unbinned_CD_1D.pdf}\\
  \includegraphics[width=\linewidth,page=2]{graphics/systematicsEfficiency/deadMaterial/secondaries_Unbinned_CD_1D.pdf}\\
  \includegraphics[width=\linewidth,page=3]{graphics/systematicsEfficiency/deadMaterial/secondaries_Unbinned_CD_1D.pdf}\\
  \includegraphics[width=\linewidth,page=1]{graphics/systematicsEfficiency/deadMaterial/secondaries_Unbinned_Charged_CD1D.pdf}\\
}~
\parbox{0.495\textwidth}{
  \centering
  \includegraphics[width=\linewidth,page=4]{graphics/systematicsEfficiency/deadMaterial/secondaries_Unbinned_CD_1D.pdf}\\
  \includegraphics[width=\linewidth,page=5]{graphics/systematicsEfficiency/deadMaterial/secondaries_Unbinned_CD_1D.pdf}\\
  \includegraphics[width=\linewidth,page=6]{graphics/systematicsEfficiency/deadMaterial/secondaries_Unbinned_CD_1D.pdf}\\
  \includegraphics[width=\linewidth,page=2]{graphics/systematicsEfficiency/deadMaterial/secondaries_Unbinned_Charged_CD1D.pdf}
}%
\end{figure}





% %% ===== DODATKI ===== ------------
\begin{appendices}
\makeatletter
\renewcommand{\@makechapterhead}[1]{%
 \vspace*{-18\p@}%
  {\parindent \z@ \raggedright
%     \LARGE \bfseries \thechapter. #1\par\nobreak
%     \vskip 40\p@
      \Huge \bfseries Appendix \thechapter\newline #1\par\nobreak
      \vskip 20\p@
  }}
\makeatother

% %% =====  TPC EFFICIENCY ====
%%===========================================================%%
%%                                                           %%
%%                   DE/DX ADJUSTMENT APPENDIX               %%
%%                                                           %%
%%===========================================================%%

\chapter{TPC track reconstruction efficiency}\label{appendix:tpcEff}


%---------------------------
\begin{figure}[hb]
\caption[TPC acceptance and reconstruction efficiency of $\pi^{-}$.]{TPC acceptance and reconstruction efficiency of $\pi^{-}$. Each plot represents the TPC efficiency $\epsilon_{\text{TPC}}$ ($z$-axis) as a function of true particle pseudorapidity $\eta$ ($x$-axis) and transverse momentum $p_{T}$ ($y$-axis) in single $z$-vertex bin whose range is given at the top. Red lines and arrows indicate region accepted in analyses.}\label{fig:tpcEff_pion_minus}
\centering
\parbox{0.495\textwidth}{
  \centering
  \includegraphics[width=\linewidth,page=3]{graphics/eff/Eff2D_TPC_pion_Minus.pdf}\\
  \includegraphics[width=\linewidth,page=5]{graphics/eff/Eff2D_TPC_pion_Minus.pdf}\\
  \includegraphics[width=\linewidth,page=7]{graphics/eff/Eff2D_TPC_pion_Minus.pdf}\\
  \includegraphics[width=\linewidth,page=9]{graphics/eff/Eff2D_TPC_pion_Minus.pdf}
}~
\parbox{0.495\textwidth}{
  \centering
  \includegraphics[width=\linewidth,page=4]{graphics/eff/Eff2D_TPC_pion_Minus.pdf}\\
  \includegraphics[width=\linewidth,page=6]{graphics/eff/Eff2D_TPC_pion_Minus.pdf}\\
  \includegraphics[width=\linewidth,page=8]{graphics/eff/Eff2D_TPC_pion_Minus.pdf}\\
  \includegraphics[width=\linewidth,page=10]{graphics/eff/Eff2D_TPC_pion_Minus.pdf}
}%
\end{figure}
\begin{figure}[hb]\ContinuedFloat
% ~\\[32pt]
\centering
\parbox{0.495\textwidth}{
  \centering
  \includegraphics[width=\linewidth,page=11]{graphics/eff/Eff2D_TPC_pion_Minus.pdf}\\
  \includegraphics[width=\linewidth,page=13]{graphics/eff/Eff2D_TPC_pion_Minus.pdf}\\
  \includegraphics[width=\linewidth,page=15]{graphics/eff/Eff2D_TPC_pion_Minus.pdf}\\
  \includegraphics[width=\linewidth,page=17]{graphics/eff/Eff2D_TPC_pion_Minus.pdf}
}~
\parbox{0.495\textwidth}{
  \centering
  \includegraphics[width=\linewidth,page=12]{graphics/eff/Eff2D_TPC_pion_Minus.pdf}\\
  \includegraphics[width=\linewidth,page=14]{graphics/eff/Eff2D_TPC_pion_Minus.pdf}\\
  \includegraphics[width=\linewidth,page=16]{graphics/eff/Eff2D_TPC_pion_Minus.pdf}\\
  \includegraphics[width=\linewidth,page=18]{graphics/eff/Eff2D_TPC_pion_Minus.pdf}
}%
\end{figure}
%---------------------------

%---------------------------
\begin{figure}[hb]
\caption[TPC acceptance and reconstruction efficiency of $\pi^{+}$.]{TPC acceptance and reconstruction efficiency of $\pi^{+}$. Each plot represents the TPC efficiency $\epsilon_{\text{TPC}}$ ($z$-axis) as a function of true particle pseudorapidity $\eta$ ($x$-axis) and transverse momentum $p_{T}$ ($y$-axis) in single $z$-vertex bin whose range is given at the top. Red lines and arrows indicate region accepted in analyses.}\label{fig:tpcEff_pion_plus}
\centering
\parbox{0.495\textwidth}{
  \centering
  \includegraphics[width=\linewidth,page=3]{graphics/eff/Eff2D_TPC_pion_Plus.pdf}\\
  \includegraphics[width=\linewidth,page=5]{graphics/eff/Eff2D_TPC_pion_Plus.pdf}\\
  \includegraphics[width=\linewidth,page=7]{graphics/eff/Eff2D_TPC_pion_Plus.pdf}\\
  \includegraphics[width=\linewidth,page=9]{graphics/eff/Eff2D_TPC_pion_Plus.pdf}
}~
\parbox{0.495\textwidth}{
  \centering
  \includegraphics[width=\linewidth,page=4]{graphics/eff/Eff2D_TPC_pion_Plus.pdf}\\
  \includegraphics[width=\linewidth,page=6]{graphics/eff/Eff2D_TPC_pion_Plus.pdf}\\
  \includegraphics[width=\linewidth,page=8]{graphics/eff/Eff2D_TPC_pion_Plus.pdf}\\
  \includegraphics[width=\linewidth,page=10]{graphics/eff/Eff2D_TPC_pion_Plus.pdf}
}%
\end{figure}
\begin{figure}[hb]\ContinuedFloat
% ~\\[32pt]
\centering
\parbox{0.495\textwidth}{
  \centering
  \includegraphics[width=\linewidth,page=11]{graphics/eff/Eff2D_TPC_pion_Plus.pdf}\\
  \includegraphics[width=\linewidth,page=13]{graphics/eff/Eff2D_TPC_pion_Plus.pdf}\\
  \includegraphics[width=\linewidth,page=15]{graphics/eff/Eff2D_TPC_pion_Plus.pdf}\\
  \includegraphics[width=\linewidth,page=17]{graphics/eff/Eff2D_TPC_pion_Plus.pdf}
}~
\parbox{0.495\textwidth}{
  \centering
  \includegraphics[width=\linewidth,page=12]{graphics/eff/Eff2D_TPC_pion_Plus.pdf}\\
  \includegraphics[width=\linewidth,page=14]{graphics/eff/Eff2D_TPC_pion_Plus.pdf}\\
  \includegraphics[width=\linewidth,page=16]{graphics/eff/Eff2D_TPC_pion_Plus.pdf}\\
  \includegraphics[width=\linewidth,page=18]{graphics/eff/Eff2D_TPC_pion_Plus.pdf}
}%
\end{figure}
%---------------------------





%---------------------------
\begin{figure}[hb]
\caption[TPC acceptance and reconstruction efficiency of $K^{-}$.]{TPC acceptance and reconstruction efficiency of $K^{-}$. Each plot represents the TPC efficiency $\epsilon_{\text{TPC}}$ ($z$-axis) as a function of true particle pseudorapidity $\eta$ ($x$-axis) and transverse momentum $p_{T}$ ($y$-axis) in single $z$-vertex bin whose range is given at the top. Red lines and arrows indicate region accepted in analyses.}\label{fig:tpcEff_kaon_minus}
\centering
\parbox{0.495\textwidth}{
  \centering
  \includegraphics[width=\linewidth,page=3]{graphics/eff/Eff2D_TPC_kaon_Minus.pdf}\\
  \includegraphics[width=\linewidth,page=5]{graphics/eff/Eff2D_TPC_kaon_Minus.pdf}\\
  \includegraphics[width=\linewidth,page=7]{graphics/eff/Eff2D_TPC_kaon_Minus.pdf}\\
  \includegraphics[width=\linewidth,page=9]{graphics/eff/Eff2D_TPC_kaon_Minus.pdf}
}~
\parbox{0.495\textwidth}{
  \centering
  \includegraphics[width=\linewidth,page=4]{graphics/eff/Eff2D_TPC_kaon_Minus.pdf}\\
  \includegraphics[width=\linewidth,page=6]{graphics/eff/Eff2D_TPC_kaon_Minus.pdf}\\
  \includegraphics[width=\linewidth,page=8]{graphics/eff/Eff2D_TPC_kaon_Minus.pdf}\\
  \includegraphics[width=\linewidth,page=10]{graphics/eff/Eff2D_TPC_kaon_Minus.pdf}
}%
\end{figure}
\begin{figure}[hb]\ContinuedFloat
% ~\\[32pt]
\centering
\parbox{0.495\textwidth}{
  \centering
  \includegraphics[width=\linewidth,page=11]{graphics/eff/Eff2D_TPC_kaon_Minus.pdf}\\
  \includegraphics[width=\linewidth,page=13]{graphics/eff/Eff2D_TPC_kaon_Minus.pdf}\\
  \includegraphics[width=\linewidth,page=15]{graphics/eff/Eff2D_TPC_kaon_Minus.pdf}\\
  \includegraphics[width=\linewidth,page=17]{graphics/eff/Eff2D_TPC_kaon_Minus.pdf}
}~
\parbox{0.495\textwidth}{
  \centering
  \includegraphics[width=\linewidth,page=12]{graphics/eff/Eff2D_TPC_kaon_Minus.pdf}\\
  \includegraphics[width=\linewidth,page=14]{graphics/eff/Eff2D_TPC_kaon_Minus.pdf}\\
  \includegraphics[width=\linewidth,page=16]{graphics/eff/Eff2D_TPC_kaon_Minus.pdf}\\
  \includegraphics[width=\linewidth,page=18]{graphics/eff/Eff2D_TPC_kaon_Minus.pdf}
}%
\end{figure}
%---------------------------


%---------------------------
\begin{figure}[hb]
\caption[TPC acceptance and reconstruction efficiency of $K^{+}$.]{TPC acceptance and reconstruction efficiency of $K^{+}$. Each plot represents the TPC efficiency $\epsilon_{\text{TPC}}$ ($z$-axis) as a function of true particle pseudorapidity $\eta$ ($x$-axis) and transverse momentum $p_{T}$ ($y$-axis) in single $z$-vertex bin whose range is given at the top. Red lines and arrows indicate region accepted in analyses.}\label{fig:tpcEff_kaon_plus}
\centering
\parbox{0.495\textwidth}{
  \centering
  \includegraphics[width=\linewidth,page=3]{graphics/eff/Eff2D_TPC_kaon_Plus.pdf}\\
  \includegraphics[width=\linewidth,page=5]{graphics/eff/Eff2D_TPC_kaon_Plus.pdf}\\
  \includegraphics[width=\linewidth,page=7]{graphics/eff/Eff2D_TPC_kaon_Plus.pdf}\\
  \includegraphics[width=\linewidth,page=9]{graphics/eff/Eff2D_TPC_kaon_Plus.pdf}
}~
\parbox{0.495\textwidth}{
  \centering
  \includegraphics[width=\linewidth,page=4]{graphics/eff/Eff2D_TPC_kaon_Plus.pdf}\\
  \includegraphics[width=\linewidth,page=6]{graphics/eff/Eff2D_TPC_kaon_Plus.pdf}\\
  \includegraphics[width=\linewidth,page=8]{graphics/eff/Eff2D_TPC_kaon_Plus.pdf}\\
  \includegraphics[width=\linewidth,page=10]{graphics/eff/Eff2D_TPC_kaon_Plus.pdf}
}%
\end{figure}
\begin{figure}[hb]\ContinuedFloat
% ~\\[32pt]
\centering
\parbox{0.495\textwidth}{
  \centering
  \includegraphics[width=\linewidth,page=11]{graphics/eff/Eff2D_TPC_kaon_Plus.pdf}\\
  \includegraphics[width=\linewidth,page=13]{graphics/eff/Eff2D_TPC_kaon_Plus.pdf}\\
  \includegraphics[width=\linewidth,page=15]{graphics/eff/Eff2D_TPC_kaon_Plus.pdf}\\
  \includegraphics[width=\linewidth,page=17]{graphics/eff/Eff2D_TPC_kaon_Plus.pdf}
}~
\parbox{0.495\textwidth}{
  \centering
  \includegraphics[width=\linewidth,page=12]{graphics/eff/Eff2D_TPC_kaon_Plus.pdf}\\
  \includegraphics[width=\linewidth,page=14]{graphics/eff/Eff2D_TPC_kaon_Plus.pdf}\\
  \includegraphics[width=\linewidth,page=16]{graphics/eff/Eff2D_TPC_kaon_Plus.pdf}\\
  \includegraphics[width=\linewidth,page=18]{graphics/eff/Eff2D_TPC_kaon_Plus.pdf}
}%
\end{figure}
%---------------------------













%---------------------------
\begin{figure}[hb]
\caption[TPC acceptance and reconstruction efficiency of $\bar{p}$.]{TPC acceptance and reconstruction efficiency of $\bar{p}$. Each plot represents the TPC efficiency $\epsilon_{\text{TPC}}$ ($z$-axis) as a function of true particle pseudorapidity $\eta$ ($x$-axis) and transverse momentum $p_{T}$ ($y$-axis) in single $z$-vertex bin whose range is given at the top. Red lines and arrows indicate region accepted in analyses.}\label{fig:tpcEff_proton_minus}
\centering
\parbox{0.495\textwidth}{
  \centering
  \includegraphics[width=\linewidth,page=3]{graphics/eff/Eff2D_TPC_proton_Minus.pdf}\\
  \includegraphics[width=\linewidth,page=5]{graphics/eff/Eff2D_TPC_proton_Minus.pdf}\\
  \includegraphics[width=\linewidth,page=7]{graphics/eff/Eff2D_TPC_proton_Minus.pdf}\\
  \includegraphics[width=\linewidth,page=9]{graphics/eff/Eff2D_TPC_proton_Minus.pdf}
}~
\parbox{0.495\textwidth}{
  \centering
  \includegraphics[width=\linewidth,page=4]{graphics/eff/Eff2D_TPC_proton_Minus.pdf}\\
  \includegraphics[width=\linewidth,page=6]{graphics/eff/Eff2D_TPC_proton_Minus.pdf}\\
  \includegraphics[width=\linewidth,page=8]{graphics/eff/Eff2D_TPC_proton_Minus.pdf}\\
  \includegraphics[width=\linewidth,page=10]{graphics/eff/Eff2D_TPC_proton_Minus.pdf}
}%
\end{figure}
\begin{figure}[hb]\ContinuedFloat
% ~\\[32pt]
\centering
\parbox{0.495\textwidth}{
  \centering
  \includegraphics[width=\linewidth,page=11]{graphics/eff/Eff2D_TPC_proton_Minus.pdf}\\
  \includegraphics[width=\linewidth,page=13]{graphics/eff/Eff2D_TPC_proton_Minus.pdf}\\
  \includegraphics[width=\linewidth,page=15]{graphics/eff/Eff2D_TPC_proton_Minus.pdf}\\
  \includegraphics[width=\linewidth,page=17]{graphics/eff/Eff2D_TPC_proton_Minus.pdf}
}~
\parbox{0.495\textwidth}{
  \centering
  \includegraphics[width=\linewidth,page=12]{graphics/eff/Eff2D_TPC_proton_Minus.pdf}\\
  \includegraphics[width=\linewidth,page=14]{graphics/eff/Eff2D_TPC_proton_Minus.pdf}\\
  \includegraphics[width=\linewidth,page=16]{graphics/eff/Eff2D_TPC_proton_Minus.pdf}\\
  \includegraphics[width=\linewidth,page=18]{graphics/eff/Eff2D_TPC_proton_Minus.pdf}
}%
\end{figure}
%---------------------------


%---------------------------
\begin{figure}[hb]
\caption[TPC acceptance and reconstruction efficiency of $p$.]{TPC acceptance and reconstruction efficiency of $p$. Each plot represents the TPC efficiency $\epsilon_{\text{TPC}}$ ($z$-axis) as a function of true particle pseudorapidity $\eta$ ($x$-axis) and transverse momentum $p_{T}$ ($y$-axis) in single $z$-vertex bin whose range is given at the top. Red lines and arrows indicate region accepted in analyses.}\label{fig:tpcEff_proton_plus}
\centering
\parbox{0.495\textwidth}{
  \centering
  \includegraphics[width=\linewidth,page=3]{graphics/eff/Eff2D_TPC_proton_Plus.pdf}\\
  \includegraphics[width=\linewidth,page=5]{graphics/eff/Eff2D_TPC_proton_Plus.pdf}\\
  \includegraphics[width=\linewidth,page=7]{graphics/eff/Eff2D_TPC_proton_Plus.pdf}\\
  \includegraphics[width=\linewidth,page=9]{graphics/eff/Eff2D_TPC_proton_Plus.pdf}
}~
\parbox{0.495\textwidth}{
  \centering
  \includegraphics[width=\linewidth,page=4]{graphics/eff/Eff2D_TPC_proton_Plus.pdf}\\
  \includegraphics[width=\linewidth,page=6]{graphics/eff/Eff2D_TPC_proton_Plus.pdf}\\
  \includegraphics[width=\linewidth,page=8]{graphics/eff/Eff2D_TPC_proton_Plus.pdf}\\
  \includegraphics[width=\linewidth,page=10]{graphics/eff/Eff2D_TPC_proton_Plus.pdf}
}%
\end{figure}
\begin{figure}[hb]\ContinuedFloat
% ~\\[32pt]
\centering
\parbox{0.495\textwidth}{
  \centering
  \includegraphics[width=\linewidth,page=11]{graphics/eff/Eff2D_TPC_proton_Plus.pdf}\\
  \includegraphics[width=\linewidth,page=13]{graphics/eff/Eff2D_TPC_proton_Plus.pdf}\\
  \includegraphics[width=\linewidth,page=15]{graphics/eff/Eff2D_TPC_proton_Plus.pdf}\\
  \includegraphics[width=\linewidth,page=17]{graphics/eff/Eff2D_TPC_proton_Plus.pdf}
}~
\parbox{0.495\textwidth}{
  \centering
  \includegraphics[width=\linewidth,page=12]{graphics/eff/Eff2D_TPC_proton_Plus.pdf}\\
  \includegraphics[width=\linewidth,page=14]{graphics/eff/Eff2D_TPC_proton_Plus.pdf}\\
  \includegraphics[width=\linewidth,page=16]{graphics/eff/Eff2D_TPC_proton_Plus.pdf}\\
  \includegraphics[width=\linewidth,page=18]{graphics/eff/Eff2D_TPC_proton_Plus.pdf}
}%
\end{figure}
%---------------------------
% %% =====  TOF EFFICIENCY ====
%%===========================================================%%
%%                                                           %%
%%                    TOF EFFICIENCY APPENDIX                %%
%%                                                           %%
%%===========================================================%%

\chapter{TOF hit reconstruction and matching efficiency}\label{appendix:tofEff}


\begin{table}[h]%
	\centering%
	\begin{tabular}{c|c|l}%\hline
		\textbf{Tray No.}&	\textbf{Module No.} & 	\textbf{RHIC fills}\\ \hline
		$8$	& $1-4$, $29-32$ & $18686-18953$\\ \hline
		$23$	& $1-4$, $29-32$ & $18686-18953$\\ \hline
		$38$	& $1-32$ & $18686-18953$\\ \hline
		$39$	& $13$ & $18686-18715$, $18719-18795$, $18797-18827$, $18829-18854$, $18856-18878$,\\
		& &$18883-18892$, $18895-18901$, $18904-18924$, $18926-18953$\\ \hline
		$41$	& $1-32$ & $18686-18953$\\ \hline
		$45$	& $1-32$ & $18877-18909$\\ \hline
		$46$	& $1-32$ & $18853-18909$\\ \hline
		$93$	& $1-4$, $29-32$ & $18686-18953$\\ \hline
		$102$	& $1-32$ & $18686-18953$\\ \hline
		$108$	& $1-4$, $29-32$ & $18686-18953$\\ \hline
	\end{tabular}%
	\caption[Dead TOF modules masked in the MC.]{Dead TOF modules masked in the MC. Table was filled with modules which were not matched with TPC tracks in the data.}\label{tab:tofDeadModules}
\end{table}

%---------------------------
\begin{figure}[hb]
\caption[TOF acceptance, reconstruction and matching efficiency of $\pi^{-}$.]{TOF acceptance, reconstruction and matching efficiency of $\pi^{-}$. Each plot represents the TOF efficiency $\epsilon_{\text{TOF}}$ ($z$-axis) as a function of true particle pseudorapidity $\eta$ ($x$-axis) and transverse momentum $p_{T}$ ($y$-axis) in single $z$-vertex bin whose range is given at the top. Red lines and arrows indicate region accepted in analyses.}\label{fig:tofEff_pion_minus}
\centering
\parbox{0.495\textwidth}{
  \centering
  \includegraphics[width=\linewidth,page=3]{graphics/eff/Eff2D_TOF_pion_Minus.pdf}\\
  \includegraphics[width=\linewidth,page=5]{graphics/eff/Eff2D_TOF_pion_Minus.pdf}\\
  \includegraphics[width=\linewidth,page=7]{graphics/eff/Eff2D_TOF_pion_Minus.pdf}\\
  \includegraphics[width=\linewidth,page=9]{graphics/eff/Eff2D_TOF_pion_Minus.pdf}
}~
\parbox{0.495\textwidth}{
  \centering
  \includegraphics[width=\linewidth,page=4]{graphics/eff/Eff2D_TOF_pion_Minus.pdf}\\
  \includegraphics[width=\linewidth,page=6]{graphics/eff/Eff2D_TOF_pion_Minus.pdf}\\
  \includegraphics[width=\linewidth,page=8]{graphics/eff/Eff2D_TOF_pion_Minus.pdf}\\
  \includegraphics[width=\linewidth,page=10]{graphics/eff/Eff2D_TOF_pion_Minus.pdf}
}%
\end{figure}
\begin{figure}[hb]\ContinuedFloat
% ~\\[32pt]
\centering
\parbox{0.495\textwidth}{
  \centering
  \includegraphics[width=\linewidth,page=11]{graphics/eff/Eff2D_TOF_pion_Minus.pdf}\\
  \includegraphics[width=\linewidth,page=13]{graphics/eff/Eff2D_TOF_pion_Minus.pdf}\\
  \includegraphics[width=\linewidth,page=15]{graphics/eff/Eff2D_TOF_pion_Minus.pdf}\\
  \includegraphics[width=\linewidth,page=17]{graphics/eff/Eff2D_TOF_pion_Minus.pdf}
}~
\parbox{0.495\textwidth}{
  \centering
  \includegraphics[width=\linewidth,page=12]{graphics/eff/Eff2D_TOF_pion_Minus.pdf}\\
  \includegraphics[width=\linewidth,page=14]{graphics/eff/Eff2D_TOF_pion_Minus.pdf}\\
  \includegraphics[width=\linewidth,page=16]{graphics/eff/Eff2D_TOF_pion_Minus.pdf}\\
  \includegraphics[width=\linewidth,page=18]{graphics/eff/Eff2D_TOF_pion_Minus.pdf}
}%
\end{figure}
%---------------------------

%---------------------------
\begin{figure}[hb]
\caption[TOF acceptance, reconstruction and matching efficiency of $\pi^{+}$.]{TOF acceptance, reconstruction and matching efficiency of $\pi^{+}$. Each plot represents the TOF efficiency $\epsilon_{\text{TOF}}$ ($z$-axis) as a function of true particle pseudorapidity $\eta$ ($x$-axis) and transverse momentum $p_{T}$ ($y$-axis) in single $z$-vertex bin whose range is given at the top. Red lines and arrows indicate region accepted in analyses.}\label{fig:tofEff_pion_plus}
\centering
\parbox{0.495\textwidth}{
  \centering
  \includegraphics[width=\linewidth,page=3]{graphics/eff/Eff2D_TOF_pion_Plus.pdf}\\
  \includegraphics[width=\linewidth,page=5]{graphics/eff/Eff2D_TOF_pion_Plus.pdf}\\
  \includegraphics[width=\linewidth,page=7]{graphics/eff/Eff2D_TOF_pion_Plus.pdf}\\
  \includegraphics[width=\linewidth,page=9]{graphics/eff/Eff2D_TOF_pion_Plus.pdf}
}~
\parbox{0.495\textwidth}{
  \centering
  \includegraphics[width=\linewidth,page=4]{graphics/eff/Eff2D_TOF_pion_Plus.pdf}\\
  \includegraphics[width=\linewidth,page=6]{graphics/eff/Eff2D_TOF_pion_Plus.pdf}\\
  \includegraphics[width=\linewidth,page=8]{graphics/eff/Eff2D_TOF_pion_Plus.pdf}\\
  \includegraphics[width=\linewidth,page=10]{graphics/eff/Eff2D_TOF_pion_Plus.pdf}
}%
\end{figure}
\begin{figure}[hb]\ContinuedFloat
% ~\\[32pt]
\centering
\parbox{0.495\textwidth}{
  \centering
  \includegraphics[width=\linewidth,page=11]{graphics/eff/Eff2D_TOF_pion_Plus.pdf}\\
  \includegraphics[width=\linewidth,page=13]{graphics/eff/Eff2D_TOF_pion_Plus.pdf}\\
  \includegraphics[width=\linewidth,page=15]{graphics/eff/Eff2D_TOF_pion_Plus.pdf}\\
  \includegraphics[width=\linewidth,page=17]{graphics/eff/Eff2D_TOF_pion_Plus.pdf}
}~
\parbox{0.495\textwidth}{
  \centering
  \includegraphics[width=\linewidth,page=12]{graphics/eff/Eff2D_TOF_pion_Plus.pdf}\\
  \includegraphics[width=\linewidth,page=14]{graphics/eff/Eff2D_TOF_pion_Plus.pdf}\\
  \includegraphics[width=\linewidth,page=16]{graphics/eff/Eff2D_TOF_pion_Plus.pdf}\\
  \includegraphics[width=\linewidth,page=18]{graphics/eff/Eff2D_TOF_pion_Plus.pdf}
}%
\end{figure}
%---------------------------





%---------------------------
\begin{figure}[hb]
\caption[TOF acceptance, reconstruction and matching efficiency of $K^{-}$.]{TOF acceptance, reconstruction and matching efficiency of $K^{-}$. Each plot represents the TOF efficiency $\epsilon_{\text{TOF}}$ ($z$-axis) as a function of true particle pseudorapidity $\eta$ ($x$-axis) and transverse momentum $p_{T}$ ($y$-axis) in single $z$-vertex bin whose range is given at the top. Red lines and arrows indicate region accepted in analyses.}\label{fig:tofEff_kaon_minus}
\centering
\parbox{0.495\textwidth}{
  \centering
  \includegraphics[width=\linewidth,page=3]{graphics/eff/Eff2D_TOF_kaon_Minus.pdf}\\
  \includegraphics[width=\linewidth,page=5]{graphics/eff/Eff2D_TOF_kaon_Minus.pdf}\\
  \includegraphics[width=\linewidth,page=7]{graphics/eff/Eff2D_TOF_kaon_Minus.pdf}\\
  \includegraphics[width=\linewidth,page=9]{graphics/eff/Eff2D_TOF_kaon_Minus.pdf}
}~
\parbox{0.495\textwidth}{
  \centering
  \includegraphics[width=\linewidth,page=4]{graphics/eff/Eff2D_TOF_kaon_Minus.pdf}\\
  \includegraphics[width=\linewidth,page=6]{graphics/eff/Eff2D_TOF_kaon_Minus.pdf}\\
  \includegraphics[width=\linewidth,page=8]{graphics/eff/Eff2D_TOF_kaon_Minus.pdf}\\
  \includegraphics[width=\linewidth,page=10]{graphics/eff/Eff2D_TOF_kaon_Minus.pdf}
}%
\end{figure}
\begin{figure}[hb]\ContinuedFloat
% ~\\[32pt]
\centering
\parbox{0.495\textwidth}{
  \centering
  \includegraphics[width=\linewidth,page=11]{graphics/eff/Eff2D_TOF_kaon_Minus.pdf}\\
  \includegraphics[width=\linewidth,page=13]{graphics/eff/Eff2D_TOF_kaon_Minus.pdf}\\
  \includegraphics[width=\linewidth,page=15]{graphics/eff/Eff2D_TOF_kaon_Minus.pdf}\\
  \includegraphics[width=\linewidth,page=17]{graphics/eff/Eff2D_TOF_kaon_Minus.pdf}
}~
\parbox{0.495\textwidth}{
  \centering
  \includegraphics[width=\linewidth,page=12]{graphics/eff/Eff2D_TOF_kaon_Minus.pdf}\\
  \includegraphics[width=\linewidth,page=14]{graphics/eff/Eff2D_TOF_kaon_Minus.pdf}\\
  \includegraphics[width=\linewidth,page=16]{graphics/eff/Eff2D_TOF_kaon_Minus.pdf}\\
  \includegraphics[width=\linewidth,page=18]{graphics/eff/Eff2D_TOF_kaon_Minus.pdf}
}%
\end{figure}
%---------------------------


%---------------------------
\begin{figure}[hb]
\caption[TOF acceptance, reconstruction and matching efficiency of $K^{+}$.]{TOF acceptance, reconstruction and matching efficiency of $K^{+}$. Each plot represents the TOF efficiency $\epsilon_{\text{TOF}}$ ($z$-axis) as a function of true particle pseudorapidity $\eta$ ($x$-axis) and transverse momentum $p_{T}$ ($y$-axis) in single $z$-vertex bin whose range is given at the top. Red lines and arrows indicate region accepted in analyses.}\label{fig:tofEff_kaon_plus}
\centering
\parbox{0.495\textwidth}{
  \centering
  \includegraphics[width=\linewidth,page=3]{graphics/eff/Eff2D_TOF_kaon_Plus.pdf}\\
  \includegraphics[width=\linewidth,page=5]{graphics/eff/Eff2D_TOF_kaon_Plus.pdf}\\
  \includegraphics[width=\linewidth,page=7]{graphics/eff/Eff2D_TOF_kaon_Plus.pdf}\\
  \includegraphics[width=\linewidth,page=9]{graphics/eff/Eff2D_TOF_kaon_Plus.pdf}
}~
\parbox{0.495\textwidth}{
  \centering
  \includegraphics[width=\linewidth,page=4]{graphics/eff/Eff2D_TOF_kaon_Plus.pdf}\\
  \includegraphics[width=\linewidth,page=6]{graphics/eff/Eff2D_TOF_kaon_Plus.pdf}\\
  \includegraphics[width=\linewidth,page=8]{graphics/eff/Eff2D_TOF_kaon_Plus.pdf}\\
  \includegraphics[width=\linewidth,page=10]{graphics/eff/Eff2D_TOF_kaon_Plus.pdf}
}%
\end{figure}
\begin{figure}[hb]\ContinuedFloat
% ~\\[32pt]
\centering
\parbox{0.495\textwidth}{
  \centering
  \includegraphics[width=\linewidth,page=11]{graphics/eff/Eff2D_TOF_kaon_Plus.pdf}\\
  \includegraphics[width=\linewidth,page=13]{graphics/eff/Eff2D_TOF_kaon_Plus.pdf}\\
  \includegraphics[width=\linewidth,page=15]{graphics/eff/Eff2D_TOF_kaon_Plus.pdf}\\
  \includegraphics[width=\linewidth,page=17]{graphics/eff/Eff2D_TOF_kaon_Plus.pdf}
}~
\parbox{0.495\textwidth}{
  \centering
  \includegraphics[width=\linewidth,page=12]{graphics/eff/Eff2D_TOF_kaon_Plus.pdf}\\
  \includegraphics[width=\linewidth,page=14]{graphics/eff/Eff2D_TOF_kaon_Plus.pdf}\\
  \includegraphics[width=\linewidth,page=16]{graphics/eff/Eff2D_TOF_kaon_Plus.pdf}\\
  \includegraphics[width=\linewidth,page=18]{graphics/eff/Eff2D_TOF_kaon_Plus.pdf}
}%
\end{figure}
%---------------------------













%---------------------------
\begin{figure}[hb]
\caption[TOF acceptance, reconstruction and matching efficiency of $\bar{p}$.]{TOF acceptance, reconstruction and matching efficiency of $\bar{p}$. Each plot represents the TOF efficiency $\epsilon_{\text{TOF}}$ ($z$-axis) as a function of true particle pseudorapidity $\eta$ ($x$-axis) and transverse momentum $p_{T}$ ($y$-axis) in single $z$-vertex bin whose range is given at the top. Red lines and arrows indicate region accepted in analyses.}\label{fig:tofEff_proton_minus}
\centering
\parbox{0.495\textwidth}{
  \centering
  \includegraphics[width=\linewidth,page=3]{graphics/eff/Eff2D_TOF_proton_Minus.pdf}\\
  \includegraphics[width=\linewidth,page=5]{graphics/eff/Eff2D_TOF_proton_Minus.pdf}\\
  \includegraphics[width=\linewidth,page=7]{graphics/eff/Eff2D_TOF_proton_Minus.pdf}\\
  \includegraphics[width=\linewidth,page=9]{graphics/eff/Eff2D_TOF_proton_Minus.pdf}
}~
\parbox{0.495\textwidth}{
  \centering
  \includegraphics[width=\linewidth,page=4]{graphics/eff/Eff2D_TOF_proton_Minus.pdf}\\
  \includegraphics[width=\linewidth,page=6]{graphics/eff/Eff2D_TOF_proton_Minus.pdf}\\
  \includegraphics[width=\linewidth,page=8]{graphics/eff/Eff2D_TOF_proton_Minus.pdf}\\
  \includegraphics[width=\linewidth,page=10]{graphics/eff/Eff2D_TOF_proton_Minus.pdf}
}%
\end{figure}
\begin{figure}[hb]\ContinuedFloat
% ~\\[32pt]
\centering
\parbox{0.495\textwidth}{
  \centering
  \includegraphics[width=\linewidth,page=11]{graphics/eff/Eff2D_TOF_proton_Minus.pdf}\\
  \includegraphics[width=\linewidth,page=13]{graphics/eff/Eff2D_TOF_proton_Minus.pdf}\\
  \includegraphics[width=\linewidth,page=15]{graphics/eff/Eff2D_TOF_proton_Minus.pdf}\\
  \includegraphics[width=\linewidth,page=17]{graphics/eff/Eff2D_TOF_proton_Minus.pdf}
}~
\parbox{0.495\textwidth}{
  \centering
  \includegraphics[width=\linewidth,page=12]{graphics/eff/Eff2D_TOF_proton_Minus.pdf}\\
  \includegraphics[width=\linewidth,page=14]{graphics/eff/Eff2D_TOF_proton_Minus.pdf}\\
  \includegraphics[width=\linewidth,page=16]{graphics/eff/Eff2D_TOF_proton_Minus.pdf}\\
  \includegraphics[width=\linewidth,page=18]{graphics/eff/Eff2D_TOF_proton_Minus.pdf}
}%
\end{figure}
%---------------------------


%---------------------------
\begin{figure}[hb]
\caption[TOF acceptance, reconstruction and matching efficiency of $p$.]{TOF acceptance, reconstruction and matching efficiency of $p$. Each plot represents the TOF efficiency $\epsilon_{\text{TOF}}$ ($z$-axis) as a function of true particle pseudorapidity $\eta$ ($x$-axis) and transverse momentum $p_{T}$ ($y$-axis) in single $z$-vertex bin whose range is given at the top. Red lines and arrows indicate region accepted in analyses.}\label{fig:tofEff_proton_plus}
\centering
\parbox{0.495\textwidth}{
  \centering
  \includegraphics[width=\linewidth,page=3]{graphics/eff/Eff2D_TOF_proton_Plus.pdf}\\
  \includegraphics[width=\linewidth,page=5]{graphics/eff/Eff2D_TOF_proton_Plus.pdf}\\
  \includegraphics[width=\linewidth,page=7]{graphics/eff/Eff2D_TOF_proton_Plus.pdf}\\
  \includegraphics[width=\linewidth,page=9]{graphics/eff/Eff2D_TOF_proton_Plus.pdf}
}~
\parbox{0.495\textwidth}{
  \centering
  \includegraphics[width=\linewidth,page=4]{graphics/eff/Eff2D_TOF_proton_Plus.pdf}\\
  \includegraphics[width=\linewidth,page=6]{graphics/eff/Eff2D_TOF_proton_Plus.pdf}\\
  \includegraphics[width=\linewidth,page=8]{graphics/eff/Eff2D_TOF_proton_Plus.pdf}\\
  \includegraphics[width=\linewidth,page=10]{graphics/eff/Eff2D_TOF_proton_Plus.pdf}
}%
\end{figure}
\begin{figure}[hb]\ContinuedFloat
% ~\\[32pt]
\centering
\parbox{0.495\textwidth}{
  \centering
  \includegraphics[width=\linewidth,page=11]{graphics/eff/Eff2D_TOF_proton_Plus.pdf}\\
  \includegraphics[width=\linewidth,page=13]{graphics/eff/Eff2D_TOF_proton_Plus.pdf}\\
  \includegraphics[width=\linewidth,page=15]{graphics/eff/Eff2D_TOF_proton_Plus.pdf}\\
  \includegraphics[width=\linewidth,page=17]{graphics/eff/Eff2D_TOF_proton_Plus.pdf}
}~
\parbox{0.495\textwidth}{
  \centering
  \includegraphics[width=\linewidth,page=12]{graphics/eff/Eff2D_TOF_proton_Plus.pdf}\\
  \includegraphics[width=\linewidth,page=14]{graphics/eff/Eff2D_TOF_proton_Plus.pdf}\\
  \includegraphics[width=\linewidth,page=16]{graphics/eff/Eff2D_TOF_proton_Plus.pdf}\\
  \includegraphics[width=\linewidth,page=18]{graphics/eff/Eff2D_TOF_proton_Plus.pdf}
}%
\end{figure}
%---------------------------

% %% =====  ENERGY LOSS CORRECTION ====
%%===========================================================%%
%%                                                           %%
%%              ENERGY LOSS CORRECTION APPENDIX            	%%
%%                                                           %%
%%===========================================================%%

\chapter{Energy Loss Correction}\label{appendix:energyLoss}
\begin{figure}[H]
\caption[Energy loss correction for $\pi^-$ as a function of reconstructed transverse momentum $p_T^{meas}$.]{Energy loss correction $p_T^{meas}-p_T^{true}$ for $\pi^-$ as a function of reconstructed transverse momentum $p_T^{meas}$ $\left(|\eta|<0.7\right)$ in single $z$-vertex bin whose range is given on each plot.}\label{fig:energyLossPrimaryPi_minus}
\centering
\parbox{0.329\textwidth}{
  \centering
  \includegraphics[width=\linewidth,page=3]{graphics/energyLoss/energyLoss3D_OnePrtAlso.pdf}\\
  \includegraphics[width=\linewidth,page=6]{graphics/energyLoss/energyLoss3D_OnePrtAlso.pdf}\\
  \includegraphics[width=\linewidth,page=9]{graphics/energyLoss/energyLoss3D_OnePrtAlso.pdf}\\
  \includegraphics[width=\linewidth,page=12]{graphics/energyLoss/energyLoss3D_OnePrtAlso.pdf}\\
  \includegraphics[width=\linewidth,page=15]{graphics/energyLoss/energyLoss3D_OnePrtAlso.pdf}\\
}~
\parbox{0.329\textwidth}{
  \centering
  \includegraphics[width=\linewidth,page=4]{graphics/energyLoss/energyLoss3D_OnePrtAlso.pdf}\\
  \includegraphics[width=\linewidth,page=7]{graphics/energyLoss/energyLoss3D_OnePrtAlso.pdf}\\
  \includegraphics[width=\linewidth,page=10]{graphics/energyLoss/energyLoss3D_OnePrtAlso.pdf}\\
  \includegraphics[width=\linewidth,page=13]{graphics/energyLoss/energyLoss3D_OnePrtAlso.pdf}\\
  \includegraphics[width=\linewidth,page=16]{graphics/energyLoss/energyLoss3D_OnePrtAlso.pdf}\\
}%
\parbox{0.329\textwidth}{
  \centering
  \includegraphics[width=\linewidth,page=5]{graphics/energyLoss/energyLoss3D_OnePrtAlso.pdf}\\
  \includegraphics[width=\linewidth,page=8]{graphics/energyLoss/energyLoss3D_OnePrtAlso.pdf}\\
  \includegraphics[width=\linewidth,page=11]{graphics/energyLoss/energyLoss3D_OnePrtAlso.pdf}\\
  \includegraphics[width=\linewidth,page=14]{graphics/energyLoss/energyLoss3D_OnePrtAlso.pdf}\\
  \includegraphics[width=\linewidth,page=17]{graphics/energyLoss/energyLoss3D_OnePrtAlso.pdf}\\
}%
\end{figure}
\vspace{-3.5em}
\begin{figure}[H]\ContinuedFloat
% ~\\[32pt]
\parbox{0.329\textwidth}{
  \includegraphics[width=\linewidth,page=18]{graphics/energyLoss/energyLoss3D_OnePrtAlso.pdf}\\
}~
\end{figure}
%%%pi+
\begin{figure}[H]
\caption[Energy loss correction for $\pi^+$ as a function of reconstructed transverse momentum $p_T^{meas}$.]{Energy loss correction $p_T^{meas}-p_T^{true}$ for $\pi^+$ as a function of reconstructed transverse momentum $p_T^{meas}$ $\left(|\eta|<0.7\right)$ in single $z$-vertex bin whose range is given on each plot.}\label{fig:energyLossPrimaryPi_plus}
\parbox{0.329\textwidth}{
  \includegraphics[width=\linewidth,page=63]{graphics/energyLoss/energyLoss3D_OnePrtAlso.pdf}\\
  \includegraphics[width=\linewidth,page=66]{graphics/energyLoss/energyLoss3D_OnePrtAlso.pdf}\\
  \includegraphics[width=\linewidth,page=69]{graphics/energyLoss/energyLoss3D_OnePrtAlso.pdf}\\
  \includegraphics[width=\linewidth,page=72]{graphics/energyLoss/energyLoss3D_OnePrtAlso.pdf}\\
  \includegraphics[width=\linewidth,page=75]{graphics/energyLoss/energyLoss3D_OnePrtAlso.pdf}\\
}~
\parbox{0.329\textwidth}{
  \includegraphics[width=\linewidth,page=64]{graphics/energyLoss/energyLoss3D_OnePrtAlso.pdf}\\
  \includegraphics[width=\linewidth,page=67]{graphics/energyLoss/energyLoss3D_OnePrtAlso.pdf}\\
  \includegraphics[width=\linewidth,page=70]{graphics/energyLoss/energyLoss3D_OnePrtAlso.pdf}\\
  \includegraphics[width=\linewidth,page=73]{graphics/energyLoss/energyLoss3D_OnePrtAlso.pdf}\\
  \includegraphics[width=\linewidth,page=76]{graphics/energyLoss/energyLoss3D_OnePrtAlso.pdf}\\
}%
\parbox{0.329\textwidth}{
  \includegraphics[width=\linewidth,page=65]{graphics/energyLoss/energyLoss3D_OnePrtAlso.pdf}\\
  \includegraphics[width=\linewidth,page=68]{graphics/energyLoss/energyLoss3D_OnePrtAlso.pdf}\\
  \includegraphics[width=\linewidth,page=71]{graphics/energyLoss/energyLoss3D_OnePrtAlso.pdf}\\
  \includegraphics[width=\linewidth,page=74]{graphics/energyLoss/energyLoss3D_OnePrtAlso.pdf}\\
  \includegraphics[width=\linewidth,page=77]{graphics/energyLoss/energyLoss3D_OnePrtAlso.pdf}\\
}%
\end{figure}

\begin{figure}[H]\ContinuedFloat
% ~\\[32pt]
\vspace{-3.5em}
\parbox{0.329\textwidth}{
  \includegraphics[width=\linewidth,page=78]{graphics/energyLoss/energyLoss3D_OnePrtAlso.pdf}\\
  \vspace{-4em}
}~
\end{figure}
%%%K+
\begin{figure}[H]
\caption[Energy loss correction for $K^+$ as a function of reconstructed transverse momentum $p_T^{meas}$.]{Energy loss correction $p_T^{meas}-p_T^{true}$ for $K^+$ as a function of reconstructed transverse momentum $p_T^{meas}$ $\left(|\eta|<0.7\right)$ in single $z$-vertex bin whose range is given on each plot.}\label{fig:energyLossPrimaryK_plus}
\parbox{0.329\textwidth}{
  \includegraphics[width=\linewidth,page=83]{graphics/energyLoss/energyLoss3D_OnePrtAlso.pdf}\\
  \includegraphics[width=\linewidth,page=86]{graphics/energyLoss/energyLoss3D_OnePrtAlso.pdf}\\
  \includegraphics[width=\linewidth,page=89]{graphics/energyLoss/energyLoss3D_OnePrtAlso.pdf}\\
  \includegraphics[width=\linewidth,page=92]{graphics/energyLoss/energyLoss3D_OnePrtAlso.pdf}\\
  \includegraphics[width=\linewidth,page=95]{graphics/energyLoss/energyLoss3D_OnePrtAlso.pdf}\\
}~
\parbox{0.329\textwidth}{
  \includegraphics[width=\linewidth,page=84]{graphics/energyLoss/energyLoss3D_OnePrtAlso.pdf}\\
  \includegraphics[width=\linewidth,page=87]{graphics/energyLoss/energyLoss3D_OnePrtAlso.pdf}\\
  \includegraphics[width=\linewidth,page=90]{graphics/energyLoss/energyLoss3D_OnePrtAlso.pdf}\\
  \includegraphics[width=\linewidth,page=93]{graphics/energyLoss/energyLoss3D_OnePrtAlso.pdf}\\
  \includegraphics[width=\linewidth,page=96]{graphics/energyLoss/energyLoss3D_OnePrtAlso.pdf}\\
}%
\parbox{0.329\textwidth}{
  \includegraphics[width=\linewidth,page=85]{graphics/energyLoss/energyLoss3D_OnePrtAlso.pdf}\\
  \includegraphics[width=\linewidth,page=88]{graphics/energyLoss/energyLoss3D_OnePrtAlso.pdf}\\
  \includegraphics[width=\linewidth,page=91]{graphics/energyLoss/energyLoss3D_OnePrtAlso.pdf}\\
  \includegraphics[width=\linewidth,page=94]{graphics/energyLoss/energyLoss3D_OnePrtAlso.pdf}\\
  \includegraphics[width=\linewidth,page=97]{graphics/energyLoss/energyLoss3D_OnePrtAlso.pdf}\\
}%
\end{figure}

\begin{figure}[H]\ContinuedFloat
% ~\\[32pt]
\vspace{-3.5em}
\parbox{0.329\textwidth}{
  \includegraphics[width=\linewidth,page=98]{graphics/energyLoss/energyLoss3D_OnePrtAlso.pdf}\\
  \vspace{-4em}
}~
\end{figure}
%%%pbar
\begin{figure}[H]
\caption[Energy loss correction for $\bar{p}$ as a function of reconstructed transverse momentum $p_T^{meas}$.]{Energy loss correction $p_T^{meas}-p_T^{true}$ for $\bar{p}$ as a function of reconstructed transverse momentum $p_T^{meas}$ $\left(|\eta|<0.7\right)$ in single $z$-vertex bin whose range is given on each plot.}\label{fig:energyLossPrimaryP_bar}
\parbox{0.329\textwidth}{
  \includegraphics[width=\linewidth,page=43]{graphics/energyLoss/energyLoss3D_OnePrtAlso.pdf}\\
  \includegraphics[width=\linewidth,page=46]{graphics/energyLoss/energyLoss3D_OnePrtAlso.pdf}\\
  \includegraphics[width=\linewidth,page=49]{graphics/energyLoss/energyLoss3D_OnePrtAlso.pdf}\\
  \includegraphics[width=\linewidth,page=52]{graphics/energyLoss/energyLoss3D_OnePrtAlso.pdf}\\
  \includegraphics[width=\linewidth,page=55]{graphics/energyLoss/energyLoss3D_OnePrtAlso.pdf}\\
}~
\parbox{0.329\textwidth}{
  \includegraphics[width=\linewidth,page=44]{graphics/energyLoss/energyLoss3D_OnePrtAlso.pdf}\\
  \includegraphics[width=\linewidth,page=47]{graphics/energyLoss/energyLoss3D_OnePrtAlso.pdf}\\
  \includegraphics[width=\linewidth,page=50]{graphics/energyLoss/energyLoss3D_OnePrtAlso.pdf}\\
  \includegraphics[width=\linewidth,page=53]{graphics/energyLoss/energyLoss3D_OnePrtAlso.pdf}\\
  \includegraphics[width=\linewidth,page=56]{graphics/energyLoss/energyLoss3D_OnePrtAlso.pdf}\\
}%
\parbox{0.329\textwidth}{
  \includegraphics[width=\linewidth,page=45]{graphics/energyLoss/energyLoss3D_OnePrtAlso.pdf}\\
  \includegraphics[width=\linewidth,page=48]{graphics/energyLoss/energyLoss3D_OnePrtAlso.pdf}\\
  \includegraphics[width=\linewidth,page=51]{graphics/energyLoss/energyLoss3D_OnePrtAlso.pdf}\\
  \includegraphics[width=\linewidth,page=54]{graphics/energyLoss/energyLoss3D_OnePrtAlso.pdf}\\
  \includegraphics[width=\linewidth,page=57]{graphics/energyLoss/energyLoss3D_OnePrtAlso.pdf}\\
}%
\end{figure}

\begin{figure}[H]\ContinuedFloat
% ~\\[32pt]
\vspace{-3.5em}
\parbox{0.329\textwidth}{
  \includegraphics[width=\linewidth,page=58]{graphics/energyLoss/energyLoss3D_OnePrtAlso.pdf}\\
  \vspace{-4em}
}~
\end{figure}
%%%p
\begin{figure}[H]
\caption[Energy loss correction for $p$ as a function of reconstructed transverse momentum $p_T^{meas}$.]{Energy loss correction $p_T^{meas}-p_T^{true}$ for $p$ as a function of reconstructed transverse momentum $p_T^{meas}$ $\left(|\eta|<0.7\right)$ in single $z$-vertex bin whose range is given on each plot.}\label{fig:energyLossPrimaryP}
\parbox{0.329\textwidth}{
  \includegraphics[width=\linewidth,page=103]{graphics/energyLoss/energyLoss3D_OnePrtAlso.pdf}\\
  \includegraphics[width=\linewidth,page=106]{graphics/energyLoss/energyLoss3D_OnePrtAlso.pdf}\\
  \includegraphics[width=\linewidth,page=109]{graphics/energyLoss/energyLoss3D_OnePrtAlso.pdf}\\
  \includegraphics[width=\linewidth,page=112]{graphics/energyLoss/energyLoss3D_OnePrtAlso.pdf}\\
  \includegraphics[width=\linewidth,page=115]{graphics/energyLoss/energyLoss3D_OnePrtAlso.pdf}\\
}~
\parbox{0.329\textwidth}{
  \includegraphics[width=\linewidth,page=104]{graphics/energyLoss/energyLoss3D_OnePrtAlso.pdf}\\
  \includegraphics[width=\linewidth,page=107]{graphics/energyLoss/energyLoss3D_OnePrtAlso.pdf}\\
  \includegraphics[width=\linewidth,page=110]{graphics/energyLoss/energyLoss3D_OnePrtAlso.pdf}\\
  \includegraphics[width=\linewidth,page=113]{graphics/energyLoss/energyLoss3D_OnePrtAlso.pdf}\\
  \includegraphics[width=\linewidth,page=116]{graphics/energyLoss/energyLoss3D_OnePrtAlso.pdf}\\
}%
\parbox{0.329\textwidth}{
  \includegraphics[width=\linewidth,page=105]{graphics/energyLoss/energyLoss3D_OnePrtAlso.pdf}\\
  \includegraphics[width=\linewidth,page=108]{graphics/energyLoss/energyLoss3D_OnePrtAlso.pdf}\\
  \includegraphics[width=\linewidth,page=111]{graphics/energyLoss/energyLoss3D_OnePrtAlso.pdf}\\
  \includegraphics[width=\linewidth,page=114]{graphics/energyLoss/energyLoss3D_OnePrtAlso.pdf}\\
  \includegraphics[width=\linewidth,page=117]{graphics/energyLoss/energyLoss3D_OnePrtAlso.pdf}\\
}%
\end{figure}

\begin{figure}[H]\ContinuedFloat
% ~\\[32pt]
\vspace{-3.5em}
\parbox{0.329\textwidth}{
  \includegraphics[width=\linewidth,page=118]{graphics/energyLoss/energyLoss3D_OnePrtAlso.pdf}\\
  \vspace{-4em}
}~
\end{figure}
%%%negative
\begin{figure}[H]
\caption[Energy loss correction for negative particles as a function of reconstructed transverse momentum $p_T^{meas}$.]{Energy loss correction $p_T^{meas}-p_T^{true}$ for negative particles as a function of reconstructed transverse momentum $p_T^{meas}$ $\left(|\eta|<0.7\right)$ in single $z$-vertex bin whose range is given on each plot.}\label{fig:energyLossPrimaryNegative}
\parbox{0.329\textwidth}{
  \includegraphics[width=\linewidth,page=123]{graphics/energyLoss/energyLoss3D_OnePrtAlso.pdf}\\
  \includegraphics[width=\linewidth,page=126]{graphics/energyLoss/energyLoss3D_OnePrtAlso.pdf}\\
  \includegraphics[width=\linewidth,page=129]{graphics/energyLoss/energyLoss3D_OnePrtAlso.pdf}\\
  \includegraphics[width=\linewidth,page=132]{graphics/energyLoss/energyLoss3D_OnePrtAlso.pdf}\\
  \includegraphics[width=\linewidth,page=135]{graphics/energyLoss/energyLoss3D_OnePrtAlso.pdf}\\
}~
\parbox{0.329\textwidth}{
  \includegraphics[width=\linewidth,page=124]{graphics/energyLoss/energyLoss3D_OnePrtAlso.pdf}\\
  \includegraphics[width=\linewidth,page=127]{graphics/energyLoss/energyLoss3D_OnePrtAlso.pdf}\\
  \includegraphics[width=\linewidth,page=130]{graphics/energyLoss/energyLoss3D_OnePrtAlso.pdf}\\
  \includegraphics[width=\linewidth,page=133]{graphics/energyLoss/energyLoss3D_OnePrtAlso.pdf}\\
  \includegraphics[width=\linewidth,page=136]{graphics/energyLoss/energyLoss3D_OnePrtAlso.pdf}\\
}%
\parbox{0.329\textwidth}{
  \includegraphics[width=\linewidth,page=125]{graphics/energyLoss/energyLoss3D_OnePrtAlso.pdf}\\
  \includegraphics[width=\linewidth,page=128]{graphics/energyLoss/energyLoss3D_OnePrtAlso.pdf}\\
  \includegraphics[width=\linewidth,page=131]{graphics/energyLoss/energyLoss3D_OnePrtAlso.pdf}\\
  \includegraphics[width=\linewidth,page=134]{graphics/energyLoss/energyLoss3D_OnePrtAlso.pdf}\\
  \includegraphics[width=\linewidth,page=137]{graphics/energyLoss/energyLoss3D_OnePrtAlso.pdf}\\
}%
\end{figure}

\begin{figure}[H]\ContinuedFloat
% ~\\[32pt]
\vspace{-3.5em}
\parbox{0.329\textwidth}{
  \includegraphics[width=\linewidth,page=138]{graphics/energyLoss/energyLoss3D_OnePrtAlso.pdf}\\
  \vspace{-4em}
}~
\end{figure}
%%%positive
\begin{figure}[H]
\caption[Energy loss correction for positive particles as a function of reconstructed transverse momentum $p_T^{meas}$.]{Energy loss correction $p_T^{meas}-p_T^{true}$ for positive particles as a function of reconstructed transverse momentum $p_T^{meas}$ $\left(|\eta|<0.7\right)$ in single $z$-vertex bin whose range is given on each plot.}\label{fig:energyLossPrimaryPositive}
\parbox{0.329\textwidth}{
  \includegraphics[width=\linewidth,page=143]{graphics/energyLoss/energyLoss3D_OnePrtAlso.pdf}\\
  \includegraphics[width=\linewidth,page=146]{graphics/energyLoss/energyLoss3D_OnePrtAlso.pdf}\\
  \includegraphics[width=\linewidth,page=149]{graphics/energyLoss/energyLoss3D_OnePrtAlso.pdf}\\
  \includegraphics[width=\linewidth,page=152]{graphics/energyLoss/energyLoss3D_OnePrtAlso.pdf}\\
  \includegraphics[width=\linewidth,page=155]{graphics/energyLoss/energyLoss3D_OnePrtAlso.pdf}\\
}~
\parbox{0.329\textwidth}{
  \includegraphics[width=\linewidth,page=144]{graphics/energyLoss/energyLoss3D_OnePrtAlso.pdf}\\
  \includegraphics[width=\linewidth,page=147]{graphics/energyLoss/energyLoss3D_OnePrtAlso.pdf}\\
  \includegraphics[width=\linewidth,page=150]{graphics/energyLoss/energyLoss3D_OnePrtAlso.pdf}\\
  \includegraphics[width=\linewidth,page=153]{graphics/energyLoss/energyLoss3D_OnePrtAlso.pdf}\\
  \includegraphics[width=\linewidth,page=156]{graphics/energyLoss/energyLoss3D_OnePrtAlso.pdf}\\
}%
\parbox{0.329\textwidth}{
  \includegraphics[width=\linewidth,page=145]{graphics/energyLoss/energyLoss3D_OnePrtAlso.pdf}\\
  \includegraphics[width=\linewidth,page=148]{graphics/energyLoss/energyLoss3D_OnePrtAlso.pdf}\\
  \includegraphics[width=\linewidth,page=151]{graphics/energyLoss/energyLoss3D_OnePrtAlso.pdf}\\
  \includegraphics[width=\linewidth,page=154]{graphics/energyLoss/energyLoss3D_OnePrtAlso.pdf}\\
  \includegraphics[width=\linewidth,page=157]{graphics/energyLoss/energyLoss3D_OnePrtAlso.pdf}\\
}%
\end{figure}

\begin{figure}[H]\ContinuedFloat
% ~\\[32pt]
\vspace{-3.5em}
\parbox{0.329\textwidth}{
  \includegraphics[width=\linewidth,page=158]{graphics/energyLoss/energyLoss3D_OnePrtAlso.pdf}\\
  \vspace{-4em}
}~
\end{figure}
% %% =====  G4 APERTURES ====
%%===========================================================%%
%%                                                           %%
%%                      Roman Pot Alignment                  %%
%%                                                           %%
%%===========================================================%%

\chapter{Apertures tuning in Geant4 simulation}\label{appendix:g4ApertureTuning}

\begin{figure}[hb]\centering%
\caption{The DX aperture envelopes fitted with circles before (left) and after (right) the DX offsets introduced in the Geant4 geometry.}\label{fig:aperturesWithFit}%
\parbox{0.495\textwidth}{
  \centering
  \includegraphics[width=\linewidth,page=1]{graphics/rpSim/Apertures_swapedAxes_withFit_beforeDxShift.pdf}\\[10pt]
  \includegraphics[width=\linewidth,page=2]{graphics/rpSim/Apertures_swapedAxes_withFit_beforeDxShift.pdf}
}~
\parbox{0.495\textwidth}{
  \centering
  \includegraphics[width=\linewidth,page=1]{graphics/rpSim/Apertures_swapedAxes_withFit.pdf}\\[10pt]
  \includegraphics[width=\linewidth,page=2]{graphics/rpSim/Apertures_swapedAxes_withFit.pdf}
}%
\end{figure}

\begin{figure}[ht]\ContinuedFloat\centering%
	%\caption[Apertures.]{Apertures.}\label{fig:aperturesWithFit}%
	\parbox{0.495\textwidth}{
		\centering
		\includegraphics[width=\linewidth,page=3]{graphics/rpSim/Apertures_swapedAxes_withFit_beforeDxShift.pdf}\\[10pt]
		\includegraphics[width=\linewidth,page=4]{graphics/rpSim/Apertures_swapedAxes_withFit_beforeDxShift.pdf}
	}~
	\parbox{0.495\textwidth}{
		\centering
		\includegraphics[width=\linewidth,page=3]{graphics/rpSim/Apertures_swapedAxes_withFit.pdf}\\[10pt]
		\includegraphics[width=\linewidth,page=4]{graphics/rpSim/Apertures_swapedAxes_withFit.pdf}
	}%
\end{figure}



% %% =====  DEAD MATERIAL CORRECTION ====
%%===========================================================%%
%%                                                           %%
%%              DEAD MATERIAL CORRECTION APPENDIX            %%
%%                                                           %%
%%===========================================================%%

\chapter{Dead material correction to TPC track reconstruction efficiency}\label{appendix:deadMaterial}
%% ===== DE/DX ADJUSTMENT ====
%%===========================================================%%
%%                                                           %%
%%                   DE/DX ADJUSTMENT APPENDIX               %%
%%                                                           %%
%%===========================================================%%

\chapter{Fits to \texorpdfstring{d$\bm{E}$/d$\bm{x}$}{dE/dx} spectra, comparison of \texorpdfstring{d$\bm{E}$/d$\bm{x}$}{dE/dx} and \texorpdfstring{$\bm{n}^{\bm{\sigma}}_{\bm{X}}$}{nSigmaX} between data and MC}\label{appendix:dEdxAdjustment}

\begin{figure}[hb]
\centering%\vspace{-20pt}
\caption[Fits to dE/dx spectra in the data.]{Fits of sum of functions from Eq.~\eqref{eq:expTail} corresponding to different particle species to dE/dx spectra in the data in  momentum bins.}\label{fig:dEdxFits}
\parbox{0.495\textwidth}{
  \centering
  \includegraphics[width=\linewidth,page=1]{graphics/dedx/dEdx_fitPerMomentumBin_4thIteration.pdf}\\[3pt]
  \includegraphics[width=\linewidth,page=3]{graphics/dedx/dEdx_fitPerMomentumBin_4thIteration.pdf}\\[3pt]
  \includegraphics[width=\linewidth,page=5]{graphics/dedx/dEdx_fitPerMomentumBin_4thIteration.pdf}\\[3pt]
  \includegraphics[width=\linewidth,page=7]{graphics/dedx/dEdx_fitPerMomentumBin_4thIteration.pdf}\\[3pt]
  \includegraphics[width=\linewidth,page=9]{graphics/dedx/dEdx_fitPerMomentumBin_4thIteration.pdf}\\[3pt]
  \includegraphics[width=\linewidth,page=11]{graphics/dedx/dEdx_fitPerMomentumBin_4thIteration.pdf}
}~
\parbox{0.495\textwidth}{
  \centering
  \includegraphics[width=\linewidth,page=2]{graphics/dedx/dEdx_fitPerMomentumBin_4thIteration.pdf}\\[3pt]
  \includegraphics[width=\linewidth,page=4]{graphics/dedx/dEdx_fitPerMomentumBin_4thIteration.pdf}\\[3pt]
  \includegraphics[width=\linewidth,page=6]{graphics/dedx/dEdx_fitPerMomentumBin_4thIteration.pdf}\\[3pt]
  \includegraphics[width=\linewidth,page=8]{graphics/dedx/dEdx_fitPerMomentumBin_4thIteration.pdf}\\[3pt]
  \includegraphics[width=\linewidth,page=10]{graphics/dedx/dEdx_fitPerMomentumBin_4thIteration.pdf}\\[3pt]
  \includegraphics[width=\linewidth,page=12]{graphics/dedx/dEdx_fitPerMomentumBin_4thIteration.pdf}
}\vspace{-400pt}%
\end{figure}
%---------------------------

\begin{figure}[ht]
\ContinuedFloat%
\centering%
\parbox{0.495\textwidth}{
  \centering
  \includegraphics[width=\linewidth,page=13]{graphics/dedx/dEdx_fitPerMomentumBin_4thIteration.pdf}\\[3pt]
  \includegraphics[width=\linewidth,page=15]{graphics/dedx/dEdx_fitPerMomentumBin_4thIteration.pdf}\\[3pt]
  \includegraphics[width=\linewidth,page=17]{graphics/dedx/dEdx_fitPerMomentumBin_4thIteration.pdf}\\[3pt]
  \includegraphics[width=\linewidth,page=19]{graphics/dedx/dEdx_fitPerMomentumBin_4thIteration.pdf}\\[3pt]
  \includegraphics[width=\linewidth,page=21]{graphics/dedx/dEdx_fitPerMomentumBin_4thIteration.pdf}\\[3pt]
  \includegraphics[width=\linewidth,page=23]{graphics/dedx/dEdx_fitPerMomentumBin_4thIteration.pdf}\\[3pt]
  \includegraphics[width=\linewidth,page=25]{graphics/dedx/dEdx_fitPerMomentumBin_4thIteration.pdf}
}~
\parbox{0.495\textwidth}{
  \centering
  \includegraphics[width=\linewidth,page=14]{graphics/dedx/dEdx_fitPerMomentumBin_4thIteration.pdf}\\[3pt]
  \includegraphics[width=\linewidth,page=16]{graphics/dedx/dEdx_fitPerMomentumBin_4thIteration.pdf}\\[3pt]
  \includegraphics[width=\linewidth,page=18]{graphics/dedx/dEdx_fitPerMomentumBin_4thIteration.pdf}\\[3pt]
  \includegraphics[width=\linewidth,page=20]{graphics/dedx/dEdx_fitPerMomentumBin_4thIteration.pdf}\\[3pt]
  \includegraphics[width=\linewidth,page=22]{graphics/dedx/dEdx_fitPerMomentumBin_4thIteration.pdf}\\[3pt]
  \includegraphics[width=\linewidth,page=24]{graphics/dedx/dEdx_fitPerMomentumBin_4thIteration.pdf}\\[3pt]
  \includegraphics[width=\linewidth,page=26]{graphics/dedx/dEdx_fitPerMomentumBin_4thIteration.pdf}
}%
\end{figure}
%---------------------------


\begin{figure}[t!]%\vspace{-87pt}
\ContinuedFloat%
\centering%
\parbox{0.495\textwidth}{
  \centering
  \includegraphics[width=\linewidth,page=27]{graphics/dedx/dEdx_fitPerMomentumBin_4thIteration.pdf}\\[3pt]
  \includegraphics[width=\linewidth,page=29]{graphics/dedx/dEdx_fitPerMomentumBin_4thIteration.pdf}\\[3pt]
  \includegraphics[width=\linewidth,page=31]{graphics/dedx/dEdx_fitPerMomentumBin_4thIteration.pdf}\\[3pt]
  \includegraphics[width=\linewidth,page=33]{graphics/dedx/dEdx_fitPerMomentumBin_4thIteration.pdf}\\[3pt]
  \includegraphics[width=\linewidth,page=35]{graphics/dedx/dEdx_fitPerMomentumBin_4thIteration.pdf}\\[3pt]
  \includegraphics[width=\linewidth,page=37]{graphics/dedx/dEdx_fitPerMomentumBin_4thIteration.pdf}\\[3pt]
  \includegraphics[width=\linewidth,page=39]{graphics/dedx/dEdx_fitPerMomentumBin_4thIteration.pdf}
}~
\parbox{0.495\textwidth}{
  \centering
  \includegraphics[width=\linewidth,page=28]{graphics/dedx/dEdx_fitPerMomentumBin_4thIteration.pdf}\\[3pt]
  \includegraphics[width=\linewidth,page=30]{graphics/dedx/dEdx_fitPerMomentumBin_4thIteration.pdf}\\[3pt]
  \includegraphics[width=\linewidth,page=32]{graphics/dedx/dEdx_fitPerMomentumBin_4thIteration.pdf}\\[3pt]
  \includegraphics[width=\linewidth,page=34]{graphics/dedx/dEdx_fitPerMomentumBin_4thIteration.pdf}\\[3pt]
  \includegraphics[width=\linewidth,page=36]{graphics/dedx/dEdx_fitPerMomentumBin_4thIteration.pdf}\\[3pt]
  \includegraphics[width=\linewidth,page=38]{graphics/dedx/dEdx_fitPerMomentumBin_4thIteration.pdf}\\[3pt]
  \includegraphics[width=\linewidth,page=40]{graphics/dedx/dEdx_fitPerMomentumBin_4thIteration.pdf}
}%
\end{figure}
%---------------------------
%% ===== TAG & PROBE TOF EFFICIENCY ====
%%===========================================================%%
%%                                                           %%
%%             TAG & PROBE TOF EFFICIENCY APPENDIX           %%
%%                                                           %%
%%===========================================================%%

\chapter{Tag\texorpdfstring{$\&$}{&}Probe fits for TOF hit reconstruction and matching efficiency}\label{appendix:tagAndProbeTofEff}

 
%---------------------------
\begin{figure}[h!]
\caption[Tag\&Probe fits to $p_{T}^{\text{miss}}$ in bins of probe $p_{T}$.]{Total transverse momentum $p_{T}^{\text{miss}}$ of the $p$+Tag+Probe+$p$ system in the data and signal+background embedded MC, in bins of $p_{T}$ of a probe. Adjacent plots are for the same $p_{T}$ bin, one for data (left) and the other for MC (right).}\label{fig:tagAndProbeTofEffFits_Pt}
\centering
\parbox{0.24\textwidth}{ 
  \centering
  \includegraphics[width=\linewidth,page=5]{graphics/correctionsToEff/TOF_tagAndProbe/Fitting_effVsPt_data.CPT2.pdf}\\
  \includegraphics[width=\linewidth,page=7]{graphics/correctionsToEff/TOF_tagAndProbe/Fitting_effVsPt_data.CPT2.pdf}\\
  \includegraphics[width=\linewidth,page=9]{graphics/correctionsToEff/TOF_tagAndProbe/Fitting_effVsPt_data.CPT2.pdf}\\
  \includegraphics[width=\linewidth,page=11]{graphics/correctionsToEff/TOF_tagAndProbe/Fitting_effVsPt_data.CPT2.pdf}

}~
\parbox{0.24\textwidth}{
  \centering
  \includegraphics[width=\linewidth,page=5]{graphics/correctionsToEff/TOF_tagAndProbe/Fitting_effVsPt_mc.CPT2.pdf}\\
  \includegraphics[width=\linewidth,page=7]{graphics/correctionsToEff/TOF_tagAndProbe/Fitting_effVsPt_mc.CPT2.pdf}\\
  \includegraphics[width=\linewidth,page=9]{graphics/correctionsToEff/TOF_tagAndProbe/Fitting_effVsPt_mc.CPT2.pdf}\\
  \includegraphics[width=\linewidth,page=11]{graphics/correctionsToEff/TOF_tagAndProbe/Fitting_effVsPt_mc.CPT2.pdf}

}~~~~
\parbox{0.24\textwidth}{ 
  \centering
  \includegraphics[width=\linewidth,page=6]{graphics/correctionsToEff/TOF_tagAndProbe/Fitting_effVsPt_data.CPT2.pdf}\\
  \includegraphics[width=\linewidth,page=8]{graphics/correctionsToEff/TOF_tagAndProbe/Fitting_effVsPt_data.CPT2.pdf}\\
  \includegraphics[width=\linewidth,page=10]{graphics/correctionsToEff/TOF_tagAndProbe/Fitting_effVsPt_data.CPT2.pdf}\\[84pt]

}~
\parbox{0.24\textwidth}{
  \centering
  \includegraphics[width=\linewidth,page=6]{graphics/correctionsToEff/TOF_tagAndProbe/Fitting_effVsPt_mc.CPT2.pdf}\\
  \includegraphics[width=\linewidth,page=8]{graphics/correctionsToEff/TOF_tagAndProbe/Fitting_effVsPt_mc.CPT2.pdf}\\
  \includegraphics[width=\linewidth,page=10]{graphics/correctionsToEff/TOF_tagAndProbe/Fitting_effVsPt_mc.CPT2.pdf}\\[84pt]

}%
\end{figure}
%---------------------------
%
%
%
%
%---------------------------
\begin{figure}[h!]
\caption[Tag\&Probe fits to $p_{T}^{\text{miss}}$ in bins of probe $\eta$.]{Total transverse momentum $p_{T}^{\text{miss}}$ of the $p$+Tag+Probe+$p$ system in the data and signal+background embedded MC, in bins of $\eta$ of a probe. Adjacent plots are for the same $\eta$ bin, one for data (left) and the other for MC (right).}\label{fig:tagAndProbeTofEffFits_Eta}
\centering
\parbox{0.24\textwidth}{ 
  \centering
  \includegraphics[width=\linewidth,page=3]{graphics/correctionsToEff/TOF_tagAndProbe/Fitting_effVsEta_data.CPT2.pdf}%\\
%   \includegraphics[width=\linewidth,page=5]{graphics/correctionsToEff/TOF_tagAndProbe/Fitting_effVsEta_data.CPT2.pdf}%\\
%   \includegraphics[width=\linewidth,page=7]{graphics/correctionsToEff/TOF_tagAndProbe/Fitting_effVsEta_data.CPT2.pdf}\\
%   \includegraphics[width=\linewidth,page=9]{graphics/correctionsToEff/TOF_tagAndProbe/Fitting_effVsEta_data.CPT2.pdf}

}~
\parbox{0.24\textwidth}{
  \centering
  \includegraphics[width=\linewidth,page=3]{graphics/correctionsToEff/TOF_tagAndProbe/Fitting_effVsEta_mc.CPT2.pdf}%\\
%   \includegraphics[width=\linewidth,page=5]{graphics/correctionsToEff/TOF_tagAndProbe/Fitting_effVsEta_mc.CPT2.pdf}%\\
%   \includegraphics[width=\linewidth,page=7]{graphics/correctionsToEff/TOF_tagAndProbe/Fitting_effVsEta_mc.CPT2.pdf}\\
%   \includegraphics[width=\linewidth,page=9]{graphics/correctionsToEff/TOF_tagAndProbe/Fitting_effVsEta_mc.CPT2.pdf}
}~~~~
\parbox{0.24\textwidth}{
  \centering
  \includegraphics[width=\linewidth,page=4]{graphics/correctionsToEff/TOF_tagAndProbe/Fitting_effVsEta_data.CPT2.pdf}%\\
%   \includegraphics[width=\linewidth,page=6]{graphics/correctionsToEff/TOF_tagAndProbe/Fitting_effVsEta_data.CPT2.pdf}%\\
%   \includegraphics[width=\linewidth,page=8]{graphics/correctionsToEff/TOF_tagAndProbe/Fitting_effVsEta_data.CPT2.pdf}\\
%   \includegraphics[width=\linewidth,page=10]{graphics/correctionsToEff/TOF_tagAndProbe/Fitting_effVsEta_data.CPT2.pdf}

}~
\parbox{0.24\textwidth}{
  \centering
  \includegraphics[width=\linewidth,page=4]{graphics/correctionsToEff/TOF_tagAndProbe/Fitting_effVsEta_mc.CPT2.pdf}%\\
%   \includegraphics[width=\linewidth,page=6]{graphics/correctionsToEff/TOF_tagAndProbe/Fitting_effVsEta_mc.CPT2.pdf}%\\
%   \includegraphics[width=\linewidth,page=8]{graphics/correctionsToEff/TOF_tagAndProbe/Fitting_effVsEta_mc.CPT2.pdf}\\
%   \includegraphics[width=\linewidth,page=10]{graphics/correctionsToEff/TOF_tagAndProbe/Fitting_effVsEta_mc.CPT2.pdf}
}
\end{figure}
\begin{figure}[ht]\ContinuedFloat
\centering
\parbox{0.24\textwidth}{ 
  \centering
%   \includegraphics[width=\linewidth,page=3]{graphics/correctionsToEff/TOF_tagAndProbe/Fitting_effVsEta_data.CPT2.pdf}\\
  \includegraphics[width=\linewidth,page=5]{graphics/correctionsToEff/TOF_tagAndProbe/Fitting_effVsEta_data.CPT2.pdf}\\
  \includegraphics[width=\linewidth,page=7]{graphics/correctionsToEff/TOF_tagAndProbe/Fitting_effVsEta_data.CPT2.pdf}\\
  \includegraphics[width=\linewidth,page=9]{graphics/correctionsToEff/TOF_tagAndProbe/Fitting_effVsEta_data.CPT2.pdf}

}~
\parbox{0.24\textwidth}{
  \centering
%   \includegraphics[width=\linewidth,page=3]{graphics/correctionsToEff/TOF_tagAndProbe/Fitting_effVsEta_mc.CPT2.pdf}\\
  \includegraphics[width=\linewidth,page=5]{graphics/correctionsToEff/TOF_tagAndProbe/Fitting_effVsEta_mc.CPT2.pdf}\\
  \includegraphics[width=\linewidth,page=7]{graphics/correctionsToEff/TOF_tagAndProbe/Fitting_effVsEta_mc.CPT2.pdf}\\
  \includegraphics[width=\linewidth,page=9]{graphics/correctionsToEff/TOF_tagAndProbe/Fitting_effVsEta_mc.CPT2.pdf}
}~~~~
\parbox{0.24\textwidth}{
  \centering
%   \includegraphics[width=\linewidth,page=4]{graphics/correctionsToEff/TOF_tagAndProbe/Fitting_effVsEta_data.CPT2.pdf}\\
  \includegraphics[width=\linewidth,page=6]{graphics/correctionsToEff/TOF_tagAndProbe/Fitting_effVsEta_data.CPT2.pdf}\\
  \includegraphics[width=\linewidth,page=8]{graphics/correctionsToEff/TOF_tagAndProbe/Fitting_effVsEta_data.CPT2.pdf}\\
  \includegraphics[width=\linewidth,page=10]{graphics/correctionsToEff/TOF_tagAndProbe/Fitting_effVsEta_data.CPT2.pdf}

}~
\parbox{0.24\textwidth}{
  \centering
%   \includegraphics[width=\linewidth,page=4]{graphics/correctionsToEff/TOF_tagAndProbe/Fitting_effVsEta_mc.CPT2.pdf}\\
  \includegraphics[width=\linewidth,page=6]{graphics/correctionsToEff/TOF_tagAndProbe/Fitting_effVsEta_mc.CPT2.pdf}\\
  \includegraphics[width=\linewidth,page=8]{graphics/correctionsToEff/TOF_tagAndProbe/Fitting_effVsEta_mc.CPT2.pdf}\\
  \includegraphics[width=\linewidth,page=10]{graphics/correctionsToEff/TOF_tagAndProbe/Fitting_effVsEta_mc.CPT2.pdf}
}
\end{figure}
%---------------------------



\begin{figure}[h!]
\caption[Tag\&Probe fits to $p_{T}^{\text{miss}}$ in bins of probe $z_{\text{vtx}}$.]{Total transverse momentum $p_{T}^{\text{miss}}$ of the $p$+Tag+Probe+$p$ system in the data and signal+background embedded MC, in bins of $z_{\text{vtx}}$ of a probe. Adjacent plots are for the same $z_{\text{vtx}}$ bin, one for data (left) and the other for MC (right).}\label{fig:tagAndProbeTofEffFits_ZVtx}
\centering
\parbox{0.24\textwidth}{  
  \centering
  \includegraphics[width=\linewidth,page=3]{graphics/correctionsToEff/TOF_tagAndProbe/Fitting_effVsZVtx_data.CPT2.pdf}\\
  \includegraphics[width=\linewidth,page=5]{graphics/correctionsToEff/TOF_tagAndProbe/Fitting_effVsZVtx_data.CPT2.pdf}\\
  \includegraphics[width=\linewidth,page=7]{graphics/correctionsToEff/TOF_tagAndProbe/Fitting_effVsZVtx_data.CPT2.pdf}\\
  \includegraphics[width=\linewidth,page=9]{graphics/correctionsToEff/TOF_tagAndProbe/Fitting_effVsZVtx_data.CPT2.pdf}

}~
\parbox{0.24\textwidth}{
  \centering
  \includegraphics[width=\linewidth,page=3]{graphics/correctionsToEff/TOF_tagAndProbe/Fitting_effVsZVtx_mc.CPT2.pdf}\\
  \includegraphics[width=\linewidth,page=5]{graphics/correctionsToEff/TOF_tagAndProbe/Fitting_effVsZVtx_mc.CPT2.pdf}\\
  \includegraphics[width=\linewidth,page=7]{graphics/correctionsToEff/TOF_tagAndProbe/Fitting_effVsZVtx_mc.CPT2.pdf}\\
  \includegraphics[width=\linewidth,page=9]{graphics/correctionsToEff/TOF_tagAndProbe/Fitting_effVsZVtx_mc.CPT2.pdf}
}~~~~
\parbox{0.24\textwidth}{
  \centering
  \includegraphics[width=\linewidth,page=4]{graphics/correctionsToEff/TOF_tagAndProbe/Fitting_effVsZVtx_data.CPT2.pdf}\\
  \includegraphics[width=\linewidth,page=6]{graphics/correctionsToEff/TOF_tagAndProbe/Fitting_effVsZVtx_data.CPT2.pdf}\\
  \includegraphics[width=\linewidth,page=8]{graphics/correctionsToEff/TOF_tagAndProbe/Fitting_effVsZVtx_data.CPT2.pdf}\\
  \includegraphics[width=\linewidth,page=10]{graphics/correctionsToEff/TOF_tagAndProbe/Fitting_effVsZVtx_data.CPT2.pdf}

}~
\parbox{0.24\textwidth}{
  \centering
  \includegraphics[width=\linewidth,page=4]{graphics/correctionsToEff/TOF_tagAndProbe/Fitting_effVsZVtx_mc.CPT2.pdf}\\
  \includegraphics[width=\linewidth,page=6]{graphics/correctionsToEff/TOF_tagAndProbe/Fitting_effVsZVtx_mc.CPT2.pdf}\\
  \includegraphics[width=\linewidth,page=8]{graphics/correctionsToEff/TOF_tagAndProbe/Fitting_effVsZVtx_mc.CPT2.pdf}\\
  \includegraphics[width=\linewidth,page=10]{graphics/correctionsToEff/TOF_tagAndProbe/Fitting_effVsZVtx_mc.CPT2.pdf}
}
\end{figure}
%
%
%
%

%
%
%
%
%
%
%
%
%

%
%---------------------------
\begin{figure}[h!]
\caption[Tag\&Probe fits to $p_{T}^{\text{miss}}$ in bins of probe $p_{T}$.]{Total transverse momentum $p_{T}^{\text{miss}}$ of the $p$+Tag+Probe+$p$ system in the data and signal+background embedded MC, in bins of $p_{T}$ of a probe for probe pseudorapidity range $-0.7<\eta<-0.3$. Adjacent plots are for the same $p_{T}$ bin, one for data (left) and the other for MC (right).}\label{fig:tagAndProbeTofEffFits_Pt_BinD}
\centering
\parbox{0.24\textwidth}{ 
  \centering
  \includegraphics[width=\linewidth,page=5]{graphics/correctionsToEff/TOF_tagAndProbe/Fitting_effVsPt_data_ETABINS_D.CPT2.pdf}\\
  \includegraphics[width=\linewidth,page=7]{graphics/correctionsToEff/TOF_tagAndProbe/Fitting_effVsPt_data_ETABINS_D.CPT2.pdf}\\
  \includegraphics[width=\linewidth,page=9]{graphics/correctionsToEff/TOF_tagAndProbe/Fitting_effVsPt_data_ETABINS_D.CPT2.pdf}\\
  \includegraphics[width=\linewidth,page=11]{graphics/correctionsToEff/TOF_tagAndProbe/Fitting_effVsPt_data_ETABINS_D.CPT2.pdf}

}~
\parbox{0.24\textwidth}{
  \centering
  \includegraphics[width=\linewidth,page=5]{graphics/correctionsToEff/TOF_tagAndProbe/Fitting_effVsPt_mc_ETABINS_D.CPT2.pdf}\\
  \includegraphics[width=\linewidth,page=7]{graphics/correctionsToEff/TOF_tagAndProbe/Fitting_effVsPt_mc_ETABINS_D.CPT2.pdf}\\
  \includegraphics[width=\linewidth,page=9]{graphics/correctionsToEff/TOF_tagAndProbe/Fitting_effVsPt_mc_ETABINS_D.CPT2.pdf}\\
  \includegraphics[width=\linewidth,page=11]{graphics/correctionsToEff/TOF_tagAndProbe/Fitting_effVsPt_mc_ETABINS_D.CPT2.pdf}

}~~~~
\parbox{0.24\textwidth}{ 
  \centering
  \includegraphics[width=\linewidth,page=6]{graphics/correctionsToEff/TOF_tagAndProbe/Fitting_effVsPt_data_ETABINS_D.CPT2.pdf}\\
  \includegraphics[width=\linewidth,page=8]{graphics/correctionsToEff/TOF_tagAndProbe/Fitting_effVsPt_data_ETABINS_D.CPT2.pdf}\\
  \includegraphics[width=\linewidth,page=10]{graphics/correctionsToEff/TOF_tagAndProbe/Fitting_effVsPt_data_ETABINS_D.CPT2.pdf}\\[84pt]

}~
\parbox{0.24\textwidth}{
  \centering
  \includegraphics[width=\linewidth,page=6]{graphics/correctionsToEff/TOF_tagAndProbe/Fitting_effVsPt_mc_ETABINS_D.CPT2.pdf}\\
  \includegraphics[width=\linewidth,page=8]{graphics/correctionsToEff/TOF_tagAndProbe/Fitting_effVsPt_mc_ETABINS_D.CPT2.pdf}\\
  \includegraphics[width=\linewidth,page=10]{graphics/correctionsToEff/TOF_tagAndProbe/Fitting_effVsPt_mc_ETABINS_D.CPT2.pdf}\\[84pt]

}%
\end{figure}
%---------------------------


%
%---------------------------
\begin{figure}[h!]
\caption[Tag\&Probe fits to $p_{T}^{\text{miss}}$ in bins of probe $p_{T}$.]{Total transverse momentum $p_{T}^{\text{miss}}$ of the $p$+Tag+Probe+$p$ system in the data and signal+background embedded MC, in bins of $p_{T}$ of a probe for probe pseudorapidity range $-0.3<\eta<0$. Adjacent plots are for the same $p_{T}$ bin, one for data (left) and the other for MC (right).}\label{fig:tagAndProbeTofEffFits_Pt_BinC}
\centering
\parbox{0.24\textwidth}{ 
  \centering
  \includegraphics[width=\linewidth,page=5]{graphics/correctionsToEff/TOF_tagAndProbe/Fitting_effVsPt_data_ETABINS_C.CPT2.pdf}%\\
%   \includegraphics[width=\linewidth,page=7]{graphics/correctionsToEff/TOF_tagAndProbe/Fitting_effVsPt_data_ETABINS_C.CPT2.pdf}\\
%   \includegraphics[width=\linewidth,page=9]{graphics/correctionsToEff/TOF_tagAndProbe/Fitting_effVsPt_data_ETABINS_C.CPT2.pdf}\\
%   \includegraphics[width=\linewidth,page=11]{graphics/correctionsToEff/TOF_tagAndProbe/Fitting_effVsPt_data_ETABINS_C.CPT2.pdf}

}~
\parbox{0.24\textwidth}{
  \centering
  \includegraphics[width=\linewidth,page=5]{graphics/correctionsToEff/TOF_tagAndProbe/Fitting_effVsPt_mc_ETABINS_C.CPT2.pdf}%\\
%   \includegraphics[width=\linewidth,page=7]{graphics/correctionsToEff/TOF_tagAndProbe/Fitting_effVsPt_mc_ETABINS_C.CPT2.pdf}\\
%   \includegraphics[width=\linewidth,page=9]{graphics/correctionsToEff/TOF_tagAndProbe/Fitting_effVsPt_mc_ETABINS_C.CPT2.pdf}\\
%   \includegraphics[width=\linewidth,page=11]{graphics/correctionsToEff/TOF_tagAndProbe/Fitting_effVsPt_mc_ETABINS_C.CPT2.pdf}

}~~~~
\parbox{0.24\textwidth}{ 
  \centering
  \includegraphics[width=\linewidth,page=6]{graphics/correctionsToEff/TOF_tagAndProbe/Fitting_effVsPt_data_ETABINS_C.CPT2.pdf}%\\
%   \includegraphics[width=\linewidth,page=8]{graphics/correctionsToEff/TOF_tagAndProbe/Fitting_effVsPt_data_ETABINS_C.CPT2.pdf}\\
%   \includegraphics[width=\linewidth,page=10]{graphics/correctionsToEff/TOF_tagAndProbe/Fitting_effVsPt_data_ETABINS_C.CPT2.pdf}\\[84pt]

}~
\parbox{0.24\textwidth}{
  \centering
  \includegraphics[width=\linewidth,page=6]{graphics/correctionsToEff/TOF_tagAndProbe/Fitting_effVsPt_mc_ETABINS_C.CPT2.pdf}%\\
%   \includegraphics[width=\linewidth,page=8]{graphics/correctionsToEff/TOF_tagAndProbe/Fitting_effVsPt_mc_ETABINS_C.CPT2.pdf}\\
%   \includegraphics[width=\linewidth,page=10]{graphics/correctionsToEff/TOF_tagAndProbe/Fitting_effVsPt_mc_ETABINS_C.CPT2.pdf}\\[84pt]

}%
\end{figure}
\begin{figure}[ht]\ContinuedFloat
\centering
\parbox{0.24\textwidth}{ 
  \centering
%   \includegraphics[width=\linewidth,page=5]{graphics/correctionsToEff/TOF_tagAndProbe/Fitting_effVsPt_data_ETABINS_C.CPT2.pdf}\\
  \includegraphics[width=\linewidth,page=7]{graphics/correctionsToEff/TOF_tagAndProbe/Fitting_effVsPt_data_ETABINS_C.CPT2.pdf}\\
  \includegraphics[width=\linewidth,page=9]{graphics/correctionsToEff/TOF_tagAndProbe/Fitting_effVsPt_data_ETABINS_C.CPT2.pdf}\\
  \includegraphics[width=\linewidth,page=11]{graphics/correctionsToEff/TOF_tagAndProbe/Fitting_effVsPt_data_ETABINS_C.CPT2.pdf}

}~
\parbox{0.24\textwidth}{
  \centering
%   \includegraphics[width=\linewidth,page=5]{graphics/correctionsToEff/TOF_tagAndProbe/Fitting_effVsPt_mc_ETABINS_C.CPT2.pdf}\\
  \includegraphics[width=\linewidth,page=7]{graphics/correctionsToEff/TOF_tagAndProbe/Fitting_effVsPt_mc_ETABINS_C.CPT2.pdf}\\
  \includegraphics[width=\linewidth,page=9]{graphics/correctionsToEff/TOF_tagAndProbe/Fitting_effVsPt_mc_ETABINS_C.CPT2.pdf}\\
  \includegraphics[width=\linewidth,page=11]{graphics/correctionsToEff/TOF_tagAndProbe/Fitting_effVsPt_mc_ETABINS_C.CPT2.pdf}

}~~~~
\parbox{0.24\textwidth}{ 
  \centering
%   \includegraphics[width=\linewidth,page=6]{graphics/correctionsToEff/TOF_tagAndProbe/Fitting_effVsPt_data_ETABINS_C.CPT2.pdf}\\
  \includegraphics[width=\linewidth,page=8]{graphics/correctionsToEff/TOF_tagAndProbe/Fitting_effVsPt_data_ETABINS_C.CPT2.pdf}\\
  \includegraphics[width=\linewidth,page=10]{graphics/correctionsToEff/TOF_tagAndProbe/Fitting_effVsPt_data_ETABINS_C.CPT2.pdf}\\[84pt]

}~
\parbox{0.24\textwidth}{
  \centering
%   \includegraphics[width=\linewidth,page=6]{graphics/correctionsToEff/TOF_tagAndProbe/Fitting_effVsPt_mc_ETABINS_C.CPT2.pdf}\\
  \includegraphics[width=\linewidth,page=8]{graphics/correctionsToEff/TOF_tagAndProbe/Fitting_effVsPt_mc_ETABINS_C.CPT2.pdf}\\
  \includegraphics[width=\linewidth,page=10]{graphics/correctionsToEff/TOF_tagAndProbe/Fitting_effVsPt_mc_ETABINS_C.CPT2.pdf}\\[84pt]

}%
\end{figure}
%---------------------------


%
%---------------------------
\begin{figure}[h!]
\caption[Tag\&Probe fits to $p_{T}^{\text{miss}}$ in bins of probe $p_{T}$.]{Total transverse momentum $p_{T}^{\text{miss}}$ of the $p$+Tag+Probe+$p$ system in the data and signal+background embedded MC, in bins of $p_{T}$ of a probe for probe pseudorapidity range $0<\eta<0.3$. Adjacent plots are for the same $p_{T}$ bin, one for data (left) and the other for MC (right).}\label{fig:tagAndProbeTofEffFits_Pt_BinB}
\centering
\parbox{0.24\textwidth}{ 
  \centering
  \includegraphics[width=\linewidth,page=5]{graphics/correctionsToEff/TOF_tagAndProbe/Fitting_effVsPt_data_ETABINS_B.CPT2.pdf}\\
  \includegraphics[width=\linewidth,page=7]{graphics/correctionsToEff/TOF_tagAndProbe/Fitting_effVsPt_data_ETABINS_B.CPT2.pdf}\\
  \includegraphics[width=\linewidth,page=9]{graphics/correctionsToEff/TOF_tagAndProbe/Fitting_effVsPt_data_ETABINS_B.CPT2.pdf}\\
  \includegraphics[width=\linewidth,page=11]{graphics/correctionsToEff/TOF_tagAndProbe/Fitting_effVsPt_data_ETABINS_B.CPT2.pdf}

}~
\parbox{0.24\textwidth}{
  \centering
  \includegraphics[width=\linewidth,page=5]{graphics/correctionsToEff/TOF_tagAndProbe/Fitting_effVsPt_mc_ETABINS_B.CPT2.pdf}\\
  \includegraphics[width=\linewidth,page=7]{graphics/correctionsToEff/TOF_tagAndProbe/Fitting_effVsPt_mc_ETABINS_B.CPT2.pdf}\\
  \includegraphics[width=\linewidth,page=9]{graphics/correctionsToEff/TOF_tagAndProbe/Fitting_effVsPt_mc_ETABINS_B.CPT2.pdf}\\
  \includegraphics[width=\linewidth,page=11]{graphics/correctionsToEff/TOF_tagAndProbe/Fitting_effVsPt_mc_ETABINS_B.CPT2.pdf}

}~~~~
\parbox{0.24\textwidth}{ 
  \centering
  \includegraphics[width=\linewidth,page=6]{graphics/correctionsToEff/TOF_tagAndProbe/Fitting_effVsPt_data_ETABINS_B.CPT2.pdf}\\
  \includegraphics[width=\linewidth,page=8]{graphics/correctionsToEff/TOF_tagAndProbe/Fitting_effVsPt_data_ETABINS_B.CPT2.pdf}\\
  \includegraphics[width=\linewidth,page=10]{graphics/correctionsToEff/TOF_tagAndProbe/Fitting_effVsPt_data_ETABINS_B.CPT2.pdf}\\[84pt]

}~
\parbox{0.24\textwidth}{
  \centering
  \includegraphics[width=\linewidth,page=6]{graphics/correctionsToEff/TOF_tagAndProbe/Fitting_effVsPt_mc_ETABINS_B.CPT2.pdf}\\
  \includegraphics[width=\linewidth,page=8]{graphics/correctionsToEff/TOF_tagAndProbe/Fitting_effVsPt_mc_ETABINS_B.CPT2.pdf}\\
  \includegraphics[width=\linewidth,page=10]{graphics/correctionsToEff/TOF_tagAndProbe/Fitting_effVsPt_mc_ETABINS_B.CPT2.pdf}\\[84pt]

}%
\end{figure}
%---------------------------


%
%---------------------------
\begin{figure}[ht]\vspace{-260pt}
\caption[Tag\&Probe fits to $p_{T}^{\text{miss}}$ in bins of probe $p_{T}$.]{Total transverse momentum $p_{T}^{\text{miss}}$ of the $p$+Tag+Probe+$p$ system in the data and signal+background embedded MC, in bins of $p_{T}$ of a probe for probe pseudorapidity range $0.3<\eta<0.7$. Adjacent plots are for the same $p_{T}$ bin, one for data (left) and the other for MC (right).}\label{fig:tagAndProbeTofEffFits_Pt_BinA}
\centering
\parbox{0.24\textwidth}{ 
  \centering
  \includegraphics[width=\linewidth,page=5]{graphics/correctionsToEff/TOF_tagAndProbe/Fitting_effVsPt_data_ETABINS_A.CPT2.pdf}\\
  \includegraphics[width=\linewidth,page=7]{graphics/correctionsToEff/TOF_tagAndProbe/Fitting_effVsPt_data_ETABINS_A.CPT2.pdf}\\
  \includegraphics[width=\linewidth,page=9]{graphics/correctionsToEff/TOF_tagAndProbe/Fitting_effVsPt_data_ETABINS_A.CPT2.pdf}\\
  \includegraphics[width=\linewidth,page=11]{graphics/correctionsToEff/TOF_tagAndProbe/Fitting_effVsPt_data_ETABINS_A.CPT2.pdf}

}~
\parbox{0.24\textwidth}{
  \centering
  \includegraphics[width=\linewidth,page=5]{graphics/correctionsToEff/TOF_tagAndProbe/Fitting_effVsPt_mc_ETABINS_A.CPT2.pdf}\\
  \includegraphics[width=\linewidth,page=7]{graphics/correctionsToEff/TOF_tagAndProbe/Fitting_effVsPt_mc_ETABINS_A.CPT2.pdf}\\
  \includegraphics[width=\linewidth,page=9]{graphics/correctionsToEff/TOF_tagAndProbe/Fitting_effVsPt_mc_ETABINS_A.CPT2.pdf}\\
  \includegraphics[width=\linewidth,page=11]{graphics/correctionsToEff/TOF_tagAndProbe/Fitting_effVsPt_mc_ETABINS_A.CPT2.pdf}

}~~~~
\parbox{0.24\textwidth}{ 
  \centering
  \includegraphics[width=\linewidth,page=6]{graphics/correctionsToEff/TOF_tagAndProbe/Fitting_effVsPt_data_ETABINS_A.CPT2.pdf}\\
  \includegraphics[width=\linewidth,page=8]{graphics/correctionsToEff/TOF_tagAndProbe/Fitting_effVsPt_data_ETABINS_A.CPT2.pdf}\\
  \includegraphics[width=\linewidth,page=10]{graphics/correctionsToEff/TOF_tagAndProbe/Fitting_effVsPt_data_ETABINS_A.CPT2.pdf}\\[84pt]

}~
\parbox{0.24\textwidth}{
  \centering
  \includegraphics[width=\linewidth,page=6]{graphics/correctionsToEff/TOF_tagAndProbe/Fitting_effVsPt_mc_ETABINS_A.CPT2.pdf}\\
  \includegraphics[width=\linewidth,page=8]{graphics/correctionsToEff/TOF_tagAndProbe/Fitting_effVsPt_mc_ETABINS_A.CPT2.pdf}\\
  \includegraphics[width=\linewidth,page=10]{graphics/correctionsToEff/TOF_tagAndProbe/Fitting_effVsPt_mc_ETABINS_A.CPT2.pdf}\\[84pt]

}%
\end{figure}
%---------------------------

\end{appendices}

\listoffigures
\addcontentsline{toc}{chapter}{List of Figures}
\begingroup
\let\clearpage\relax
\listoftables
\addcontentsline{toc}{chapter}{List of Tables}
\endgroup

\bibliography{references.bib}{}
\bibliographystyle{utphys}
\addcontentsline{toc}{chapter}{References}

\end{document}          
