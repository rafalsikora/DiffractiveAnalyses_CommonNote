%%===========================================================%%
%%                                                           %%
%%                   TPC TRACK QUALITY CUTS                  %%
%%                                                           %%
%%===========================================================%%


\chapter{TPC track cuts}\label{chap:TpcTrackQualityCuts}

Charged particle tracks reconstructed in the TPC are selected in our analyses with the set of cuts listed below. The goal of these criteria is to reject off-time pile-up tracks and ensure satisfactory resolution of the track momentum and specific energy loss. In all sub-analyses described in this note and notes dedicated to our physics analyses (\cite{AnalysisNoteRafal,AnalysisNoteLukasz}) the above cuts were used, unless stated differently.

\section{Quality cuts}\label{enum:TpcQualityCuts}
TPC tracks must satisfy quality criteria:
\begin{enumerate}
\item $N_{\textrm{hits}}^{\textrm{fit}}\geq25$ - at least 25 hits used in the helix fit,
\item $N_{\textrm{hits}}^{\textrm{dE/dx}}\geq15$ - at least 15 hits used in $dE/dx$ calculation,
\item $N_{\textrm{hits}}^{\textrm{fit}}/N_{\textrm{hits}}^{\textrm{poss}}\geq0.52$ - number of hits used in the fit not lest than 52\% of the number of hits potentially left by the particle,
\item $|d_{0}|<1.5$~cm - transverse impact parameter w.r.t. the beamline (see Fig.~\ref{fig:d0sketch}) not larger then 1.5~cm.
\end{enumerate}
\section{Vertex matching}\label{enum:TpcDcaCuts}
TPC tracks must match well to the primary vertex (by definition applies only to primary tracks):
\begin{enumerate}
 \item $\textrm{DCA}(R)<1.5$~cm,
 \item $\textrm{DCA}(z)<1$~cm.
 \end{enumerate}
\section{Kinematic range}\label{enum:TpcKinematicCuts}
TPC tracks must be contained within the kinematic range:
\begin{enumerate}
\item $|\eta|<0.7$,
\item $p_{T}>0.2~\textrm{GeV}/c$.
\end{enumerate}