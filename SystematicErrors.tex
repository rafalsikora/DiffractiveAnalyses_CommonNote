%%===========================================================%%
%%                                                           %%
%%                   SYSTEMATIC ERRORS                       %%
%%                                                           %%
%%===========================================================%%


\chapter{Systematic errors}\label{chap:systematicErrors}
\section{TPC track reconstruction efficiency}\label{sec:tpcSystematics}
\subsection{Embedding (pile-up) effect}\label{subsec:TpcEffSystPileUp}
One major difference between simulation and real data is the presence of pile-up
events. The average number of pile-up tracks in
a triggering event is proportional to the BBC coincidence rate. It is expected that
the difference between simulation and real data drops at lower BBC rates, and the
effects of pile-up tracks could be much reduced by fitting the tracking efficiency as a
function of BBC rate and using the extrapolated value at zero luminosity to compare
with simulation.\newline

%---------------------------
\begin{wrapfigure}{r}{0.45\textwidth}\vspace*{-9pt}
	\centering
	\includegraphics[width=0.45\textwidth, page=5]{graphics/systematicsEfficiency/bbc_and/Out.pdf}
	\caption[Number of events in embedded MC as a function of BBC\_AND rate.]
	{Number of events in embedded MC as a function of BBC\_AND rate. The black and red lines represent the events with \mbox{$<\text{BBC\_AND}>=700$~kHz} and \mbox{$<\text{BBC\_AND}>=1400$~kHz},  respectively.}
	\label{fig:events_bbc_and}%\vspace*{-29pt}
\end{wrapfigure}
%---------------------------
\noindent The embedded MC was divided into two samples due to mean BBC\_AND rate: \mbox{$<\text{BBC\_AND}>=700$~kHz} and \mbox{$<\text{BBC\_AND}>=1400$~kHz}. Next, the track reconstruction efficiency was calculated for those two samples and no-pile-up MC corresponding to them. The difference between TPC track reconstruction efficiences for pile-up and no-pile-up MCs was calculated as:
\begin{equation}
	\Delta\epsilon_{ TPC}^{1400/700\text{ kHz}} = \frac{N_{reco}^{no-pile-up}-N_{reco}^{pile-up}}{N_{gen}}\\
	\label{eq:tpcSyst}
\end{equation}
where:\\
$N_{gen}$-number of MC tracks,\\
$N_{reco}^{no-pile-up}$ - number of reconstructed tracks, matched with MC tracks in no-pile-up MC,\\
$N_{reco}^{pile-up}$ - number of reconstructed tracks, matched with MC tracks in pile-up MC.

The difference between high and low pile-up runs is given by:
\begin{equation}
\Delta\epsilon_{ TPC} =\Delta\epsilon_{ TPC}^{1400\text{ kHz}}-2\cdot\Delta\epsilon_{ TPC}^{700\text{ kHz}}
\label{eq:tpcSystDifference}
\end{equation}
Finally, above difference, shown in  \Cref{fig:systError1Dtpc,fig:systError2Dtpc} for $\pi^\pm$, varies between $2-3\%$ and was taken as systematic uncertainty related to TPC track reconstruction efficiency.
%\vspace{10em}
\begin{figure}[hb]
	\caption[$\pi^\pm$ TPC track reconstruction efficiency as a function of $p_T$ $\left(|\eta|<0.7, |V_z|<80\text{ cm}\right)$ for embedded MC samples with \mbox{$<\text{BBC\_AND}>=700$~kHz} and \mbox{$<\text{BBC\_AND}>=1400$~kHz}]{$\pi^\pm$ TPC track reconstruction efficiency as a function of $p_T$ $\left(|\eta|<0.7, |V_z|<80\text{ cm}\right)$ for embedded MC samples with \mbox{$<\text{BBC\_AND}>=700$~kHz} and \mbox{$<\text{BBC\_AND}>=1400$~kHz}. The efficiences from corresponding no-pile-up MC samples were also shown. Additionally, the differences  from Eq. \ref{eq:tpcSystDifference} were drawn in the bottom of each plot.}
	\label{fig:systError1Dtpc}
	\centering
	\parbox{0.495\textwidth}{
		\centering
		\includegraphics[width=\linewidth,page=1]{graphics/systematicsEfficiency/bbc_and/tpcEffi_d0_1_5_etapt_1.pdf}\\
	}~
	\parbox{0.495\textwidth}{
		\centering
		\includegraphics[width=\linewidth,page=2]{graphics/systematicsEfficiency/bbc_and/tpcEffi_d0_1_5_etapt_1.pdf}\\
	}%
\end{figure}
\begin{figure}[H]
	\caption[The difference $\Delta\epsilon_{ TPC} =\Delta\epsilon_{ TPC}^{1400\text{ kHz}}-2\cdot\Delta\epsilon_{ TPC}^{700\text{ kHz}}$ for $\pi^\pm$ as a function of $p_T$ and $\eta$ $\left(|V_z|<80\text{ cm}\right)$]{The difference $\Delta\epsilon_{ TPC} =\Delta\epsilon_{ TPC}^{1400\text{ kHz}}-2\cdot\Delta\epsilon_{ TPC}^{700\text{ kHz}}$ for $\pi^\pm$ as a function of $p_T$ and $\eta$ $\left(|V_z|<80\text{ cm}\right)$. }
	\label{fig:systError2Dtpc}
	\centering
	\parbox{0.495\textwidth}{
		\centering
		\includegraphics[width=\linewidth,page=1]{graphics/systematicsEfficiency/bbc_and/tpcEffi_d0_1_5_etapt_12D.pdf}\\
	}~
	\parbox{0.495\textwidth}{
		\centering
		\includegraphics[width=\linewidth,page=2]{graphics/systematicsEfficiency/bbc_and/tpcEffi_d0_1_5_etapt_12D.pdf}\\
	}%
\end{figure}

\subsection{Dead material effect on TPC track reconstruction efficiency}\label{sec:deadMaterialSystematics}
The amount of dead material in front of TPC differs up to $20\%$ between data and simulation (see Sec.~\ref{chap:deadMaterial}). First, the~amount of lost particles, $\delta\epsilon_{ TPC}$, due to the interaction with dead material in front of TPC was estimated using  no-pile-up  MC samples. The results for $\pi^-$ in CD are shown in Fig. \ref{fig:dead_materialCD3D}. Then the symmetric systematic uncertainty to the TPC track reconstruction efficiency due to dead material was introduced as $\pm 0.2 \cdot\delta\epsilon_{ TPC}$.
In Fig. \ref{fig:dead_materialCD1D}  the systematic uncertainty is shown for each particle species in CD as a function of $p_T$ $\left(|\eta|<0.7, |V_{z}|<80 \text{ cm}\right)$. 
The results for other particles and SD are shown in Figs. in Appendix \ref{appendix:deadMaterial}.
\begin{figure}[hb]
\caption[The amount of lost $\pi^-$ due to the interaction with dead material in front of TPC as a function of $p_T$, $\eta$ and $z$-vertex in CD]{The amount of lost $\pi^-$ due to the interaction with dead material in front of TPC. Each plot represents the fraction of lost $\pi^-$, $\delta\epsilon_{ TPC}$ ($z$-axis), as a function of true particle pseudorapidity $\eta$ ($y$-axis) and transverse momentum $p_{T}$ ($x$-axis) in single $z$-vertex bin.}\label{fig:dead_materialCD3D}
\centering
\parbox{0.495\textwidth}{
  \centering
  \includegraphics[width=\linewidth,page=1]{graphics/systematicsEfficiency/deadMaterial/secondaries_Unbinned_CD_.pdf}\\
  \includegraphics[width=\linewidth,page=3]{graphics/systematicsEfficiency/deadMaterial/secondaries_Unbinned_CD_.pdf}\\
  \includegraphics[width=\linewidth,page=5]{graphics/systematicsEfficiency/deadMaterial/secondaries_Unbinned_CD_.pdf}\\
  \includegraphics[width=\linewidth,page=7]{graphics/systematicsEfficiency/deadMaterial/secondaries_Unbinned_CD_.pdf}\\
}~
\parbox{0.495\textwidth}{
  \centering
  \includegraphics[width=\linewidth,page=2]{graphics/systematicsEfficiency/deadMaterial/secondaries_Unbinned_CD_.pdf}\\
  \includegraphics[width=\linewidth,page=4]{graphics/systematicsEfficiency/deadMaterial/secondaries_Unbinned_CD_.pdf}\\
  \includegraphics[width=\linewidth,page=6]{graphics/systematicsEfficiency/deadMaterial/secondaries_Unbinned_CD_.pdf}\\
  \includegraphics[width=\linewidth,page=8]{graphics/systematicsEfficiency/deadMaterial/secondaries_Unbinned_CD_.pdf}
}%
\end{figure}
\begin{figure}[hb]\ContinuedFloat
% ~\\[32pt]
\centering
\parbox{0.495\textwidth}{
  \centering
  \includegraphics[width=\linewidth,page=9]{graphics/systematicsEfficiency/deadMaterial/secondaries_Unbinned_CD_.pdf}\\
  \includegraphics[width=\linewidth,page=11]{graphics/systematicsEfficiency/deadMaterial/secondaries_Unbinned_CD_.pdf}\\
  \includegraphics[width=\linewidth,page=13]{graphics/systematicsEfficiency/deadMaterial/secondaries_Unbinned_CD_.pdf}\\
  \includegraphics[width=\linewidth,page=15]{graphics/systematicsEfficiency/deadMaterial/secondaries_Unbinned_CD_.pdf}\\
}~
\parbox{0.495\textwidth}{
  \centering
  \includegraphics[width=\linewidth,page=10]{graphics/systematicsEfficiency/deadMaterial/secondaries_Unbinned_CD_.pdf}\\
  \includegraphics[width=\linewidth,page=12]{graphics/systematicsEfficiency/deadMaterial/secondaries_Unbinned_CD_.pdf}\\
  \includegraphics[width=\linewidth,page=14]{graphics/systematicsEfficiency/deadMaterial/secondaries_Unbinned_CD_.pdf}\\
  \includegraphics[width=\linewidth,page=16]{graphics/systematicsEfficiency/deadMaterial/secondaries_Unbinned_CD_.pdf}
}%
\end{figure}
\begin{figure}[hb]
\caption[The systematic uncertainty to the TPC track reconstruction efficiency due to  amount of dead material in front of TPC using MC samples for CD]{The systematic uncertainty to the TPC track reconstruction efficiency due to  amount of dead material in front of TPC using MC samples for CD. Each plot represents the systematic uncertainty as a~function of true particle $p_T$ $\left(|\eta|<0.7, |V_{z}|<80 \text{ cm}\right)$ for given particle species: $\pi^-$,$\pi^+$, $K^-$, $K^+$, $\bar{p}$ and $p$. It was also calculated for negative and positive particles without identification. }\label{fig:dead_materialCD1D}
\centering
\parbox{0.495\textwidth}{
  \centering
  \includegraphics[width=\linewidth,page=1]{graphics/systematicsEfficiency/deadMaterial/secondaries_Unbinned_CD_1D.pdf}\\
  \includegraphics[width=\linewidth,page=2]{graphics/systematicsEfficiency/deadMaterial/secondaries_Unbinned_CD_1D.pdf}\\
  \includegraphics[width=\linewidth,page=3]{graphics/systematicsEfficiency/deadMaterial/secondaries_Unbinned_CD_1D.pdf}\\
  \includegraphics[width=\linewidth,page=1]{graphics/systematicsEfficiency/deadMaterial/secondaries_Unbinned_Charged_CD1D.pdf}\\
}~
\parbox{0.495\textwidth}{
  \centering
  \includegraphics[width=\linewidth,page=4]{graphics/systematicsEfficiency/deadMaterial/secondaries_Unbinned_CD_1D.pdf}\\
  \includegraphics[width=\linewidth,page=5]{graphics/systematicsEfficiency/deadMaterial/secondaries_Unbinned_CD_1D.pdf}\\
  \includegraphics[width=\linewidth,page=6]{graphics/systematicsEfficiency/deadMaterial/secondaries_Unbinned_CD_1D.pdf}\\
  \includegraphics[width=\linewidth,page=2]{graphics/systematicsEfficiency/deadMaterial/secondaries_Unbinned_Charged_CD1D.pdf}
}%
\end{figure}



\section{TOF matching efficiency}\label{sec:tofSystematics}
\subsection{Embedding (pile-up) effect}\label{sec:tofSystematicsPileUpEffect}
The approach to calculate the systematic uncertainty on TOF matching efficiency related to pile-up was quite similar to the one used for TPC track reconstruction efficiency (Sec.~\ref{subsec:TpcEffSystPileUp}). However, the TOF matching efficiency is conditional and depends on TPC track reconstruction efficiency. Since that, the difference between high and low pile-up runs is given by:
\begin{equation}
\Delta\epsilon_{ TOF}^{1400/700\text{ kHz}}=\frac{N_{TPC-TOF}^{no-pile-up}}{N_{TPC}^{no-pile-up}}-\frac{N_{TPC-TOF}^{pile-up}}{N_{TPC}^{pile-up}}
\label{eq:tofSyst}
\end{equation}
where:\\
$N_{TPC-TOF}^{pile-up}$ - number of reconstructed tracks, matched with MC tracks and TOF hit in pile-up MC,\\
$N_{TPC-TOF}^{no-pile-up}$ - number of reconstructed tracks, matched with MC tracks and TOF hit in no-pile-up MC,\\
$N_{TPC}^{pile-up}$ - number of reconstructed tracks, matched with MC tracks in pile-up MC,\\
$N_{TPC}^{no-pile-up}$ - number of reconstructed tracks, matched with MC tracks in no-pile-up MC.

\noindent Next the difference between high and low pile-up events was calculated withe the formula similar to the one given by Eq. \ref{eq:tpcSystDifference} and is shown in \Cref{fig:systError1Dtof,fig:systError2Dtof}. The origin of  $N_{TPC-TOF}$ increase is not known (it may be due to lack of pile-up in TPC or TOF). Since that, it is impossible to correctly calculate the statistical error for $\Delta\epsilon_{ TOF}$. Nevertheless, $\Delta\epsilon_{ TOF}$ is  smaller than $0.5\%$ and can be neglected in comparison with other systematic uncertainties.
\begin{figure}[hb]
	\caption[$\pi^\pm$ TOF matching efficiency as a function of $p_T$ $\left(|\eta|<0.7, |V_z|<80\text{ cm}\right)$ for embedded MC samples with \mbox{$<\text{BBC\_AND}>=700$~kHz} and \mbox{$<\text{BBC\_AND}>=1400$~kHz}]{$\pi^\pm$ TOF matching efficiency as a function of $p_T$ $\left(|\eta|<0.7, |V_z|<80\text{ cm}\right)$ for embedded MC samples with \mbox{$<\text{BBC\_AND}>=700$~kHz} and \mbox{$<\text{BBC\_AND}>=1400$~kHz}. The efficiences from corresponding no-pile-up MC samples were also shown. Additionally, the differences  from Eq. \ref{eq:tpcSystDifference} were drawn in the bottom of each plot.}
	\label{fig:systError1Dtof}
	\centering
	\parbox{0.495\textwidth}{
		\centering
		\includegraphics[width=\linewidth,page=1]{graphics/systematicsEfficiency/bbc_and/tofEffi_d0_1_5_etapt_1.pdf}\\
	}~
	\parbox{0.495\textwidth}{
		\centering
		\includegraphics[width=\linewidth,page=2]{graphics/systematicsEfficiency/bbc_and/tofEffi_d0_1_5_etapt_1.pdf}\\
	}%
\end{figure}
\begin{figure}[H]
	\caption[The difference $\Delta\epsilon_{ TOF} =\Delta\epsilon_{ TOF}^{1400\text{ kHz}}-2\cdot\Delta\epsilon_{ TOF}^{700\text{ kHz}}$ for $\pi^\pm$ as a function of $p_T$ and $\eta$ $\left(|V_z|<80\text{ cm}\right)$]{The difference $\Delta\epsilon_{ TOF} =\Delta\epsilon_{ TOF}^{1400\text{ kHz}}-2\cdot\Delta\epsilon_{ TOF}^{700\text{ kHz}}$ for $\pi^\pm$ as a function of $p_T$ and $\eta$ $\left(|V_z|<80\text{ cm}\right)$. }
	\label{fig:systError2Dtof}
	\centering
	\parbox{0.495\textwidth}{
		\centering
		\includegraphics[width=\linewidth,page=1]{graphics/systematicsEfficiency/bbc_and/tofEffi_d0_1_5_etapt_12D.pdf}\\
	}~
	\parbox{0.495\textwidth}{
		\centering
		\includegraphics[width=\linewidth,page=2]{graphics/systematicsEfficiency/bbc_and/tofEffi_d0_1_5_etapt_12D.pdf}\\
	}%
\end{figure}



\subsection{Absolute error on TOF efficiency}\label{subsec:tofAbsEffSystAndCorr}

Systematic uncertainty of the TOF efficiency related to the accuracy of the TOF system simulation in STARsim and the TOF efficiency correction derived in Sec.~\ref{sec:tofAbsEffCorr} was estimated by comparing that nominal TOF efficiency with the one obtained with an independent method described below.

In some STAR analyses the TOF hit reconstrucion and matching efficiency is determined from the data with the use of BEMC: real (in-time) tracks are selected based on the fact that they match to BEMC cluster. If they do, the TOF efficiency is calculated as a ratio of number of TOF-matched tracks to number of all tracks. This solution may provide slightly biased efficiency, because the signal in the detector placed behind TOF, such as BEMC, ensures that particle still followed the original helical path past the last hit of the track in TPC (Fig.~\ref{fig:hftEffSketch}). Also, BEMC clusters are more efficiently reconstructed as the energy deposits in the calorimeter increase, which may favor tracks which generated secondaries in front of the BEMC, hence possibly also in front of TOF thus increasing a chance to reconstruct hit in TOF.

To estimate systematic error of the TOF efficiency we decided to calculate efficiency utilizing the TPC tracks containing hits in HFT. The HFT is a group of silicon detectors (PXL, IST, SST) which differ from the gaseous detectors (like TPC) in many aspects. The difference that is most important for this study is the time of response/memory - much shorter in HFT than in TPC. Therefore if the TPC track contain hits in the silicon of HFT it is very probably a real track of particle produced in the proton-proton interaction in the corresponding bunch crossing. With this HFT-tagged tracks we ommit potential bias related to matching with BEMC cluster.

We used the data from st\_ssdmb stream (VPDMB-5-ssd trigger) from the same runs as the data used in our physics analyses. The HFT-tagged tracks were selected as the primary tracks passing the quality cuts \ref{sec:TpcQualityCuts} (only the TPC hits were counted). These tracks were required to contain hits in two HFT layers: IST ans SST, which vastly reduced probability to select an off-time track in TPC (PXL was not used in reconstrucion due to problems in firmware). As shown in Fig.~\ref{fig:zVtxHFT}, the $z_{\text{vtx}}$ coverage of the HFT-tagged tracks is limited to about $\pm20$~cm. We imposes cut on the $z$ position of the vertex $|z_{\text{\text{vtx}}}|<20$~cm to remove tracks from the tails, which generally have large $|\eta|$.

Identification of particles was done using the specific energy loss measured in the TPC ($n^{\sigma}$ variables were used). The following requirements were imposed on $n^{\sigma}$ variables in order to select three species of particles whose tracks were selected for the TOF efficiency analysis:
\begin{itemize}
 \item pions:~~~~~$|n^{\sigma}_{\text{pion}}| < 2$,
 \item kaons:~~~~~$-2 < n^{\sigma}_{\text{proton}} < 2.5$~~~\&\&~~~$|n^{\sigma}_{\text{pion}}| > 3.5$~~~\&\&~~~$|n^{\sigma}_{\text{electron}}| > 3.5$~~~\&\&~~~$|n^{\sigma}_{\text{proton}}| > 3.5$,
 \item protons:~~$-2 < n^{\sigma}_{\text{proton}} < 3$~~~~~\&\&~~~~$|n^{\sigma}_{\text{pion}}| > 3.5$~~~\&\&~~~$|n^{\sigma}_{\text{electron}}| > 3.5$~~~\&\&~~~$|n^{\sigma}_{\text{kaon}}| > 3.5$.
\end{itemize}

%---------------------------
\begin{figure}[b!]%\vspace{-2pt}%
\centering%
\begin{minipage}{.4725\textwidth}%
  \centering%\vspace{11pt}
  \includegraphics[width=0.965\linewidth]{graphics/systematicsEfficiency/TofSyst/effSketch.pdf}%\vspace{-5pt}%
  \caption[Sketch of the track with points in HFT.]%
  {Sketch of the cross section of the central detector and the track reconstructed with points in HFT. Presence of HFT points in a reconstructed track can be used as a tagger of the in-time tracks.}
  \label{fig:hftEffSketch}
\end{minipage}%
\quad\quad%
\begin{minipage}{.4725\textwidth}%
  \centering%
  \includegraphics[width=\linewidth]{graphics/systematicsEfficiency/TofSyst/zVtxHFT.pdf}%\vspace*{-5pt}
  \caption[Distribution of $z$-position of vertices with TPC tracks containing hits in HFT.]
   {Distribution of $z$-position of vertices containing TPC tracks with HFT hits (st\_ssd stream). Open circles represent vertices with tracks with hits in IST or SST, full circles - IST and SST.}
   \label{fig:zVtxHFT}%\vspace*{-29pt} 
\end{minipage}%
\end{figure}%
%---------------------------  

\noindent Selection of pions by cut solely on $n^{\sigma}_{\text{pion}}$ (without additional cuts on $n^{\sigma}$ for kaon, proton and electron hypothesis) is driven by the dominance of pion production over other species and by the fact that the dE/dx of pions overlap with kaons and protons at momenta which are relatively large, hence the TOF efficiency is saturated and the same for all particle species. More sophisticated selection was used for kaons and protons. Figure~\ref{fig:hftTracksNSigmaVsPt} shows the $n^{\sigma}$ variables before and after the selection of kaons (\ref{fig:hftTracksNSigmaKaonVsPt}) and protons (\ref{fig:hftTracksNSigmaProtonVsPt}), where one can find proof that clean samples of these particles were selected, for the price of limited coverage in track $p_{T}$.
%---------------------------
\begin{figure}[t!]
\centering
\parbox{0.4725\textwidth}{
  \centering
  \begin{subfigure}[b]{\linewidth}
                \subcaptionbox{\label{fig:hftTracksNSigmaKaonVsPt}}{\includegraphics[width=\linewidth]{graphics/systematicsEfficiency/TofSyst/NSigmaKaonVsPt.pdf}}\vspace{-5pt}
  \end{subfigure}
}%
\quad\quad%
\parbox{0.4725\textwidth}{
  \centering
  \begin{subfigure}[b]{\linewidth}
                \subcaptionbox{\label{fig:hftTracksNSigmaProtonVsPt}}{\includegraphics[width=\linewidth]{graphics/systematicsEfficiency/TofSyst/NSigmaProtonVsPt.pdf}}\vspace{-5pt}
  \end{subfigure}
}%
\caption[Distribution of $n^{\sigma}$ (kaon and proton) vs. transverse momentum for tracks containing HFT hits.]%
    {Distribution of $n^{\sigma}_{\text{kaon}}$ (\ref{fig:hftTracksNSigmaKaonVsPt}) and $n^{\sigma}_{\text{proton}}$ (\ref{fig:hftTracksNSigmaProtonVsPt}) vs. track $p_{T}$ for tracks containing HFT hits. The insert in each subfigure shows the corresponding $n^{\sigma}$ vs. $p_{T}$ distribution after preselection of tracks of given spiecies (without cut on variable in $y$-axis) according to description provided in the text in preceding page. Dashed magenta lines represent final cuts on corresponding $n^{\sigma}$ quantity used to select tracks of given species.}\label{fig:hftTracksNSigmaVsPt}%
\end{figure}
%---------------------------

%---------------------------
\begin{figure}[b!]\vspace{-34pt}
\centering
\parbox{0.31\textwidth}{
  \centering
  \begin{subfigure}[b]{\linewidth}{
                \subcaptionbox{\label{fig:TPcorrectionTofEff2D}}{\includegraphics[width=\linewidth]{graphics/systematicsEfficiency/TofSyst/TofEffCorrection2D_pion.pdf}\vspace{-12pt}}}
  \end{subfigure} 
} 
\quad
\parbox{0.65\textwidth}{ 
  \centering
		\begin{minipage}[t][0.64\linewidth][t]{\linewidth}\vspace{73pt}
			\caption[Comparison of the TOF eff. correction from tag\&probe method and the difference between TOF eff. calculated using standard method from the HFT-tagged tracks and efficiency from embedded single particle MC.]%
    {Comparison of the TOF efficiency correction obtained with tag\&probe method on CEP $\pi^{+}\pi^{-}$ events (\ref{fig:TPcorrectionTofEff2D}, description in Sec.~\ref{sec:tofAbsEffCorr}) and the difference between TOF efficiency calculated using standard method from the HFT-tagged tracks and efficiency from embedded single particle MC for pions (\ref{fig:tofEffDifference_pion}), kaons (\ref{fig:tofEffDifference_kaon}) and protons (\ref{fig:tofEffDifference_proton}). Yellow hatched area mark empty bins. Dashed horizontal lines represent minimum track $p_{T}$ thresholds used in our analyses: $0.3$~GeV for kaons (green) and $0.4$~GeV for protons (blue).}\label{fig:tofEffSystematics2DComparison}% 
		\end{minipage}
}\\[-25pt]
\parbox{0.31\textwidth}{
  \centering
  \begin{subfigure}[b]{\linewidth}{
                \subcaptionbox{\label{fig:tofEffDifference_pion}}{\includegraphics[width=\linewidth]{graphics/systematicsEfficiency/TofSyst/tofEffDifference_pion.pdf}\vspace{-12pt}}}
  \end{subfigure}
}
\quad
\parbox{0.31\textwidth}{
  \centering
  \begin{subfigure}[b]{\linewidth}{
                \subcaptionbox{\label{fig:tofEffDifference_kaon}}{\includegraphics[width=\linewidth]{graphics/systematicsEfficiency/TofSyst/tofEffDifference_kaon.pdf}\vspace{-12pt}}}
  \end{subfigure}
} 
\quad
\parbox{0.31\textwidth}{
  \centering
  \begin{subfigure}[b]{\linewidth}{
                \subcaptionbox{\label{fig:tofEffDifference_proton}}{\includegraphics[width=\linewidth]{graphics/systematicsEfficiency/TofSyst/tofEffDifference_proton.pdf}\vspace{-12pt}}}
  \end{subfigure}
}

\end{figure}
%---------------------------


From selected sample of pion, kaon and proton tracks the TOF hit reconstruction and matching efficiency was calculated using the standard method - as a ratio of number of tracks matched with TOF and number of all tracks. This efficiency was compared with the efficiency extracted from the zero-bias-embedded single particle MC, calculated for $|z_{\text{vtx}}|<20$~cm and averaged between positive- and negative-charge particles. The result of comparison - the difference between efficiency calculated with HFT-tagged tracks and efficiency from single particle MC, is presented in Fig.~\ref{fig:tofEffSystematics2DComparison} (subfigures \ref{fig:tofEffDifference_pion}-\ref{fig:tofEffDifference_proton}). This difference could be interpreted as a data-driven correction to the TOF efficiency calculated from single particle MC, alternative to correction derived with tag\&probe method on CEP events, described in Sec.~\ref{sec:tofAbsEffCorr}.

The difference between the correction from tag\&probe (Fig.\ref{fig:TPcorrectionTofEff2D}) and alternative correction in Figs.~\ref{fig:tofEffDifference_pion}-\ref{fig:tofEffDifference_proton}, $\Delta\delta\varepsilon_{\text{TOF}}$, can be used as a measure of the uncertainty of the overall TOF efficiency. Aforementioned difference is depicted in Fig.~\ref{fig:tofEffDifference_Delta}. We decided to symmetrize the systematic uncertainty of the TOF efficiency. For this purpose, on top of the correction to the TOF efficiency from CEP tag\&probe method, we add the half of the difference from Fig.~\ref{fig:tofEffDifference_Delta} to the TOF efficiency of corresponding particle type. We then assign a systematic uncertainty of the TOF efficiency to each $(\eta,p_{T})$ bin as an absolute value of the half of that difference, $\frac{1}{2}|\Delta\delta\varepsilon_{\text{TOF}}|$. For high track $p_{T}$, when there are no estimates of $\Delta\delta\varepsilon_{\text{TOF}}$, the value from the last non-empty $p_{T}$ bin (at given $\eta$ bin) is used as a correction, and maximum absolute value among the last 3 non-empty $p_{T}$ bins (at given $\eta$ bin) is used as a systematic uncertainty.

%---------------------------
\begin{figure}[h]%\vspace{-38pt} 
\centering
\parbox{0.31\textwidth}{
  \centering
  \begin{subfigure}[b]{\linewidth}{
                \subcaptionbox{\label{fig:tofEffDifference_Delta_pion}}{\includegraphics[width=\linewidth]{graphics/systematicsEfficiency/TofSyst/tofEffDifference_Delta_pion.pdf}\vspace{-12pt}}}
  \end{subfigure}
}
\quad
\parbox{0.31\textwidth}{
  \centering
  \begin{subfigure}[b]{\linewidth}{
                \subcaptionbox{\label{fig:tofEffDifference_Delta_kaon}}{\includegraphics[width=\linewidth]{graphics/systematicsEfficiency/TofSyst/tofEffDifference_Delta_kaon.pdf}\vspace{-12pt}}}
  \end{subfigure}
} 
\quad
\parbox{0.31\textwidth}{
  \centering
  \begin{subfigure}[b]{\linewidth}{
                \subcaptionbox{\label{fig:tofEffDifference_Delta_proton}}{\includegraphics[width=\linewidth]{graphics/systematicsEfficiency/TofSyst/tofEffDifference_Delta_proton.pdf}\vspace{-12pt}}}
  \end{subfigure}
}
\caption[Difference between the TOF eff. correction estimated with tag\&probe method and with the HFT-tagged tracks.]%
    {Difference between the TOF eff. correction estimated with tag\&probe method on CEP $\pi^{+}\pi^{-}$ events and with the HFT-tagged tracks for pions (\ref{fig:tofEffDifference_Delta_pion}), kaons (\ref{fig:tofEffDifference_Delta_kaon}) and protons (\ref{fig:tofEffDifference_Delta_proton}). Figure \ref{fig:tofEffDifference_Delta_pion} is the difference between \ref{fig:tofEffDifference_pion} and \ref{fig:TPcorrectionTofEff2D}, Figure \ref{fig:tofEffDifference_Delta_kaon} is the difference between \ref{fig:tofEffDifference_kaon} and \ref{fig:TPcorrectionTofEff2D}, and Figure \ref{fig:tofEffDifference_Delta_proton} is the difference between \ref{fig:tofEffDifference_proton} and \ref{fig:TPcorrectionTofEff2D}. Yellow hatched area mark bins which were empty in histograms for HFT-tagged tracks (thus difference is incalculable). Dashed horizontal lines represent minimum track $p_{T}$ thresholds used in our analyses: $0.3$~GeV for kaons (green) and $0.4$~GeV for protons (blue).}\label{fig:tofEffDifference_Delta}% 
\end{figure}
%---------------------------



\section{Roman Pot track recontruction efficiency}\label{sec:rpTrackRecoEffSystematics}


